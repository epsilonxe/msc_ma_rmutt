\cleardoublepage
\setcounter{page}{1}
\pagestyle{headings}

\chapter{ข้อมูลทั่วไป}

\section{รหัสและชื่อหลักสูตร}

\begin{tabular}{p{0.2\textwidth}p{0.7\textwidth}}
	รหัสหลักสูตร & xxxxxx\\
	ชื่อหลักสูตร & \\
	ภาษาไทย & หลักสูตร\thdegree\,สาขาวิชา\thdegreebranch \\
	ภาษาอังกฤษ & \engdegree\,Progam in \engdegreebranch
\end{tabular}

\section{ชื่อปริญญาและสาขาวิชา}

\begin{tabular}{p{0.2\textwidth}p{0.7\textwidth}}
	ชื่อเต็ม (ไทย) & \thdegree\,(\thdegreebranch)\\
	ชื่อย่อ (ไทย) & \thshortdegree\,(\thdegreebranch)\\
	ชื่อเต็ม (อังกฤษ) & \engdegree\,(\engdegreebranch)\\
	ชื่อย่อ (อังกฤษ) & \engshortdegree\,(\engdegreebranch)\\
\end{tabular}

\section{วิชาเอก}
ไม่มี

\section{จำนวนหน่วยกิตที่เรียนตลอดหลักสูตร}
36 หน่วยกิต

\section{รูปแบบของหลักสูตร}

\subsection{รูปแบบ}
หลักสูตรระดับปริญญาโท หลักสูตร 2 ปี

\subsection{แผนการศึกษา}
\subsubsection*{ปริญญาโท}
\begin{itemize}
	\item แผน ก แบบวิชาการ
		\begin{itemize}
		\item แผน ก แบบ ก 1 (ทำวิทยานิพนธ์อย่างเดียว)
		\item แผน ก แบบ ก 2 (ศึกษารายวิชาและทำวิทยานิพนธ์)
		\end{itemize}
	\item แผน ข แบบวิชาชีพ
\end{itemize}

\subsection{กลุ่มหลักสูตร}
\begin{itemize}
	\item กลุ่มสาขาวิชาวิทยาศาสตร์และเทคโนโลยี
\end{itemize}


\subsection{ภาษาที่ใช้}
หลักสูตรจัดการเรียนการสอนเป็นภาษาไทยและภาษาอังกฤษ

\subsection{การรับเข้าศึกษา}
รับนักศึกษาไทยและนักศึกษาต่างชาติ

\subsection{การบูรณาการหลักสูตร(ถ้ามี)}
ไม่มี

\subsection{ความร่วมมือกับสถาบันอื่น}
เป็นหลักสูตรเฉพาะของสถาบันที่จัดการเรียนการสอนโดยตรง

\subsection{การให้ปริญญาแก่ผู้สำเร็จการศึกษา}
ให้ปริญญาเพียงสาขาวิชาเดียว

\section{สถานภาพของหลักสูตรและการพิจารณาอนุมัติ/เห็นชอบหลักสูตร}
\begin{itemize}
	\item หลักสูตรใหม่ พ.ศ. 2569
	\item[$\square$] หลักสูตรปรับปรุง พ.ศ. .....  ปรับปรุงมาจากหลักสูตรปรับปรุง พ.ศ. .....\\
	สภาวิชาการ เห็นชอบในการนำเสนอหลักสูตรต่อสภามหาวิทยาลัยฯ ในการประชุม \\ ครั้งที่ ....... วันที่ ...............
	สภามหาวิทยาลัยฯ ให้ความเห็นชอบหลักสูตร ในการประชุม \\ ครั้งที่ ....... วันที่ .............. \\
	เปิดสอน ภาคการศึกษาที่ 1 ปีการศึกษา 2569
\end{itemize}
\section{ความพร้อมในการเผยแพร่หลักสูตรคุณภาพและมาตรฐาน}
	หลักสูตรจะได้รับการเผยแพร่ว่าเป็นหลักสูตรที่มีคุณภาพและมาตรฐานตามกรอบมาตรฐานคุณวุฒิระดับอุดมศึกษา พ.ศ. 2569 ในปีการศึกษา 2571
\section{อาชีพที่สามารถประกอบได้หลังสำเร็จการศึกษา}
\begin{enumerate}
	\item นักวิชาการ/นักวิจัย ในกลุ่มธุรกิจอุตสาหกรรม/กลุ่มธุรกิจการเงิน/กลุ่มธุรกิจนวัตกรรม หรือองค์กรวิจัยทางวิทยาศาสตร์และเทคโนโลยี
	\item นักพัฒนาซอฟท์แวร์คอมพิวเตอร์ โดยเฉพาะด้านเทคโนโลยี AI
	\item วิศวกรข้อมูล 
	\item นักวิทยาศาสตร์ข้อมูล
	\item นักวิเคราะห์ข้อมูล 
	\item อาชีพอื่นๆ ที่เกี่ยวข้อง
\end{enumerate}

\section{ชื่อ-สกุลตำแหน่ง และคุณวุฒิการศึกษาของอาจารย์ผู้รับผิดชอบหลักสูตร}
{\small
\begin{center}
\begin{longtable}{|p{0.05\textwidth}|p{0.32\textwidth}|p{0.59\textwidth}|}
	\hline
	\multicolumn{1}{|c|}{\textbf{ลำดับ}} &
	\multicolumn{1}{c|}{\textbf{ชื่อ-นามสกุล}} &
	\multicolumn{1}{c|}{\textbf{ผลงานทางวิชาการ}}\\
	\hline
\endhead	


%====================================================

1. &
นายพงศกร สุนทรายุทธ์ \newline 
รองศาสตราจารย์	\newline
ปร.ด.(คณิตศาสตร์ประยุกต์) \newline ม.เทคโนโลยีพระจอมเกล้าธนบุรี 2558 \newline
วท.ม.(คณิตศาสตร์ประยุกต์) \newline ม.เทคโนโลยีพระจอมเกล้าธนบุรี 2553  \newline
วท.บ.(คณิตศาสตร์) \newline ม.เทคโนโลยีพระจอมเกล้าธนบุรี 2551
&
\begin{enumerate}[series=tar]
	\item \underline{P. Sunthrayuth}, K. Kankam, R. Promkam and S. Srisawat, Three novel inertial subgradient extragradient methods for quasi-monotone variational inequalities in Banach spaces, Computational and Applied Mathematics (2024) 43:421 https://doi.org/10.1007/s40314-024-02929-7, (2024: Scopus Q1)
	\item \underline{P. Sunthrayuth}, K. Kankam, R. Promkam and S. Srisawat, Novel inertial methods for fixed point problems in reflexive Banach spaces with applications, Rendiconti del Circolo Matematico di Palermo Series 2, 10.1007/s12215-023-00976-3 (2023: Scopus Q1)
	\item R. Promkam, \underline{P. Sunthrayuth}, S. Kesornprom and E. Tanprayoon, New inertial self-adaptive algorithms for the split common null-point problem: application to data classifications, Journal of Inequalities and Applications, Article number: 136, 2023 (2022: Scopus Q1)
	\item C.C. Okeke, A. Adamu, R. Promkam and \underline{P. Sunthrayuth}, Two-step inertial method for solving split common null point problem with multiple output sets in Hilbert spaces, AIMS Mathematics, 2023, Volume 8, Issue 9: 20201-20222 (2022: Scopus Q1)
	\item L.O. Jolaoso, \underline{P. Sunthrayuth}, P. Cholamjiak and Y.J. Cho, Inertial projection and contraction methods for solving variational inequalities with applications to image restoration problems, Carpathian Journal of Mathematics, Volume 39 (2023), No. 3, Pages 683 – 704 (2022: Scopus Q1)
	\item L.O. Jolaoso, N. Pholasa, \underline{P. Sunthrayuth} and P. Cholamjiak, Inertial-like Bregman projection method for solving systems of variational inequalities, Mathematical Methods in the Applied Sciences, 2023: 1–23, DOI: 10.1002/mma.9479 (2022: Scopus Q1)
	\item Z.B. Wang, \underline{P. Sunthrayuth}, A. Adamu and P. Cholamjiak, Modified accelerated Bregman projection methods for solving quasi-monotone variational inequalities, Optimization (2023), https://doi.org/10.1080/02331934.2023.2187663 (2022: Scopus Q1)
	\item B. Tan, \underline{P. Sunthrayuth}, P. Cholamjiak and Y.J. Cho, Modified inertial extragradient methods for finding minimum-norm solution of the variational inequality problem with applications to optimal control problem, International Journal of Computer Mathematics, 2023, VOL. 100, NO. 3, 525–545 (2022: Scopus Q2)
\end{enumerate}
\\ \hline
&&
\begin{enumerate}[resume*=tar]
	\item M. Arfan, Maha M. A. Lashin, \underline{P. Sunthrayuth}, K. Shah, A. Ullah, K. Iskakova, M. R. Gorji and T. Abdeljawad, On nonlinear dynamics of COVID-19 disease model corresponding to nonsingular fractional order derivative, Medical and Biological Engineering and Computing, volume 60, pages 3169–3185 (2022) (2022: Scopus Q2)
	\item Y. Zhao, E.E. Elattar, M.A. Khan, Fatmawati, M. Asiri and \underline{P. Sunthrayuth}, The dynamics of the HIV/AIDS infection in the framework of piecewise fractional differential equation, Results in Physics, 40 (2022) 105842 (2022: Scopus Q1) 
	\item M.I. Asjad, \underline{P. Sunthrayuth}, M.D. Ikrama, T. Muhammad and A.S. Alshomrani, Analysis of non-singular fractional bioconvection and thermal memory with generalized Mittag-Leffler kernel, Chaos, Solitons and Fractals, 159 (2022) 112090 (2022: Scopus Q1)
	\item L.O. Jolaoso, \underline{P. Sunthrayuth}, P. Cholamjiak and Y.J. Cho, Analysis of two versions of relaxed inertial algorithms with Bregman divergences for solving variational inequalities, Computational and Applied Mathematics, (2022) 41:300 (2022: Scopus Q1)
	\item Y.M. Chu, M.F. Yassen, I. Ahmad, \underline{P. Sunthrayuth} and M.A. Khan, A FRACTIONAL SARS-COV-2 MODEL WITH ATANGANA–BALEANU DERIVATIVE: APPLICATION TO FOURTH WAVE, Fractals, Vol. 30, No. 08, 2240210 (2022) (2022: Scopus Q1)
	\item P. Jailokaa, S. Suantai and \underline{P. Sunthrayuth}, A Self-Adaptive Method for Split Common Null Point Problems and Fixed Point Problems for Multivalued Bregman Quasi-Nonexpansive Mappings in Banach Spaces, Filomat, 36:10 (2022), 3279–3300 (2022: Scopus Q2) 
	\item M. Huang, \underline{P. Sunthrayuth}, A.A. Pasha and M.A. Khan, Numerical solution of stochastic and fractional competition model in Caputo derivative using Newton method, AIMS Mathematics, 2022, Volume 7, Issue 5: 8933-8952 (2022: Scopus Q1) 
	\item J. Yang, P. Cholamjiak and \underline{P. Sunthrayuth}, Weak and strong convergence results for solving monotone variational inequalities in reflexive Banach spaces, Optimization (2022), https://doi.org/10.1080/02331934.2022.2069568 (2022: Scopus Q1)
	\item \underline{P. Sunthrayuth} and T.M. Tuyen, A Generalized Self-Adaptive Algorithm for the Split Feasibility Problem in Banach Spaces, Bulletin of the Iranian Mathematical Society, (2022) 48:1869–1893 (2022: Scopus Q3)
\end{enumerate} 
\\ \hline
&&
\begin{enumerate}[resume*=tar]
	\item T.M. Tuyen, \underline{P. Sunthrayuth}, and N.M. Trang, An inertial self-adaptive algorithm for the generalized split common null point problem in Hilbert spaces, Rendiconti del Circolo Matematico di Palermo Series 2 (2022) 71:537–557 (2022: Scopus Q1)
	\item \underline{P. Sunthrayuth}, L.O. Jolaoso and P. Cholamjiak, New Bregman projection methods for solving pseudo-monotone variational inequality problem, Journal of Applied Mathematics and Computing, volume 68, pages1565–1589 (2022) (2022: Scopus Q1) 
	\item M.I. Asjad, \underline{P. Sunthrayuth}, M.D. Ikrama, T. Muhammad and A.S. Alshomrani, Analysis of non-singular fractional bioconvection and thermal memory with generalized Mittag-Leffler kernel, Chaos, Solitons and Fractals, 159 (2022) 112090 (2022: Scopus Q1) 
	\item J. Yang, P. Cholamjiak and \underline{P. Sunthrayuth}, Modified Tseng’s splitting algorithms for the sum of two monotone operators in Banach spaces, AIMS Mathematics, 6(5): 4873–4900, (2021) (2022: Scopus Q1)
	\item T.M. Tuyen, R. Promkam and \underline{P. Sunthrayuth}, Strong convergence of a generalized forward–backward splitting method in reflexive Banach spaces, Optimization, Volume 71, 2022, 1483-1508 (2022: Scopus Q1) 
	\item P. Cholamjiak, N. Pholasa, S. Suantai and \underline{P. Sunthrayuth}, The generalized viscosity explicit rules for solving variational inclusion problems in Banach spaces, Optimization, Volume 70, 2021, 2607-2633 (2022: Scopus Q1)
	\item \underline{P. Sunthrayuth} and P. Cholamjiak, A modified extragradient method for variational inclusion and fixed point problems in Banach spaces, Applicable Analysis, Volume 100, 2021, 2049-2068 (2022: Scopus Q2)
	\item P. Cholamjiak, S. Suantai and \underline{P. Sunthrayuth}, An explicit parallel algorithm for solving variational inclusion problem and fixed point problem in Banach spaces, Banach Journal of Mathematical Analysis, January 2020, Volume 14, Issue 1, 20–40 (2022: Scopus Q2)
\end{enumerate}
\\ \hline
&&
\begin{enumerate}[resume*=tar]
	\item P. Cholamjiak, S. Suantai and \underline{P. Sunthrayuth}, An iterative method with residual vectors for solving the fixed point and the split inclusion problems in Banach spaces, Computational and Applied Mathematics (2019) 38:12 (2022: Scopus Q1)
	\item S. Suantai, P. Cholamjiak and \underline{P. Sunthrayuth}, Iterative methods with perturbations for the sum of two accretive operators in q-uniformly smooth Banach spaces, RACSAM (2019) 113:203–223 (2022: Scopus Q1) 
\end{enumerate}
\\ \hline
2. &
นายวงศ์วิศรุต เขื่องสตุ่ง \newline 
รองศาสตราจารย์	\newline
ปร.ด.(คณิตศาสตร์ประยุกต์) \newline สถาบันเทคโนโลยีพระจอมเกล้าเจ้าคุณทหารลาดกระบัง 2559 \newline
วท.ม.(คณิตศาสตร์ประยุกต์) \newline สถาบันเทคโนโลยีพระจอมเกล้าเจ้าคุณทหารลาดกระบัง 2555  \newline
วท.บ.(คณิตศาสตร์) \newline สถาบันเทคโนโลยีพระจอมเกล้าเจ้าคุณทหารลาดกระบัง 2553
& 
\begin{enumerate}[series=note]
	\item \underline{W. Khuangsatung}, A.G. Gebrie, C. Suanooma,  2024. “Some New Results on Fixed Points for 𝜛-Distances in Complex-Valued Metric Spaces” Science and Technology Asia Vol.29 No.2 (April-June 2024) 
	\item A. Kheawborisut, \underline{W. Khuangsatung}, 2024. “A modified krasnoselskii-type subgradient extragradient algorithm with inertial effects for solving variational inequality problems and fixed point problem” Nonlinear Functional Analysis and Applications Vol. 29, No. 2(2024), pp. 393-418.
	\item \underline{W. Khuangsatung}, A. Singta, A., and A. Kangtunyakarn, 2024. “A regularization method for solving the G-variational inequality problem and fixed-point problems in Hilbert spaces endowed with graphs” Journal of Inequalities and Applications, 2024(15): 1-25. 
	\item P. Jailoka, C. Suanoom, \underline{W. Khuangsatung} and S. Suantai. Self-adaptive CQ-type algorithms for the split feasibility problem involving two bounded linear operators in Hilbert spaces. Carpathian Journal of Mathematics. 40(1): 77-98, (2024). 
	\item \underline{W. Khuangsatung}, A. Kangtunyakarn. An intermixed method for solving the combination of mixed variational inequality problems and fixed-point problems. J Inequal Appl 2023, 1 (2023). https://doi.org/10.1186/s13660-022-02908-8.
	\item C. Suanoom, \underline{W. Khuangsatung}, T. Bantaojai. On an Open Problem in Complex Valued Rectangular b-Metric Spaces with an Application. Science  Technology Asia. 27(2), 78-83, (2022). 
\end{enumerate} 
\\ \hline
&&
\begin{enumerate}[resume*=note]
	\item \underline{W. Khuangsatung}, A. Kangtunyakarn. Strong Convergence for the Modified Split Monotone Variational Inclusion and Fixed Point Problem. Thai Journal of Mathematics.  20(2),  889–904, (2022).
	\item \underline{W. Khuangsatung}, A. Kangtunyakarn. A Method for Solving the Variational Inequality Problem and Fixed Point Problems in Banach Spaces. Tamkang Journal of Mathematics. 53(1), 23-36,  (2022).
	\item P. Sukprasert, V. Yang, R. Khunprasert, \underline{W. Khuangsatung}, Convergence results for modified SP-iteration in uniformly convex metric spaces, Journal of Mathematics and Computer Science, 26(2), 162-171, (2021).
	\item C. Suanoom, \underline{W. Khuangsatung}. The Convergence Results for an AK-Generalized Nonexpansive Mapping in Hilbert Spaces. Thai Journal of Mathematics. 19(2), 623–634 (2021).
	\item \underline{W. Khuangsatung}, S. Suantai, A. Kangtunyakarn. The Modification of Generalized Mixed Equilibrium Problems for Convergence Theorem of Variational Inequalities Problems and Fixed Point Problems. Thai Journal of Mathematics. 19 (1), 271-296. (2021).
	\item \underline{W. Khuangsatung}, S. Suwannaut. Fixed Point Theorems for a Demicontractive Mapping and Equilibrium Problems in Hilbert Spaces. Communications in Mathematics and Applications 11 (2), 181-198. (2020)
	\item T. Bantaojai, C. Suanoom, \underline{W. Khuangsatung}. The Convergence Theorem for a Square  -Nonexpansive Mapping in a Hyperbolic Space. Thai Journal of Mathematics. 18(3), 1597–1609 (2020).
	\item \underline{W. Khuangsatung}, S. Chan-iam, P. Muangkarn, C. Suanoom, The Rectangular Quasi-Metric Space and Common Fixed Point Theorem for -Contraction and -Kannan Mappings. Thai Journal of Mathematics. 89-101 (2020).
\end{enumerate}
\\ \hline
&&
\begin{enumerate}[resume*=note]
	\item \underline{W. Khuangsatung}, A. Kangtunyakarn, The Method for Solving Fixed Point Problem of G-Nonexpansive Mapping in Hilbert Spaces Endowed with Graphs and Numerical Example. Indian J Pure Appl Math. 51, 155–170 (2020).
	\item \underline{W. Khuangsatung}, P. Jailoka, S. Suantai, An iterative method for solving proximal split feasibility problems and fixed point problems. Comp. Appl. Math. 38, 177 (2019). 
	\item C. Suanoom, K. Sriwichai, C. Klin-Eam, \underline{W. Khuangsatung}. The Finite Family L-Lipschitzian Suzuki-Generalized Nonexpansive Mappings. Communications in Mathematics and Applications. 10(1), 55–69. (2019).
	\item C. Suanoom, K. Sriwichai, C. Klin-Eam, \underline{W. Khuangsatung}. The Generalized -Nonexpansive Mappings and Related Convergence Theorems in Hyperbolic Spaces. Journal of Informatics and Mathematical Sciences. 11 (1), 1-17 (2019)
\end{enumerate}
\\ \hline
3. &
นายรัฐพรหม พรหมคำ \newline 
อาจารย์	\newline
Dr.rer.nat. (Mathematik) \newline Universi\"{a}t W\"{u}rzburg 2562 \newline
วท.ม.(คณิตศาสตร์ประยุกต์) \newline ม.ธรรมศาสตร์ 2552  \newline
วท.บ.(คณิตศาสตร์) \newline ม.ธรรมศาสตร์ 2550

& 
\begin{enumerate}[series=dear]
	\item P. Sunthrayuth, K. Kankam,  \underline{R.Promkam} and S. Srisawat, (2023). Three novel inertial subgradient extragradient methods for quasi-monotone variational inequalities in Banach spaces, Computational and Applied Mathematics (2024) 43:421 https://doi.org/10.1007/s40314-024-02929-7, (2024: Scopus Q1)
	\item P. Sunthrayuth, K. Kankam, \underline{R. Promkam} and S. Srisawat, (2023). Novel inertial methods for fixed point problems in reflexive Banach spaces with applications, Rendiconti del Circolo Matematico di Palermo Series 2, 10.1007/s12215-023-00976-3 (2023: Scopus Q1)
	\item \underline{R. Promkam}, P. Sunthrayuth, S. Kesornprom and E. Tanprayoon, (2023). New inertial self-adaptive algorithms for the split common null-point problem: application to data classifications, Journal of Inequalities and Applications, Article number: 136, 2023 (2022: Scopus Q1)
	\item C.C. Okeke, A. Adamu, \underline{R. Promkam} and P. Sunthrayuth, Two-step inertial method for solving split common null point problem with multiple output sets in Hilbert spaces, AIMS Mathematics, 2023, Volume 8, Issue 9: 20201-20222 (2022: Scopus Q1)
	\item T.M. Tuyen, \underline{R. Promkam} and P. Sunthrayuth, Strong convergence of a generalized forward–backward splitting method in reflexive Banach spaces, Optimization, Volume 71, 2022, 1483-1508 (2022: Scopus Q1)
	\item Y. Tang, \underline{R. Promkam}, P. Cholamjiak, and P. Sunthrayuth, “Convergence Results of Iterative Algorithms for the Sum of Two Monotone Operators in Reflexive Banach Spaces,” Appl Math, Sep. 2021, doi: 10.21136/AM.2021.0108-20.
\end{enumerate}
 \\ \hline

\end{longtable}
\end{center}

\section{สถานที่จัดการเรียนการสอน}
	คณะวิทยาศาสตร์และเทคโนโลยี มหาวิทยาลัยเทคโนโลยีราชมงคลธัญบุรี
\section{สถานการณ์ภายนอกหรือการพัฒนาที่จำเป็นต้องนำมาพิจารณาในการวางแผนหลักสูตร} \label{event_plan}
\subsection{สถานการณ์หรือการพัฒนาทางเศรษฐกิจ}
ในยุคที่เทคโนโลยีสารสนเทศ ปัญญาประดิษฐ์ วิทยาศาสตร์ข้อมูล และนวัตกรรมดิจิทัลเปลี่ยนแปลงอย่างรวดเร็ว การพัฒนาศักยภาพด้านเทคโนโลยีและนวัตกรรมดิจิทัลจึงเป็นปัจจัยสำคัญต่อการเติบโตอย่างยั่งยืน การเปลี่ยนแปลงนี้กระตุ้นให้บุคลากรด้านเทคโนโลยีสารสนเทศต้องพัฒนาตนเองอย่างต่อเนื่องเพื่อให้ทันต่อความก้าวหน้า

แผนอุดมศึกษาระยะยาว 20 ปีของสำนักงานคณะกรรมการการอุดมศึกษาวางเป้าหมายให้อุดมศึกษาเป็นกลไกหลักในการขับเคลื่อนการพัฒนาประเทศ โดยมุ่งเน้นการสร้างนวัตกรรม การวิเคราะห์เชิงรุก และการวิจัย เพื่อปรับปรุงระบบอุดมศึกษาให้มีประสิทธิภาพ สอดคล้องกับเป้าหมายของกระทรวงดิจิทัลเพื่อเศรษฐกิจและสังคมที่ต้องการพัฒนาเศรษฐกิจและสังคมด้วยเทคโนโลยีดิจิทัล โดยเฉพาะการพัฒนาบุคลากรให้พร้อมสำหรับยุคเศรษฐกิจและสังคมดิจิทัล

เพื่อตอบสนองความต้องการของบุคลากรในยุคดิจิทัล หลักสูตร\thdegree\,สาขาวิชา\thdegreebranch{} (หลักสูตร พ.ศ.2569) จึงได้รับการพัฒนาขึ้น โดยเน้นการบริหารจัดการองค์ความรู้อย่างเป็นระบบ การสร้างองค์ความรู้ใหม่ และการประยุกต์ใช้เทคโนโลยีสารสนเทศเพื่อสร้างนวัตกรรมที่นำไปสู่การพัฒนาประเทศอย่างยั่งยืน

ในการพัฒนาหลักสูตรให้มีคุณภาพและตอบสนองความต้องการของประเทศ ได้มีการพิจารณาสถานการณ์ภายนอกที่สำคัญ 3 ประการ ได้แก่ ยุทธศาสตร์ชาติ 20 ปี (พ.ศ. 2561-2580) แผนการศึกษาแห่งชาติ (พ.ศ. 2560-2579) และความต้องการของตลาดแรงงาน
\begin{enumerate}
	\item ยุทธศาสตร์ชาติ 20 ปี: มุ่งหวังให้ประเทศไทยมีความมั่นคง มั่งคั่ง และยั่งยืน โดยเน้นการพัฒนาบุคลากรให้มีคุณภาพและมีทักษะที่จำเป็นในศตวรรษที่ 21 
	\item แผนการศึกษาแห่งชาติ: เน้นการผลิตและพัฒนาบุคลากร การวิจัย และนวัตกรรม เพื่อสร้างขีดความสามารถในการแข่งขันของประเทศ โดยมุ่งให้บุคลากรมีทักษะที่ตรงกับความต้องการของตลาดแรงงาน
\end{enumerate}

ดังนั้นจึงมีความจำเป็นที่จะต้องพัฒนาหลักสูตรที่สามารถสร้างนวัตกรที่เชี่ยวชาญด้านคณิตศาสตร์และเทคโนโลยี สามารถเชื่อมโยงความรู้กับปัญหาต่างๆ และนำความรู้ไปประยุกต์ใช้ได้อย่างมีประสิทธิภาพ
\subsection{สถานการณ์หรือการพัฒนาทางสังคมและวัฒนธรรม}
สถานการณ์หรือการพัฒนาทางสังคมและวัฒนธรรมที่นำมาพิจารณาในการวางแผนหลักสูตร คือ มาตรฐานการอุดมศึกษา พ.ศ. 2561 ซึ่งประกอบด้วยมาตรฐาน 5 ด้านคือ
\begin{enumerate}[label=มาตรฐานที่ \arabic*, leftmargin=5\parindent]
	\item ด้านผลลัพธ์ผู้เรียน
	\item ด้านการวิจัยและนวัตกรรม
	\item ด้านการบริการวิชาการ
	\item ด้านศิลปวัฒนธรรมและความเป็นไทย
	\item ด้านการบริหารจัดการ
\end{enumerate}
		
ในส่วนของมาตรฐานที่ 1 ด้านผลลัพธ์ผู้เรียนนั้น ได้กำหนดผลลัพธ์ผู้เรียนไว้ว่า        เป็นบุคคลที่มีความรู้ ความสามารถ และความรอบรู้ด้านต่าง ๆ มีทักษะการเรียนรู้ตลอดชีวิต เป็นผู้ร่วมสร้างสรรค์นวัตกรรม มีทักษะศตวรรษที่ 21 มีความสามารถในการบูรณาการศาสตร์ต่าง ๆ เพื่อพัฒนาหรือแก้ไขปัญหาสังคม มีคุณลักษณะความเป็นผู้ประกอบการ รู้เท่าทัน     การเปลี่ยนแปลงของสังคมและโลก เป็นพลเมืองที่เข้มแข็ง มีความกล้าหาญทางจริยธรรม    ยึดมั่นในความถูกต้อง รู้คุณค่าและรักษ์ความเป็นไทย
ดังนั้น การพัฒนาหลักสูตร\thdegree\,สาขาวิชา\thdegreebranch หลักสูตรวิทยาศาสตรบัณฑิตสาขาวิชาคณิตศาสตร์ประยุกต์ หลักสูตรใหม่ พ.ศ. 2569 จึงได้นำสถานการณ์ต่าง ๆ ที่สำคัญดังกล่าวข้างต้นมาเป็นกรอบและแนวทางในการพัฒนาหลักสูตร เพื่อให้ได้หลักสูตรที่มีคุณภาพและตอบสนองต่อความต้องการของประเทศ
\section{ผลกระทบจากข้อ \ref{event_plan} ต่อการพัฒนาหลักสูตรและความเกี่ยวข้องกับพันธกิจของมหาวิทยาลัย}
\subsection{การพัฒนาหลักสูตร}
การออกแบบและพัฒนาหลักสูตรได้นำสถานการณ์ในข้อ \ref{event_plan} มาใช้เป็นกรอบและแนวทาง โดยกำหนดปรัชญา วัตถุประสงค์ แผนการศึกษาและแนวทางหลักสูตรฐานสมรรถนะของหลักสูตร จากยุทธศาสตร์ชาติ 20 ปี (พ.ศ. 2561 - 2580) แผนการศึกษาแห่งชาติ        (พ.ศ.2560 – 2579) และมาตรฐานการอุดมศึกษา พ.ศ.2561 กำหนดผลลัพธ์การเรียนรู้ระดับหลักสูตร (PLO) และรายวิชาในหลักสูตรจากความต้องการของตลาดแรงงานในยุคดิจิทัล และจัดทำ มคอ.2 หมวดที่ 4 ผลการเรียนรู้ กลยุทธ์การสอน และการประเมินผล หมวดที่ 7 การประกันคุณภาพหลักสูตร และหมวดที่ 8 การประเมินและปรับปรุงการดำเนินการของหลักสูตรตามแนวทางของมาตรฐานการอุดมศึกษา พ.ศ. 2561
ในการออกแบบและพัฒนาหลักสูตร นอกจากการพิจารณาสถานการณ์ต่าง ๆ ที่กล่าวข้างต้นแล้ว ยังได้นำองค์ความรู้ด้านศึกษาศาสตร์มาประยุกต์ใช้พัฒนาหลักสูตรตามแนวทางการศึกษาที่มุ่งผลลัพธ์ (outcome-based education) โดยกำหนด PLOs ตามอาชีพที่เป็นความต้องการของตลาดแรงงาน แล้วเชื่อมโยงไปสู่ TQF การกระจายความรับผิดชอบสู่รายวิชา กลยุทธ์การสอน และการประเมินกลยุทธ์การสอน 
\subsection{ความเกี่ยวข้องกับพันธกิจของมหาวิทยาลัย}
หลักสูตรได้ถูกพัฒนาให้สอดคล้องตามแผนยุทธศาสตร์ 20 ปี (พ.ศ. 2561 – 2580) ของมหาวิทยาลัยเทคโนโลยีราชมงคลธัญบุรี เพื่อผลิตนักปฏิบัติมืออาชีพชั้นนำด้านวิทยาศาสตร์เทคโนโลยีและนวัตกรรมในระดับประเทศและระดับสากล โดยจัดการศึกษาวิชาชีพระดับอุดมศึกษาบนพื้นฐานวิทยาศาสตร์เทคโนโลยี และนวัตกรรมอย่างมีคุณภาพ มุ่งเน้นให้บัณฑิตสามารถสร้างงานวิจัย สิ่งประดิษฐ์ นวัตกรรมและงานสร้างสรรค์สู่การผลิต เชิงพาณิชย์และสามารถถ่ายทอดเทคโนโลยีเพื่อเพิ่มขีดความสามารถในการแข่งขัน           ของประเทศ อีกทั้งหลักสูตรนี้ยังมีความพร้อมในการให้บริการโครงการบริการวิชาการที่มีแนวคิดเชิงสร้างสรรค์แก่ชุมชนและพื้นที่เป้าหมาย เพื่อการมีอาชีพอิสระและพัฒนาอาชีพสู่ การเพิ่มศักยภาพ และยกระดับคุณภาพชีวิตอย่างยั่งยืน เป็นการพัฒนาการบริหารทรัพยากรมนุษย์เข้าสู่สังคมแห่งการเปลี่ยนแปลงให้สนองต่อยุทธศาสตร์ชาติและสิทธิประโยชน์บนพื้นฐานความสุขและความก้าวหน้าในวิชาชีพ
\section{ความสัมพันธ์กับหลักสูตรที่เปิดสอนในคณะ/ภาควิชาอื่นของมหาวิทยาลัย}
\subsection{กลุ่มวิชา/รายวิชาในหลักสูตรนี้เปิดสอนโดยคณะ/ภาควิชา/หลักสูตรอื่น}
ไม่มี
\subsection{กลุ่มวิชา/รายวิชาในหลักสูตรที่เปิดสอนให้ภาควิชา/หลักสูตรอื่นต้องมาเรียน}
ไม่มี
\subsection{การบริหารจัดการ}
ไม่มี


























