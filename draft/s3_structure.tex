\chapter{ระบบการจัดการศึกษา การดำเนินการ และโครงสร้างหลักสูตร}

\section{ระบบการจัดการศึกษา}
\subsection{ระบบ}
การจัดการศึกษาเป็นระบบทวิภาค ในปีการศึกษาหนึ่งจะแบ่งออกเป็นสองภาคการศึกษาซึ่งเป็นภาคการศึกษาบังคับ มีระยะเวลาศึกษาไม่น้อยกว่าสิบห้าสัปดาห์ต่อหนึ่งภาคการศึกษา 
ทั้งนี้ไม่รวมเวลาสำหรับการสอบด้วย และข้อกำหนดต่าง ๆ เป็นไปตามข้อบังคับมหาวิทยาลัยเทคโนโลยีราชมงคลธัญบุรี ว่าด้วยการศึกษาระดับบัณฑิตศึกษา พ.ศ. 2559  และที่แก้ไขเพิ่มเติม

\subsection{การจัดการศึกษาภาคฤดูร้อน}
มีการจัดการเรียนการสอนภาคการศึกษาฤดูร้อน ทั้งนี้ขึ้นอยู่กับการพิจารณาของคณะกรรมการบริหารหลักสูตร หรือ อาจารย์ผู้รับผิดชอบหลักสูตร  ระยะเวลาการจัดการเรียนการสอน          ไม่น้อยกว่า 8 สัปดาห์ โดยเพิ่มชั่วโมงการศึกษาในแต่ละรายวิชาให้เท่ากับการศึกษาปกติ

\subsection{การเทียบเคียงหน่วยกิตในระบบทวิภาค}
ไม่มี

\section{การดำเนินการหลักสูตร}
\subsection{วัน-เวลาในดำเนินการเรียนการสอน}

\begin{tabular}{p{0.2\textwidth}p{0.7\textwidth}}
ภาคการศึกษาที่ 1 &	เดือนสิงหาคม–ธันวาคม \\
ภาคการศึกษาที่ 2 &	เดือนมกราคม–พฤษภาคม \\
ภาคการศึกษาฤดูร้อน &	เดือนมิถุนายน–กรกฎาคม \\
\end{tabular}



\subsection{คุณสมบัติของผู้เข้าศึกษา}
\begin{enumerate}
	\item เป็นผู้สำเร็จการศึกษาระดับปริญญาตรีหรือเทียบเท่าทางคณิตศาสตร์ สถิติ คอมพิวเตอร์ วิทยาการข้อมูล หรือสาขาวิชาอื่นที่เกี่ยวข้อง
	\item คุณสมบัติอื่น ๆ เป็นไปตามข้อบังคับมหาวิทยาลัยเทคโนโลยีราชมงคลธัญบุรีว่าด้วยการศึกษาระดับบัณฑิตศึกษา พ.ศ. 2559 และที่แก้ไขเพิ่มเติม ซึ่งอยู่ในดุลยพินิจของคณะกรรมการบริหารหลักสูตร
\end{enumerate}

\newpage
\subsection{แผนการรับนักศึกษาและผู้สำเร็จการศึกษาในระยะ 5 ปี}

\renewcommand{\arraystretch}{1.2}
\begin{tabular}{|C{0.35\textwidth}|C{0.1\textwidth}|C{0.1\textwidth}|C{0.1\textwidth}|C{0.1\textwidth}|C{0.1\textwidth}|}
\hline
\multirow{2}{0.35\textwidth}{\hfill{}จำนวนนักศึกษา\hfill\,} & \multicolumn{5}{C{0.5\textwidth}|}{จำนวนนักศึกษาแต่ละปีการศึกษา} \\
\cline{2-6}
 & 2569 & 2570 & 2571 & 2572 & 2573 \\
\hline
ชั้นปีที่ 1 & 10 & 10 & 10 & 10 & 10 \\
\hline
ชั้นปีที่ 2 & - & 10 & 10 & 10 & 10 \\
\hline
รวม & 10 & 20 & 20 & 20 & 20 \\
\hline
คาดว่าจะสำเร็จการศึกษา & - & 10 & 10 & 10 & 10 \\
\hline
\end{tabular}


\subsection{งบประมาณตามแผน}
\subsubsection{งบประมาณรายรับ(หน่วย:บาท)}

\renewcommand{\arraystretch}{1.2}
\begin{tabular}{|p{0.35\textwidth}|C{0.1\textwidth}|C{0.1\textwidth}|C{0.1\textwidth}|C{0.1\textwidth}|C{0.1\textwidth}|}
\hline
\multirow{2}{0.35\textwidth}{รายละเอียดรายรับ} & \multicolumn{5}{c|}{ปีงบประมาณ} \\
\cline{2-6}
 & 2569 & 2570 & 2571 & 2572  & 2573 \\
\hline
1. ค่าบำรุงการศึกษาและค่าลงทะเบียน & 1,000,000 & 1,000,000 & 1,000,000 & 1,000,000 & 1,000,000 \\
\hline
2. เงินอุดหนุนจากรัฐบาล & 300,000 & 300,000 & 300,000 & 300,000 & 300,000 \\
\hline
3. อื่น ๆ (ถ้ามี) & - & - & - & - & - \\
\hline
\textbf{รวมรายรับ} & 1,300,000 & 1,300,000 & 1,300,000 & 1,300,000 & 1,300,000 \\
\hline
\end{tabular}


\subsubsection{งบประมาณรายจ่าย(หน่วย:บาท)}

\renewcommand{\arraystretch}{1.2}
\begin{tabular}{|p{0.35\textwidth}|C{0.1\textwidth}|C{0.1\textwidth}|C{0.1\textwidth}|C{0.1\textwidth}|C{0.1\textwidth}|}
\hline
\multicolumn{1}{|l|}{\textbf{หมวดเงิน}} & \multicolumn{5}{c|}{\textbf{ปีงบประมาณ}} \\
\cline{2-6}
 & 2569 & 2570 & 2571 & 2572  & 2573\\
\hline
\textbf{ก. งบดำเนินการ} & & & & & \\
\hline
1. ค่าใช้จ่ายบุคลากร & 460,000 & 460,000 & 460,000 & 460,000 & 460,000 \\
\hline
2. ค่าใช้จ่ายดำเนินงาน (ไม่รวม 3) & 159,600 & 244,200 & 244,200 & 244,200 & 244,200 \\
\hline
3. ทุนการศึกษา & - & - & - & - & - \\
\hline
4. รายจ่ายระดับมหาวิทยาลัย & 141,000 & 282,000 & 282,000 & 282,000 & 282,000 \\
\hline
\multicolumn{1}{|c|}{\textbf{(รวม ก)}} & 760,600 & 986,200 & 986,200 & 986,200 & 986,200 \\
\hline
\textbf{ข. งบลงทุน} & & & & & \\
\hline
ค่าครุภัณฑ์ & 50,000 & 50,000 & 50,000 & 50,000 & 50,000 \\
\hline
\multicolumn{1}{|c|}{\textbf{(รวม ข)}} & 50,000 & 50,000 & 50,000 & 50,000 & 50,000 \\
\hline
\textbf{รวม (ก) + (ข)} & 810,600 & 1,036,200 & 1,036,200 & 1,036,200 & 1,036,200 \\
\hline
\textbf{กำไร (รายรับ - รายจ่าย)} & 255,444 & 745,675 & 756,831 & 768,322 & 780,158 \\
\hline
\textbf{จำนวนนักศึกษา} & 10 & 20 & 20 & 20 & 20 \\
\hline
\textbf{ค่าใช้จ่ายต่อหัวนักศึกษา} & 54,040 & 34,540 & 34,540 & 34,540 & 34,540 \\
\hline
\multicolumn{1}{|l|}{\textbf{จำนวนนักศึกษาที่จุดคุ้มทุน}} & \multicolumn{5}{c|}{7 คน} \\ 
\hline
\end{tabular}


\subsection{ระบบการศึกษา}
\begin{itemize}
	\item แบบชั้นเรียน
\end{itemize}

\subsection{การเทียบโอนหน่วยกิต รายวิชา และการลงทะเบียนเรียนข้ามสถาบันอุดมศึกษา}
การเทียบโอนหน่วยกิต รายวิชา และการลงทะเบียนเรียนข้ามสถาบันอุดมศึกษา ให้เป็นไปตามข้อบังคับมหาวิทยาลัยเทคโนโลยีราชมงคลธัญบุรี ว่าด้วยการศึกษาระดับบัณฑิตศึกษา พ.ศ. 2559 และที่แก้ไขเพิ่มเติม และระเบียบมหาวิทยาลัยเทคโนโลยีราชมงคลธัญบุรี ว่าด้วยการเทียบโอนผลการเรียน พ.ศ. 2562

\section{หลักสูตรและอาจารย์ผู้สอน}
\subsection{หลักสูตร}

\subsubsection{จำนวนหน่วยกิต}	
รวมตลอดหลักสูตร 36 หน่วยกิต

\subsubsection{โครงสร้างหลักสูตร}
 
\subsubsection*{หลักสูตรแผน ก แบบวิชาการ ก 1} 
แผนการศึกษานี้เป็นการทำวิจัยโดยมีการทำเฉพาะวิทยานิพนธ์ มีโครงสร้างหลักสูตรดังนี้

\begin{enumerate}
	\item วิทยานิพนธ์ \hfill 36 หน่วยกิต
\end{enumerate}

\subsubsection*{หลักสูตรแผน ก แบบวิชาการ ก 2} 
แผนการศึกษานี้เป็นการทำวิจัยโดยมีการทำวิทยานิพนธ์ และศึกษารายวิชาในระดับบัณฑิตศึกษา มีโครงสร้างหลักสูตรดังนี้

\begin{enumerate}
	\item หมวดวิชาบังคับ \hfill 15 หน่วยกิต
	\item หมวดวิชาเลือก \hfill 9 หน่วยกิต
	\item วิทยานิพนธ์ \hfill 12 หน่วยกิต
\end{enumerate}


\subsubsection*{หลักสูตรแผน ข แบบวิชาชีพ} 
แผนการศึกษานี้เน้นการศึกษารายวิชาโดยไม่ทำวิทยานิพนธ์ มีโครงสร้างหลักสูตรดังนี้

\begin{enumerate}
	\item หมวดวิชาบังคับ \hfill 16 หน่วยกิต
	\item หมวดวิชาเลือก \hfill 15 หน่วยกิต
	\item หมวดวิชาการค้นคว้าอิสระ \hfill 6 หน่วยกิต
\end{enumerate}

\clearpage
\subsubsection{รายวิชา}

\subsubsection*{1. หมวดวิชาพื้นฐาน (ไม่นับหน่วยกิต)}

\begin{longtable}{p{0.15\textwidth}p{0.6\textwidth}r{0.15\textwidth}}
09-110-601 & การนำเข้าข้อมูลสู่รูปแบบดิจิทัล & 3(2-2)\\
& Data Digitalization & \\[3mm]
09-110-702 & สัมมนา & 1(0-0)\\
& Seminar & \\[3mm]
\end{longtable}


\par\noindent\textbf{หมายเหตุ} ผู้สำเร็จศึกษาการระดับปริญญาตรีในสาขาที่ไม่เกี่ยวข้องทางคณิตศาสตร์ สถิติ วิทยาการคอมพิวเตอร์ หรือวิทยาการข้อมูล จะต้องศึกษาเพิ่มเติมเพื่อปรับพื้นฐาน โดยประเมินผลเป็น S/U และไม่นับหน่วยกิต


\subsubsection*{2. หมวดวิชาบังคับ จำนวน 15 หน่วยกิต ให้ศึกษาจากรายวิชาต่อไปนี้}

\begin{longtable}{p{0.15\textwidth}p{0.6\textwidth}r{0.15\textwidth}}
09-111-501 & การวิเคราะห์เชิงคณิตศาสตร์ & 3(3-0-9)\\
 & Mathematical Analysis & \\[3mm]
09-111-502 & การหาค่าเหมาะที่สุดเชิงตัวเลข & 3(3-0-9)\\
 & Numerical Optimization & \\[3mm]
09-111-503 & การคำนวณเชิงตัวเลข & 3(3-0-9)\\
 & Numerical Computation & \\[3mm]
09-111-504 & การเล่าเรื่องด้วยข้อมูล & 3(2-2-5)\\
 & Data Storytelling & \\[3mm]
09-111-505 & การจัดการและวิเคราะห์ข้อมูลด้วยไพธอน & 3(2-2-5)\\
 & Data Management and Analysis With Python & \\[3mm]
09-111-606 & สัมมนา & 0(0-3-2)\\
 & Seminar & \\[3mm]
\end{longtable}


\subsubsection*{3. หมวดวิชาเลือก จำนวน 9 หน่วยกิต สำหรับแผน ก แบบวิชาการ ก 2 หรือ จำนวน 15 หน่วยกิต สำหรับแผน ข ให้เลือกศึกษาจากรายวิชาต่อไปนี้}

\subsubsection*{กลุ่มวิชาการวิเคราะห์เชิงคณิตศาสตร์}

\begin{longtable}{p{0.15\textwidth}p{0.6\textwidth}r{0.15\textwidth}}
09-112-601 & การวิเคราะห์เชิงฟังก์ชัน & 3(3-0)\\
& Functional Analysis & \\[3mm]
09-112-702 & ทฤษฎีจุดตรึงและการประยุกต์ & 3(3-0)\\
& Fixed Point Theory and Applications & \\[3mm]
\end{longtable}


\subsubsection*{กลุ่มวิชาคณิตศาสตร์เชิงคำนวณและการประยุกต์}

\begin{longtable}{p{0.15\textwidth}p{0.6\textwidth}r{0.15\textwidth}}
09-113-601 & การหาค่าเหมาะที่สุดสำหรับการเรียนรู้ของเครื่อง & 3(3-0)\\
& Optimization for Machine Learning & \\[3mm]
09-113-702 & คณิตศาสตร์ขั้นสูงสำหรับการเรียนรู้ของเครื่อง & 3(3-0)\\
& Advanced Mathematics for Machine Learning & \\[3mm]
09-113-703 & ขั้นตอนวิธีเชิงตัวเลขสำหรับค่าเหมาะที่สุด  & 3(2-2)\\
& Numerical Algorithm for Optimization & \\[3mm]
09-113-704 & หัวข้อพิเศษของคณิตศาสตร์เชิงคำนวณ  & 3(3-0)\\
& Special Topic in Computational Mathematics for Machine Learning & \\[3mm]
\end{longtable}


\subsubsection*{กลุ่มวิชาวิทยาการข้อมูลและการเรียนรู้ของเครื่อง}

\begin{longtable}{p{0.15\textwidth}p{0.6\textwidth}r{0.15\textwidth}}
09-114-701 & โครงสร้างข้อมูลและอัลกอริทึมสำหรับการเรียนรู้ของเครื่อง & 3(3-0)\\
& Data Structures and Algorithms for Machine Learning & \\[3mm]
09-114-702 & การเรียนรู้ของเครื่อง 2 & 3(2-2)\\
& Machine Learning 2 & \\[3mm]
09-114-703 & การเรียนรู้ของเครื่องแบบเสริมแรง & 3(2-2)\\
& Reinforcement Machine Learning & \\[3mm]
09-114-704 & วิศวกรรมการเรียนรู้ของเครื่อง & 3(2-2)\\
& Machine Learning Engineering & \\[3mm]
09-114-705 & การวิเคราะห์ข้อมูล & 3(2-2)\\
& Data Analytics & \\[3mm]
09-114-706 & การสร้างแผนภาพและการเล่าเรื่องด้วยข้อมูล & 3(2-2)\\
& Data Visualization and Data Storytelling & \\[3mm]
09-114-707 & แบบจำลองภาษาขนาดใหญ่ & 3(2-2)\\
& Large Language Models & \\[3mm]
09-114-708 & การประยุกต์ใช้การเรียนรู้ของเครื่องในงานด้านประมวลผลภาพและสัญญาณ & 3(2-2)\\
& Applications of Machine Learning in Image and Signal Processing & \\[3mm]
09-114-709 & การประยุกต์ใช้การเรียนรู้ของเครื่องในงานด้านสุขภาพ  & 3(2-2)\\
& Applications of Machine Learning in Healthcare & \\[3mm]
09-114-710 & การประยุกต์ใช้การเรียนรู้ของเครื่องในด้านธุรกิจและการเงิน & 3(2-2)\\
& Applications of Machine Learning in Business and Finance & \\[3mm]
09-114-711 & หัวข้อพิเศษของการเรียนรู้ของเครื่อง  & 3(3-0)\\
& Special Topic in Machine Learning & \\[3mm]
\end{longtable}


\subsubsection*{4. หมวดวิชาการค้นคว้าอิสระ จำนวน 6 หน่วยกิต สำหรับแผน ข ให้ศึกษาจากรายวิชาต่อไปนี้}

\begin{longtable}{p{0.15\textwidth}p{0.6\textwidth}r{0.15\textwidth}}
09-115-701 & สารนิพนธ์ & 6(0-0)\\
& Independent Study & \\[3mm]
\end{longtable}


\subsubsection*{5. วิทยานิพนธ์ จำนวน 36 หน่วยกิต สำหรับแผน ก แบบวิชาการ ก 1 หรือ จำนวน 12 หน่วยกิต สำหรับแผน ก แบบวิชาการ ก 2 ให้ศึกษาจากรายวิชาต่อไปนี้}

\begin{longtable}{p{0.15\textwidth}p{0.6\textwidth}r{0.15\textwidth}}
09-115-702 & วิทยานิพนธ์ & 12(0-0)\\
& Thesis & \\[3mm]
09-115-703 & วิทยานิพนธ์ & 36(0-0)\\
& Thesis & \\[3mm]
\end{longtable}


\clearpage
\subsubsection{แผนการศึกษาเสนอแนะ}

\subsubsection*{แผน ก แบบวิชาการ ก 1}


\renewcommand{\arraystretch}{1.4}
\begin{tabular}{|cp{0.58\textwidth}|ccc|}
\hline
\multicolumn{2}{|c|}{ปีที่ 1 / ภาคการศึกษาที่ 1} & \multicolumn{1}{c|}{หน่วยกิต} & \multicolumn{1}{c|}{ทฤษฎี} & \multicolumn{1}{c|}{ปฏิบัติ}  \\ \hline
\multicolumn{1}{|c|}{09-xxx-xxx}  & วิทยานิพนธ์  & \multicolumn{1}{c|}{9}        & \multicolumn{1}{c|}{0}     & \multicolumn{1}{c|}{0}                    \\ \hline
\multicolumn{2}{|c|}{รวม}                        & \multicolumn{3}{c|}{9 หน่วยกิต}                                                                            \\ \hline
\end{tabular}

\vspace{5ex}\par\noindent 
\renewcommand{\arraystretch}{1.4}
\begin{tabular}{|cp{0.58\textwidth}|ccc|}
\hline
\multicolumn{2}{|c|}{ปีที่ 1 / ภาคการศึกษาที่ 2} & \multicolumn{1}{c|}{หน่วยกิต} & \multicolumn{1}{c|}{ทฤษฎี} & \multicolumn{1}{c|}{ปฏิบัติ}  \\ \hline
\multicolumn{1}{|c|}{09-xxx-xxx}  & วิทยานิพนธ์  & \multicolumn{1}{c|}{9}        & \multicolumn{1}{c|}{0}     & \multicolumn{1}{c|}{0}                    \\ \hline
\multicolumn{2}{|c|}{รวม}                        & \multicolumn{3}{c|}{9 หน่วยกิต}                                                                            \\ \hline
\end{tabular}

\vspace{5ex}\par\noindent 
\renewcommand{\arraystretch}{1.4}
\begin{tabular}{|cp{0.58\textwidth}|ccc|}
\hline
\multicolumn{2}{|c|}{ปีที่ 2 / ภาคการศึกษาที่ 1} & \multicolumn{1}{c|}{หน่วยกิต} & \multicolumn{1}{c|}{ทฤษฎี} & \multicolumn{1}{c|}{ปฏิบัติ}  \\ \hline
\multicolumn{1}{|c|}{09-xxx-xxx}  & วิทยานิพนธ์  & \multicolumn{1}{c|}{9}        & \multicolumn{1}{c|}{0}     & \multicolumn{1}{c|}{0}                    \\ \hline
\multicolumn{2}{|c|}{รวม}                        & \multicolumn{3}{c|}{9 หน่วยกิต}                                                                            \\ \hline
\end{tabular}

\vspace{5ex}\par\noindent 
\renewcommand{\arraystretch}{1.4}
\begin{tabular}{|cp{0.58\textwidth}|ccc|}
\hline
\multicolumn{2}{|c|}{ปีที่ 2 / ภาคการศึกษาที่ 2} & \multicolumn{1}{c|}{หน่วยกิต} & \multicolumn{1}{c|}{ทฤษฎี} & \multicolumn{1}{c|}{ปฏิบัติ}  \\ \hline
\multicolumn{1}{|c|}{09-xxx-xxx}  & วิทยานิพนธ์  & \multicolumn{1}{c|}{9}        & \multicolumn{1}{c|}{0}     & \multicolumn{1}{c|}{0}                   \\ \hline
\multicolumn{2}{|c|}{รวม}                        & \multicolumn{3}{c|}{9 หน่วยกิต}                                                                            \\ \hline
\end{tabular}


\newpage
\subsubsection*{แผน ก แบบวิชาการ ก 2}

\renewcommand{\arraystretch}{1.4}
\begin{tabular}{|cp{0.58\textwidth}|ccc|}
\hline
\multicolumn{2}{|c|}{ปีที่ 1 / ภาคการศึกษาที่ 1} & \multicolumn{1}{c|}{หน่วยกิต} & \multicolumn{1}{c|}{ทฤษฎี} & \multicolumn{1}{c|}{ปฏิบัติ}  \\ \hline
\multicolumn{1}{|c|}{09-xxx-xxx}  & วิชาบังคับ  & \multicolumn{1}{c|}{3}        & \multicolumn{1}{c|}{3}     & \multicolumn{1}{c|}{0}                   \\ \hline
\multicolumn{1}{|c|}{09-xxx-xxx}  & วิชาบังคับ  & \multicolumn{1}{c|}{3}        & \multicolumn{1}{c|}{3}     & \multicolumn{1}{c|}{0}                    \\ \hline
\multicolumn{1}{|c|}{09-xxx-xxx}  & วิชาบังคับ  & \multicolumn{1}{c|}{3}        & \multicolumn{1}{c|}{3}     & \multicolumn{1}{c|}{0}                   \\ \hline
\multicolumn{2}{|c|}{รวม}                        & \multicolumn{3}{c|}{9 หน่วยกิต}                                                                            \\ \hline
\end{tabular}

\vspace{5ex}\par\noindent
\renewcommand{\arraystretch}{1.4}
\begin{tabular}{|cp{0.58\textwidth}|ccc|}
\hline
\multicolumn{2}{|c|}{ปีที่ 1 / ภาคการศึกษาที่ 2} & \multicolumn{1}{c|}{หน่วยกิต} & \multicolumn{1}{c|}{ทฤษฎี} & \multicolumn{1}{c|}{ปฏิบัติ}  \\ \hline
\multicolumn{1}{|c|}{09-xxx-xxx}  & วิชาบังคับ  & \multicolumn{1}{c|}{3}        & \multicolumn{1}{c|}{3}     & \multicolumn{1}{c|}{0}                    \\ \hline
\multicolumn{1}{|c|}{09-xxx-xxx}  & วิชาบังคับ  & \multicolumn{1}{c|}{3}        & \multicolumn{1}{c|}{3}     & \multicolumn{1}{c|}{0}                    \\ \hline
\multicolumn{1}{|c|}{09-xxx-xxx}  & วิชาเลือก  & \multicolumn{1}{c|}{3}        & \multicolumn{1}{c|}{3}     & \multicolumn{1}{c|}{0}                   \\ \hline
\multicolumn{2}{|c|}{รวม}                        & \multicolumn{3}{c|}{9 หน่วยกิต}                                                                            \\ \hline
\end{tabular}

\vspace{5ex}\par\noindent
\renewcommand{\arraystretch}{1.4}
\begin{tabular}{|cp{0.58\textwidth}|ccc|}
\hline
\multicolumn{2}{|c|}{ปีที่ 2 / ภาคการศึกษาที่ 1} & \multicolumn{1}{c|}{หน่วยกิต} & \multicolumn{1}{c|}{ทฤษฎี} & \multicolumn{1}{c|}{ปฏิบัติ}  \\ \hline
\multicolumn{1}{|c|}{09-xxx-xxx}  & วิชาเลือก  & \multicolumn{1}{c|}{3}        & \multicolumn{1}{c|}{3}     & \multicolumn{1}{c|}{0}                    \\ \hline
\multicolumn{1}{|c|}{09-xxx-xxx}  & วิชาเลือก  & \multicolumn{1}{c|}{3}        & \multicolumn{1}{c|}{3}     & \multicolumn{1}{c|}{0}                    \\ \hline
\multicolumn{1}{|c|}{09-xxx-xxx}  & วิทยานิพนธ์  & \multicolumn{1}{c|}{3}        & \multicolumn{1}{c|}{0}     & \multicolumn{1}{c|}{0}                    \\ \hline
\multicolumn{2}{|c|}{รวม}                        & \multicolumn{3}{c|}{9 หน่วยกิต}                                                                            \\ \hline
\end{tabular}

\vspace{5ex}\par\noindent
\renewcommand{\arraystretch}{1.4}
\begin{tabular}{|cp{0.58\textwidth}|ccc|}
\hline
\multicolumn{2}{|c|}{ปีที่ 2 / ภาคการศึกษาที่ 2} & \multicolumn{1}{c|}{หน่วยกิต} & \multicolumn{1}{c|}{ทฤษฎี} & \multicolumn{1}{c|}{ปฏิบัติ}  \\ \hline
\multicolumn{1}{|c|}{09-xxx-xxx}  & วิทยานิพนธ์  & \multicolumn{1}{c|}{9}        & \multicolumn{1}{c|}{0}     & \multicolumn{1}{c|}{0}                    \\ \hline
\multicolumn{2}{|c|}{รวม}                        & \multicolumn{3}{c|}{9 หน่วยกิต}                                                                            \\ \hline
\end{tabular}


\newpage
\subsubsection*{แผน ข แบบวิชาชีพ}

\renewcommand{\arraystretch}{1.4}
\begin{tabular}{|cp{0.58\textwidth}|ccc|}
\hline
\multicolumn{2}{|c|}{ปีที่ 1 / ภาคการศึกษาที่ 1} & \multicolumn{1}{c|}{หน่วยกิต} & \multicolumn{1}{c|}{ทฤษฎี} & \multicolumn{1}{c|}{ปฏิบัติ}  \\ \hline
\multicolumn{1}{|c|}{09-xxx-xxx}  & วิชาบังคับ  & \multicolumn{1}{c|}{3}        & \multicolumn{1}{c|}{3}     & \multicolumn{1}{c|}{0}                    \\ \hline
\multicolumn{1}{|c|}{09-xxx-xxx}  & วิชาบังคับ  & \multicolumn{1}{c|}{3}        & \multicolumn{1}{c|}{3}     & \multicolumn{1}{c|}{0}                    \\ \hline
\multicolumn{1}{|c|}{09-xxx-xxx}  & วิชาบังคับ  & \multicolumn{1}{c|}{3}        & \multicolumn{1}{c|}{3}     & \multicolumn{1}{c|}{0}                    \\ \hline
\multicolumn{2}{|c|}{รวม}                        & \multicolumn{3}{c|}{9 หน่วยกิต}                                                                            \\ \hline
\end{tabular}

\vspace{5ex}\par\noindent
\renewcommand{\arraystretch}{1.4}
\begin{tabular}{|cp{0.58\textwidth}|ccc|}
\hline
\multicolumn{2}{|c|}{ปีที่ 1 / ภาคการศึกษาที่ 2} & \multicolumn{1}{c|}{หน่วยกิต} & \multicolumn{1}{c|}{ทฤษฎี} & \multicolumn{1}{c|}{ปฏิบัติ}  \\ \hline
\multicolumn{1}{|c|}{09-xxx-xxx}  & วิชาบังคับ  & \multicolumn{1}{c|}{3}        & \multicolumn{1}{c|}{3}     & \multicolumn{1}{c|}{0}                    \\ \hline
\multicolumn{1}{|c|}{09-xxx-xxx}  & วิชาบังคับ  & \multicolumn{1}{c|}{3}        & \multicolumn{1}{c|}{3}     & \multicolumn{1}{c|}{0}                    \\ \hline
\multicolumn{1}{|c|}{09-xxx-xxx}  & วิชาเลือก  & \multicolumn{1}{c|}{3}        & \multicolumn{1}{c|}{3}     & \multicolumn{1}{c|}{0}                    \\ \hline
\multicolumn{2}{|c|}{รวม}                        & \multicolumn{3}{c|}{9 หน่วยกิต}                                                                            \\ \hline
\end{tabular}

\vspace{5ex}\par\noindent
\renewcommand{\arraystretch}{1.4}
\begin{tabular}{|cp{0.58\textwidth}|ccc|}
\hline
\multicolumn{2}{|c|}{ปีที่ 2 / ภาคการศึกษาที่ 1} & \multicolumn{1}{c|}{หน่วยกิต} & \multicolumn{1}{c|}{ทฤษฎี} & \multicolumn{1}{c|}{ปฏิบัติ}  \\ \hline
\multicolumn{1}{|c|}{09-xxx-xxx}  & วิชาเลือก  & \multicolumn{1}{c|}{3}        & \multicolumn{1}{c|}{3}     & \multicolumn{1}{c|}{0}                    \\ \hline
\multicolumn{1}{|c|}{09-xxx-xxx}  & วิชาเลือก  & \multicolumn{1}{c|}{3}        & \multicolumn{1}{c|}{3}     & \multicolumn{1}{c|}{0}                    \\ \hline
\multicolumn{1}{|c|}{09-xxx-xxx}  & วิชาเลือก & \multicolumn{1}{c|}{3}        & \multicolumn{1}{c|}{0}     & \multicolumn{1}{c|}{0}                    \\ \hline
\multicolumn{2}{|c|}{รวม}                        & \multicolumn{3}{c|}{9 หน่วยกิต}                                                                            \\ \hline
\end{tabular}

\vspace{5ex}\par\noindent
\renewcommand{\arraystretch}{1.4}
\begin{tabular}{|cp{0.58\textwidth}|ccc|}
\hline
\multicolumn{2}{|c|}{ปีที่ 2 / ภาคการศึกษาที่ 2} & \multicolumn{1}{c|}{หน่วยกิต} & \multicolumn{1}{c|}{ทฤษฎี} & \multicolumn{1}{c|}{ปฏิบัติ}  \\ \hline
\multicolumn{1}{|c|}{09-xxx-xxx}  & วิชาเลือก  & \multicolumn{1}{c|}{3}        & \multicolumn{1}{c|}{3}     & \multicolumn{1}{c|}{0}                    \\ \hline
\multicolumn{1}{|c|}{09-xxx-xxx}  & สารนิพนธ์ & \multicolumn{1}{c|}{6}        & \multicolumn{1}{c|}{0}     & \multicolumn{1}{c|}{0}                    \\ \hline
\multicolumn{2}{|c|}{รวม}                        & \multicolumn{3}{c|}{9 หน่วยกิต}                                                                            \\ \hline
\end{tabular}


\clearpage
\subsubsection{คำอธิบายรายวิชา}

\begin{longtable}{p{0.15\textwidth}p{0.6\textwidth}r{0.15\textwidth}}
09-110-601 & ระเบียบวิธีวิจัย & 3(3-0)\\*
 & Research Methodology & \phantom{x} \vspace{3mm} \\*
&  \multicolumn{2}{p{0.75\textwidth}}{กระบวนการทำวิจัย ประเภทการวิจัย การกำหนดปัญหาการวิจัย การทบทวนวรรณกรรม การสร้างข้อคาดการณ์หรือสมมติฐานการวิจัย การเขียนโครงร่างและรายงานการวิจัย การอ้างอิงผลงาน การนำเสนอผลงานวิจัยจรรยาบรรณของนักวิจัย เทคนิควิธีการวิจัยเฉพาะทางการหาค่าเหมาะที่สุดเชิงคำนวณและการเรียนรู้ของเครื่อง ภาษาอังกฤษเพื่อการเขียนงานวิจัย } \vspace{3mm} \\*
&  \multicolumn{2}{p{0.75\textwidth}}{Research process, research types, research problem determination, literature review; conjecture or assumption construction, proposal and research report writing, reference writing, ethics of researchers, research techniques in computational optimization and machine learning, English for research writing.} \vspace{8mm} \\*
09-110-602 & สัมมนา & 1(0-0)\\*
 & Seminar & \phantom{x} \vspace{3mm} \\*
&  \multicolumn{2}{p{0.75\textwidth}}{ศึกษาค้นคว้าบทความที่อยู่ในฐานข้อมูลทางวิทยาศาสตร์ นำเสนอผลการวิจัย วิเคราะห์ อภิปราย สรุปผล ตั้ง คำถามและตอบคำถามจากผู้ร่วมสัมมนาได้ นักศึกษาต้องเขียนรายงานและนำเสนอต่อคณะกรรมการของสาขาวิชา} \vspace{3mm} \\*
&  \multicolumn{2}{p{0.75\textwidth}}{Seminar on articles selected from scientific journals focusing on topics concerning computational optimization and machine learning, the students are obliged to analyze, summaries, give an oral presentation, discuss, and answer the questions, required written report and presentation the selected topics.} \vspace{8mm} \\*
09-111-601 & สถิติและความน่าจะเป็นสำหรับการเรียนรู้ของเครื่อง & 3(3-0)\\*
 & Statistics and Probability for Machine Learning & \phantom{x} \vspace{3mm} \\*
&  \multicolumn{2}{p{0.75\textwidth}}{ทฤษฎีพื้นฐานในสถิติสำหรับการเรียนรู้ของเครื่อง ความน่าจะเป็น ตัวแปรสุ่มแบบไม่ต่อเนื่อง ตัวแปรสุ่มแบบต่อเนื่อง การแจกแจงร่วม ค่าคาดหวัง ค่าคาดหวังแบบมีเงื่อนไข ทฤษฎีลิมิตทางสถิติ การประมาณค่าพารามิเตอร์ การประมาณภาวะน่าจะเป็นสูงสุด วิธีการแบบเบย์ในการประมาณค่าพารามิเตอร์ การทดสอบสมมติฐาน ช่วงความเชื่อมั่น กระบวนการเฟ้นสุ่ม} \vspace{3mm} \\*
&  \multicolumn{2}{p{0.75\textwidth}}{Basic theories in statistics for machine learning, probability, discrete random variables, continuous random variables, joint distributions, expectation, conditional expectation, statistical limit theorems, estimation of parameters, maximum likelihood estimation, Bayesian approach to parameter estimation, hypothesis testing, confidence intervals, random processes.} \vspace{8mm} \\*
09-111-602 & คณิตศาสตร์สำหรับการเรียนรู้ของเครื่อง & 3(3-0)\\*
 & Mathematics for Machine Learning & \phantom{x} \vspace{3mm} \\*
&  \multicolumn{2}{p{0.75\textwidth}}{เมทริกซ์และการดำเนินการบนเมทริกซ์ ระบบสมการเชิงเส้นและการหาผลเฉลย ปริภูมิเวกเตอร์ ความเป็นอิสระเชิงเส้น ฐานหลัก ฐานหลักเชิงตั้งฉาก การแปลงเชิงเส้น ค่าเจาะจงและเวกเตอร์เจาะจง การทำให้เป็นเมทริกซ์ทแยงมุม นอร์ม ผลคูณภายใน ความยาวและระยะทาง ส่วนประกอบเชิงตั้งฉาก การแยกเมทริกซ์ การแยกโชเลสกี การประมาณค่าเมทริกซ์} \vspace{3mm} \\*
&  \multicolumn{2}{p{0.75\textwidth}}{Matrices and matrix algebra, system of linear equation and solving systems of linear equations, vector space, linear Independence, basis, orthonormal basis, linear transformation, eigen value and eigen vector, diagonalization of matrices, norm, inner product, lengths and distances, orthogonal complement, matrix decompositions, Cholesky decomposition, matrix approximation.} \vspace{8mm} \\*
09-111-603 & การเรียนรู้ของเครื่องแบบมีผู้สอน & 3(2-2)\\*
 & Supervised Machine Learning & \phantom{x} \vspace{3mm} \\*
&  \multicolumn{2}{p{0.75\textwidth}}{แนวคิดและหลักการของการเรียนรู้ของเครื่องแบบมีผู้สอน การเตรียมข้อมูล ขั้นตอนการเรียนรู้ เช่น การถดถอยเชิงเส้น การถดถอยเชิงเส้นพหุคูณ การถดถอยโลจีสติกส์ ย่านใกล้เคียงที่สุดเค เบย์อย่างง่าย ต้นไม้ตัดสินใจ การทดสอบประสิทธิภาพตัวแบบ การใช้ตัวแบบไปประยุกต์ใช้ในการพยากรณ์และการจำแนกข้อมูล} \vspace{3mm} \\*
&  \multicolumn{2}{p{0.75\textwidth}}{Concepts and principles of supervised machine learning, data preparation, learning algorithm, such as linear regression, multiple linear regression, logistic regression, k-nearest neighbors, decision tree. Model evaluation, application model to forecasting and data classification.} \vspace{8mm} \\*
09-111-704 & การหาค่าเหมาะที่สุดสำหรับการเรียนรู้ของเครื่อง & 3(3-0)\\*
 & Optimization for Machine Learning & \phantom{x} \vspace{3mm} \\*
&  \multicolumn{2}{p{0.75\textwidth}}{ทฤษฎีพื้นฐานของปัญหาการหาค่าเหมาะที่สุด ปัญหาการหาค่าเหมาะที่สุดแบบมีข้อจำกัด ปัญหาการหาค่าเหมาะที่สุดแบบไม่มีข้อจำกัด ปัญหาการหาค่าเหมาะที่สุดแบบปรับเรียบ และไม่ปรับเรียบ อัลกอริทึมค่าเหมาะที่สุดอันดับหนึ่ง อัลกอริทึมค่าเหมาะที่สุดอันดับสอง อัลกอริทึมเคลื่อนลงตามความชันสโตแคสติก อัลกอริทึมเคลื่อนลงแบบใกล้เคียง การใช้โปรแกรมไพธอนในการพัฒนาอัลกอริทึม} \vspace{3mm} \\*
&  \multicolumn{2}{p{0.75\textwidth}}{Basic theories of optimization, constrained optimization, unconstrained optimization, smooth and nonsmooth optimization, first-order optimization algorithms, second-order optimization algorithms, stochastic gradient descent algorithm, proximal gradient method, algorithm implementation in Python.} \vspace{8mm} \\*
09-111-705 & โครงสร้างข้อมูลและอัลกอริทึมสำหรับการเรียนรู้ของเครื่อง & 3(3-0)\\*
 & Data Structures and Algorithms for Machine Learning & \phantom{x} \vspace{3mm} \\*
&  \multicolumn{2}{p{0.75\textwidth}}{แนวคิดของโครงสร้างข้อมูล โครงสร้างข้อมูลเบื้องต้น การดำเนินการบนโครงสร้างข้อมูล เทคนิคการค้นและเทคนิคการเรียงลำดับ การวิเคราะห์โครงสร้างข้อมูล การประยุกต์และอัลกอริทึมสำหรับการแก้ปัญหาในกระบวนการของการเรียนรู้ของเครื่อง} \vspace{3mm} \\*
&  \multicolumn{2}{p{0.75\textwidth}}{Concepts of data structures, fundamental data structures, operations of data structures, basic searching and sorting techniques, data structure analysis, applications and problem solving algorithms for machine learning processes.} \vspace{8mm} \\*
09-112-601 & การวิเคราะห์เชิงฟังก์ชัน & 3(3-0)\\*
 & Functional Analysis & \phantom{x} \vspace{3mm} \\*
&  \multicolumn{2}{p{0.75\textwidth}}{ปริภูมิเมตริก ปริภูมินอร์ม ปริภูมิบานาค ตัวดำเนินการเชิงเส้น ปริภูมิผลคูณภายในและปริภูมิฮิลเบิร์ต ทฤษฎีบทฮาห์น-บานาค ทฤษฎีบทการมีขอบเขตแบบเอกรูป ปริภูมิคู่กัน} \vspace{3mm} \\*
&  \multicolumn{2}{p{0.75\textwidth}}{Metric space, normed space, Banach spaces, linear operator, inner product and Hilbert spaces, Hahn-Banach theorem, uniform boundedness theorem, dual space.} \vspace{8mm} \\*
09-112-702 & ทฤษฎีจุดตรึงและการประยุกต์ & 3(3-0)\\*
 & Fixed Point Theory and Applications & \phantom{x} \vspace{3mm} \\*
&  \multicolumn{2}{p{0.75\textwidth}}{ทฤษฎีจุดตรึงในปริภูมิเมตริก ทฤษฎีจุดตรึงในปริภูมิฮิลเบิร์ต ทฤษฎีจุดตรึงในปริภูมิบานาค การทำซ้ำเพื่อหาจุดตรึง} \vspace{3mm} \\*
&  \multicolumn{2}{p{0.75\textwidth}}{Fixed point theory in metric space, fixed point theory in Hilbert space, fixed point theory in Banach space, fixed point iteration.} \vspace{8mm} \\*
09-113-601 & คณิตศาสตร์ขั้นสูงสำหรับการเรียนรู้ของเครื่อง & 3(3-0)\\*
 & Advanced Mathematics for Machine Learning & \phantom{x} \vspace{3mm} \\*
&  \multicolumn{2}{p{0.75\textwidth}}{แคลคูลัสสำหรับการเรียนรู้ของเครื่อง: ฟังก์ชันหลายตัวแปร ลิมิตและความต่อเนื่อง อนุพันธ์ของฟังก์ชันหลายตัวแปร กฎลูกโซ่ จาโคเบียน เกรเดียนของฟังก์ชันค่าเวกเตอร์ เกรเดียนของเมทริกซ์ อนุพันธ์อันดับสูง ทฤษฎีของเทย์เลอร์ โครงข่ายประสาทเทียม ฟังก์ชันกระตุ้น ฟังก์ชันการสูญเสีย อัลกอริทึมเพิ่มประสิทธิภาพเพื่อปรับปรุงความแม่นยำของเครือข่ายประสาทเทียม สมการเชิงอนุพันธ์สามัญและทฤษฎีพื้นฐาน การพยากรณ์ข้อมูลด้วยแบบจำลองของสมการเชิงอนุพันธ์สามัญ สมการเชิงอนุพันธ์ย่อยและทฤษฎีพื้นฐาน แบบจำลองของสมการเชิงอนุพันธ์ย่อย การพยากรณ์ข้อมูลด้วยแบบจำลองของสมการเชิงอนุพันธ์ย่อย} \vspace{3mm} \\*
&  \multicolumn{2}{p{0.75\textwidth}}{Calculus for machine learning: multivariable functions, limit and continuity, derivative of multivariable functions, chain rule, Jacobian, gradient of vector-valued function, gradient of matrices, high order derivatives and Taylor’s Theorem. Neural networks: activation functions, loss function, backpropagation algorithm. Ordinary differential equations (ODEs) and basic theory: Prediction with model of ODEs. Partial differential equations (PDEs) and basic theory: Model of PDEs, prediction with model of PDEs.} \vspace{8mm} \\*
09-113-702 & ขั้นตอนวิธีเชิงตัวเลขสำหรับค่าเหมาะที่สุด  & 3(2-2)\\*
 & Numerical Algorithm for Optimization & \phantom{x} \vspace{3mm} \\*
&  \multicolumn{2}{p{0.75\textwidth}}{ทฤษฎีค่าเหมาะที่สุดในปริภูมิฮิลเบิร์ตและปริภูมิบานาค ขั้นตอนวิธีสำหรับจุดตรึง วิธีอินเนอร์เทียล ปัญหาอสมการเชิงแปรผัน ปัญหาดุลยภาพ ปัญหารวมแบบกึ่ง ปัญหาเป็นไปได้แบบแยก การสร้างขั้นตอนวิธีเพื่อหาผลเฉลยของปัญหาค่าเหมาะที่สุด} \vspace{3mm} \\*
&  \multicolumn{2}{p{0.75\textwidth}}{Optimization in Hilbert and Banach spaces, algorithm for fixed point, inertial method, variational inequality problem, equilibrium problem, quasi-inclusion problem, split feasibility problem, construction algorithm for solution of optimization problems.} \vspace{8mm} \\*
09-113-603 & การตัดสินใจอย่างชาญฉลาดด้วยกำหนดการวิจัยดำเนินงาน & 3(2-2)\\*
 & Intelligence Decision Making with Operation Research & \phantom{x} \vspace{3mm} \\*
&  \multicolumn{2}{p{0.75\textwidth}}{กำหนดการเชิงเส้น: ตัวแบบกำหนดการเชิงเส้นและการหาผลเฉลยโดยวิธีกราฟ หลักการของวิธีซิมเพล็กซ์ ปัญหาควบคู่และการวิเคราะห์ความไว หลักการของวิธีซิมเพล็กซ์ควบคู่ ปัญหาการขนส่ง ปัญหาเครือข่าย ปัญหาการลงทุน กำหนดการพลวัต การพัฒนาแอพพลิเคชั่นช่วยตัดสินใจในการแก้ปัญหาโดยใช้การวิจัยดำเนินงาน} \vspace{3mm} \\*
&  \multicolumn{2}{p{0.75\textwidth}}{Linear programming: linear programming model and graphical solution, principles of the simplex method, dual problem and sensitivity analysis, principles of the dual simplex method, transportation models and its applications, logistics problems, network problems, investment problems, dynamic programming, development to operation research as an application to assist decision making.} \vspace{8mm} \\*
09-113-704 & หัวข้อพิเศษของคณิตศาสตร์เชิงคำนวณ  & 3(3-0)\\*
 & Special Topic in Computational Mathematics for Machine Learning & \phantom{x} \vspace{3mm} \\*
&  \multicolumn{2}{p{0.75\textwidth}}{ความก้าวหน้าเชิงทฤษฎีและการประยุกต์คณิตศาสตร์เชิงคำนวณสำหรับการเรียนรู้ของเครื่อง เรื่องเฉพาะแปรเปลี่ยนตามความสนใจของผู้สอนและนักศึกษา ซึ่งสอดคล้องกับ ความก้าวหน้าทางวิทยาศาสตร์และเทคโนโลยีในปัจจุบัน} \vspace{3mm} \\*
&  \multicolumn{2}{p{0.75\textwidth}}{Theoretical advances and applications of computational mathematics for machine learning, specific topics based on contemporary advances in science and technology, interests of individual instructor and students.} \vspace{8mm} \\*
09-114-601 & การเรียนรู้ของเครื่องแบบไม่มีผู้สอน & 3(2-2)\\*
 & Unsupervised Machine Learning & \phantom{x} \vspace{3mm} \\*
&  \multicolumn{2}{p{0.75\textwidth}}{แนวคิดและหลักการของการเรียนรู้ของเครื่องแบบไม่มีผู้สอน การจัดกลุ่มข้อมูล การจัดกลุ่มแบบค่าเฉลี่ยเค การหากฎความสัมพันธ์ ปัจจัยสนับสนุนนและปัจจัยความเชื่อมั่น ขั้นตอนวิธีแบบนิรนัย การใช้ตัวแบบไปประยุกต์ใช้ในการจัดกลุ่มข้อมูล} \vspace{3mm} \\*
&  \multicolumn{2}{p{0.75\textwidth}}{Concepts and principles of unsupervised machine learning, data clustering, k-means clustering. Association data: support and confident factors, apriori algorithm, application model to data clustering.} \vspace{8mm} \\*
09-114-702 & การเรียนรู้ของเครื่องแบบเสริมแรง & 3(2-2)\\*
 & Reinforcement Machine Learning & \phantom{x} \vspace{3mm} \\*
&  \multicolumn{2}{p{0.75\textwidth}}{แนวคิดและหลักการของการเรียนรู้ของเครื่องแบบเสริมแรง การทดลองของตัวแทน การให้รางวัลและการลงโทษ การเรียนรู้คิว โครงข่ายคิวเชิงลึก การตัดสินใจแบบต่อเนื่องกัน การควบคุมหุ่นยนต์ การจัดการจราจร และการวางกลยุทธ์ในเกม} \vspace{3mm} \\*
&  \multicolumn{2}{p{0.75\textwidth}}{Concepts and principles of reinforcement machine learning, agent, reward and punishment, Q-learning, deep Q-network, sequential decision-making, robot control, traffic control and gameplay strategy.} \vspace{8mm} \\*
09-114-703 & การเรียนรู้เชิงลึกและการประยุกต์   & 2(2-0)\\*
 & Deep Learning and Applications & \phantom{x} \vspace{3mm} \\*
&  \multicolumn{2}{p{0.75\textwidth}}{พื้นฐานโครงข่ายประสาทเทียม การเรียนรู้แบบป้อนหน้าและการแพร่กลับ การใช้งานเฟรมเวิร์กเช่น TensorFlow หรือ PyTorch โครงข่ายประสาทแบบคอนโว ลูชัน โครงข่ายประสาทแบบหมุนเวียน เทคนิคป้องกันการฟิตเกิน การปรับจูนไฮ เปอร์พารามิเตอร์ การประยุกต์ใช้ในการประมวลผลภาพ และการประมวลผลภาษาธรรมชาติ} \vspace{3mm} \\*
&  \multicolumn{2}{p{0.75\textwidth}}{Fundamentals of neural networks, feed-forward and backpropagation learning, implementation with frameworks such as tensor flow or PyTorch, convolutional neural networks, recurrent neural networks, techniques to prevent overfitting, hyperparameter tuning, applications in image processing, and natural language processing.} \vspace{8mm} \\*
09-114-704 & วิศวกรรมการเรียนรู้ของเครื่อง & 2(2-0)\\*
 & Machine Learning Engineering & \phantom{x} \vspace{3mm} \\*
&  \multicolumn{2}{p{0.75\textwidth}}{หลักการของวิศวกรรมการเรียนรู้ของเครื่อง กระบวนการและการออกแบบการเรียนรู้ของเครื่อง การพัฒนาและการปรับใช้การเรียนรู้ของเครื่อง การดึงข้อมูลสำหรับการเรียนรู้ของเครื่อง การพัฒนาเว็บแอพพลิเคชั่นเพื่องานการเรียนรู้ของเครื่อง } \vspace{3mm} \\*
&  \multicolumn{2}{p{0.75\textwidth}}{Principles of Machine learning engineering, process and design of machine learning, machine learning development and deployment, data scraping for machine learning, machine learning web application development.} \vspace{8mm} \\*
09-114-605 & การวิเคราะห์ข้อมูล & 2(2-0)\\*
 & Data Analytics & \phantom{x} \vspace{3mm} \\*
&  \multicolumn{2}{p{0.75\textwidth}}{เทคโนโลยีในการจัดการเก็บและวิเคราะห์ข้อมูลขนาดใหญ่ การจัดกลุ่มข้อมูล การวิเคราะห์เส้นทาง การวิเคราะห์ปัจจัย และตัวแบบสมการโครงสร้าง การใช้โปรแกรมสำเร็จรูปหรือภาษาโปรแกรมในการวิเคราะห์} \vspace{3mm} \\*
&  \multicolumn{2}{p{0.75\textwidth}}{Technologies used in manipulating, storing, and analyzing big data, clustering data, path analysis, factor analysis and structural equation model, utilization of software packages or programming language for analysis.} \vspace{8mm} \\*
09-114-606 & การทำให้เห็นข้อมูล & 2(2-0)\\*
 & Data Visualization & \phantom{x} \vspace{3mm} \\*
&  \multicolumn{2}{p{0.75\textwidth}}{ระบบพิกัดและแกน สเกลสี การแสดงภาพจำนวน การแสดงภาพการกระจาย การแสดงภาพสัดส่วน การแสดงภาพอนุกรมเวลา การแสดงภาพแนวโน้ม การแสดงภาพความไม่แน่นอน หลักการออกแบบภาพ เช่น หลักการของน้ำหมึกตามสัดส่วน และการจัดการจุดที่ทับซ้อนกัน} \vspace{3mm} \\*
&  \multicolumn{2}{p{0.75\textwidth}}{Coordinate systems and axes, color scales, visualizing amounts, visualizing distributions, visualizing proportions, visualizing time series, visualizing trends, visualizing uncertainty, principles of figure design, the principle of proportional ink, and handling overlapping points.} \vspace{8mm} \\*
09-114-707 & แบบจำลองภาษาขนาดใหญ่ & 2(2-0)\\*
 & Large Language Model & \phantom{x} \vspace{3mm} \\*
&  \multicolumn{2}{p{0.75\textwidth}}{ความหมายและลักษณะของแบบจำลองภาษาขนาดใหญ่ การประมวลผลข้อความล่วงหน้า การวิเคราะห์ความหมายและไวยากรณ์ การแยกคุณสมบัติ รูปแบบ TF-IDF รูปแบบของคำและเอกสารที่เป็นเวกเตอร์ การรู้จําและการสังเคราะห์เสียง การประยุกต์ LLM ในการจำแนก การสกัดข้อมูล การขุดและการดึงข้อความ การประยุกต์การเรียนรู้เชิงลึกใน LLM โครงข่ายประสาทเทียมใน LLM} \vspace{3mm} \\*
&  \multicolumn{2}{p{0.75\textwidth}}{Definition and characteristic of large language model (LLM), text pre-processing, semantic and grammatical analysis, features extraction, TF-IDF model, word and document vectors, speech recognition and synthesis, application of LLM to classification, information extraction, text mining and information retrieval, application of deep learning to LLM, neural networks in LLM.} \vspace{8mm} \\*
09-114-708 & การประยุกต์ใช้การเรียนรู้ของเครื่องในงานด้านประมวลผลภาพและสัญญาณ & 2(2-0)\\*
 & Applications of Machine Learning in Image and Signal Processing & \phantom{x} \vspace{3mm} \\*
&  \multicolumn{2}{p{0.75\textwidth}}{การประยุกต์ใช้การเรียนรู้ของเครื่องในงานด้านประมวลผลภาพและสัญญาณ เช่น การใช้การเรียนรู้ของเครื่องสำหรับวิเคราะห์ภาพถ่ายและสัญญาณ ปัญหาภาพเบลอ ภาพเบลอแบบเกาส์เซียน ภาพเบลอแบบเคลื่อนไหว ภาพเบลอแบบหลุดความสนใจ การกำจัดสัญญาณรบกวน การบีบอัดภาพและสัญญาณ เทคนิคการประมวลผลภาพล่วงหน้าและการเพิ่มภาพ } \vspace{3mm} \\*
&  \multicolumn{2}{p{0.75\textwidth}}{Applications of Machine Learning in Image and signal processing: machine learning for analyze image and signal. Image deblurring problem, Gaussian blur, motion blur, out of focus blur, noise reduction, image and signal compression, image preprocessing and augmentation techniques.} \vspace{8mm} \\*
09-114-709 & การประยุกต์ใช้การเรียนรู้ของเครื่องในงานด้านการแพทย์  & 2(2-0)\\*
 & Applications of Machine Learning in Medical & \phantom{x} \vspace{3mm} \\*
&  \multicolumn{2}{p{0.75\textwidth}}{การประยุกต์ใช้การเรียนรู้ของเครื่องในด้านการแพทย์ เช่น การใช้การเรียนรู้ของเครื่องสำหรับการวินิจฉัยโรค การใช้การเรียนรู้ของเครื่องสำหรับการระบุชนิดของโรค การวิเคราะห์ข้อมูลขนาดใหญ่ทางการแพทย์} \vspace{3mm} \\*
&  \multicolumn{2}{p{0.75\textwidth}}{Application of machine learning in the medical: machine learning for disease diagnosis and identifying the types of diseases, data analytic of medical big data.} \vspace{8mm} \\*
09-114-710 & การประยุกต์ใช้การเรียนรู้ของเครื่องในด้านธุรกิจและการเงิน & 2(2-0)\\*
 & Applications of Machine Learning in Business and Finance & \phantom{x} \vspace{3mm} \\*
&  \multicolumn{2}{p{0.75\textwidth}}{การประยุกต์ใช้การเรียนรู้ของเครื่องในงานด้านธุรกิจและการเงิน เช่น การใช้การเรียนรู้ของเครื่องสำหรับการวิเคราะห์แนวโน้มของหุ้น การใช้การเรียนรู้ของเครื่องสำหรับการตัดสินใจทางธุรกิจ} \vspace{3mm} \\*
&  \multicolumn{2}{p{0.75\textwidth}}{Application of machine learning in business and finance: machine learning for stock trend analysis and making business decisions.} \vspace{8mm} \\*
09-114-711 & หัวข้อพิเศษของการเรียนรู้ของเครื่อง  & 2(2-0)\\*
 & Special Topic in Machine Learning & \phantom{x} \vspace{3mm} \\*
&  \multicolumn{2}{p{0.75\textwidth}}{ความก้าวหน้าของการประยุกต์ใช้การเรียนรู้ของเครื่อง เรื่องเฉพาะแปรเปลี่ยนตามความสนใจของผู้สอนและนักศึกษา ซึ่งสอดคล้องกับ ความก้าวหน้าทางวิทยาศาสตร์และเทคโนโลยีในปัจจุบัน } \vspace{3mm} \\*
&  \multicolumn{2}{p{0.75\textwidth}}{Theoretical advances for applications of machine learning, specific topics based on contemporary advances in science and technology, interests of individual instructor and students.} \vspace{8mm} \\*
09-115-701 & สารนิพนธ์ & 6(0-0)\\*
 & Independent Study & \phantom{x} \vspace{3mm} \\*
&  \multicolumn{2}{p{0.75\textwidth}}{นักศึกษาที่จะทำสารนิพนธ์จะต้องผ่านวิชาบังคับในหลักสูตรอย่างน้อย 10 หน่วยกิต หรือตามที่ภาควิชาฯ เห็นชอบ หัวข้อสารนิพนธ์จะต้องได้รับการเห็นชอบจากอาจารย์ที่ปรึกษาและภาควิชาฯ และต้องเป็นหัวข้อที่เกี่ยวข้องกับเนื้อหาวิชาที่ได้เรียนมาในหลักสูตร } \vspace{3mm} \\*
&  \multicolumn{2}{p{0.75\textwidth}}{Students are expected to complete at least 10 credits of study with approval from advisors. This must be related with the subject or knowledge, which students have learned from the courses.} \vspace{8mm} \\*
09-115-702 & วิทยานิพนธ์ & 12(0-0)\\*
 & Thesis & \phantom{x} \vspace{3mm} \\*
&  \multicolumn{2}{p{0.75\textwidth}}{นักศึกษาต้องทำวิทยานิพนธ์ภายใต้คำแนะนำของอาจารยที่ปรึกษาที่ได้รับการแต่ง ตั้งโดยบัณฑิตวิทยาลัย นักศึกษาต้องปฏิบัติตามกูฏและข้อบังคับที่กำหนดโดยภาควิชาและบัณฑิตวิทยาลัยอย่างเคร่งครัด} \vspace{3mm} \\*
&  \multicolumn{2}{p{0.75\textwidth}}{Students are required to conduct a thesis under supervision of advisors appointed by graduate college. Rules and regulations for under taking thesis set by students’ department and graduate college must be observed strictly.} \vspace{8mm} \\*
09-115-703 & วิทยานิพนธ์ & 36(0-0)\\*
 & Thesis & \phantom{x} \vspace{3mm} \\*
&  \multicolumn{2}{p{0.75\textwidth}}{นักศึกษาต้องทำวิทยานิพนธ์ภายใต้คำแนะนำของอาจารยที่ปรึกษาที่ได้รับการแต่ง ตั้งโดยบัณฑิตวิทยาลัย นักศึกษาต้องปฏิบัติตามกูฏและข้อบังคับที่กำหนดโดยภาควิชาและบัณฑิตวิทยาลัยอย่างเคร่งครัด} \vspace{3mm} \\*
&  \multicolumn{2}{p{0.75\textwidth}}{Students are required to conduct a thesis under supervision of advisors appointed by graduate college. Rules and regulations for under taking thesis set by students’ department and graduate college must be observed strictly.} \vspace{8mm} \\*
\end{longtable}



\newpage
\subsection{ชื่อ-สกุล ตำแหน่ง และคุณวุฒิของอาจารย์}
\subsubsection{อาจารย์ประจำหลักสูตร}

{\small
\begin{center}
\begin{longtable}{|p{0.05\textwidth}|p{0.32\textwidth}|p{0.35\textwidth}|
	C{0.08\textwidth}|
	C{0.08\textwidth}|}
	\hline
	
	\multirow{6}{0.05\textwidth}{\textbf{ลำดับ}} &
	\multirow{6}{0.32\textwidth}{ \textbf{ชื่อ-นามสกุล\newline ตำแหน่งทางวิชาการ\newline คุณวุฒิ-สาขาวิชา\newline ชื่อสถาบัน, ปี พ.ศ.ที่สำเร็จการศึกษา }} &
	\multirow{6}{0.35\textwidth}{\hfill\textbf{ผลงานทางวิชาการ}\hfill\,} &
	\multicolumn{2}{C{0.16\textwidth}|}{\textbf{ภาระการสอน ชม./สัปดาห์/ปีการศึกษา}} \\ \cline{4-5}	
& &	& \textbf{ปัจจุบัน} & \textbf{เมื่อเปิดหลักสูตรนี้แล้ว} \\
	\hline
\endhead	


%====================================================

1. &
นายพงศกร สุนทรายุทธ์ \newline 
รองศาสตราจารย์	\newline
ปร.ด.(คณิตศาสตร์ประยุกต์) \newline ม.เทคโนโลยีพระจอมเกล้าธนบุรี 2558 \newline
วท.ม.(คณิตศาสตร์ประยุกต์) \newline ม.เทคโนโลยีพระจอมเกล้าธนบุรี 2553  \newline
วท.บ.(คณิตศาสตร์) \newline ม.เทคโนโลยีพระจอมเกล้าธนบุรี 2551

&
\begin{enumerate}[series=tar]
	\item \underline{P. Sunthrayuth}, K. Kankam, R. Promkam and S. Srisawat, Three novel inertial subgradient extragradient methods for quasi-monotone variational inequalities in Banach spaces, Computational and Applied Mathematics (2024) 43:421 https://doi.org/10.1007/s40314-024-02929-7, (2024: Scopus Q1)
	\item \underline{P. Sunthrayuth}, K. Kankam, R. Promkam and S. Srisawat, Novel inertial methods for fixed point problems in reflexive Banach spaces with applications, Rendiconti del Circolo Matematico di Palermo Series 2, 10.1007/s12215-023-00976-3 (2023: Scopus Q1)
	\item R. Promkam, \underline{P. Sunthrayuth}, S. Kesornprom and E. Tanprayoon, New inertial self-adaptive algorithms for the split common null-point problem: application to data classifications, Journal of Inequalities and Applications, Article number: 136, 2023 (2022: Scopus Q1)
%	\item C.C. Okeke, A. Adamu, R. Promkam and \underline{P. Sunthrayuth}, Two-step inertial method for solving split common null point problem with multiple output sets in Hilbert spaces, AIMS Mathematics, 2023, Volume 8, Issue 9: 20201-20222 (2022: Scopus Q1)
\end{enumerate}
& 9 
& 12
\\ \hline
&&
\begin{enumerate}[resume*=tar]
	\item L.O. Jolaoso, \underline{P. Sunthrayuth}, P. Cholamjiak and Y.J. Cho, Inertial projection and contraction methods for solving variational inequalities with applications to image restoration problems, Carpathian Journal of Mathematics, Volume 39 (2023), No. 3, Pages 683 – 704 (2022: Scopus Q1)
	\item L.O. Jolaoso, N. Pholasa, \underline{P. Sunthrayuth} and P. Cholamjiak, Inertial-like Bregman projection method for solving systems of variational inequalities, Mathematical Methods in the Applied Sciences, 2023: 1–23, DOI: 10.1002/mma.9479 (2022: Scopus Q1)
	\item Z.B. Wang, \underline{P. Sunthrayuth}, A. Adamu and P. Cholamjiak, Modified accelerated Bregman projection methods for solving quasi-monotone variational inequalities, Optimization (2023), https://doi.org/10.1080/02331934.2023.2187663 (2022: Scopus Q1)
	\item B. Tan, \underline{P. Sunthrayuth}, P. Cholamjiak and Y.J. Cho, Modified inertial extragradient methods for finding minimum-norm solution of the variational inequality problem with applications to optimal control problem, International Journal of Computer Mathematics, 2023, VOL. 100, NO. 3, 525–545 (2022: Scopus Q2)
\end{enumerate} 
&  
&   \\ \hline
&& 
\begin{enumerate}[resume*=tar]
	\item M. Arfan, Maha M. A. Lashin, \underline{P. Sunthrayuth}, K. Shah, A. Ullah, K. Iskakova, M. R. Gorji and T. Abdeljawad, On nonlinear dynamics of COVID-19 disease model corresponding to nonsingular fractional order derivative, Medical and Biological Engineering and Computing, volume 60, pages 3169–3185 (2022) (2022: Scopus Q2)
	\item Y. Zhao, E.E. Elattar, M.A. Khan, Fatmawati, M. Asiri and \underline{P. Sunthrayuth}, The dynamics of the HIV/AIDS infection in the framework of piecewise fractional differential equation, Results in Physics, 40 (2022) 105842 (2022: Scopus Q1) 
	\item M.I. Asjad, \underline{P. Sunthrayuth}, M.D. Ikrama, T. Muhammad and A.S. Alshomrani, Analysis of non-singular fractional bioconvection and thermal memory with generalized Mittag-Leffler kernel, Chaos, Solitons and Fractals, 159 (2022) 112090 (2022: Scopus Q1)
	\item L.O. Jolaoso, \underline{P. Sunthrayuth}, P. Cholamjiak and Y.J. Cho, Analysis of two versions of relaxed inertial algorithms with Bregman divergences for solving variational inequalities, Computational and Applied Mathematics, (2022) 41:300 (2022: Scopus Q1) 
\end{enumerate} 
&  
&   \\ \hline 
&& 
\begin{enumerate}[resume*=tar]
	\item Y.M. Chu, M.F. Yassen, I. Ahmad, \underline{P. Sunthrayuth} and M.A. Khan, A FRACTIONAL SARS-COV-2 MODEL WITH ATANGANA–BALEANU DERIVATIVE: APPLICATION TO FOURTH WAVE, Fractals, Vol. 30, No. 08, 2240210 (2022) (2022: Scopus Q1)
	\item P. Jailokaa, S. Suantai and \underline{P. Sunthrayuth}, A Self-Adaptive Method for Split Common Null Point Problems and Fixed Point Problems for Multivalued Bregman Quasi-Nonexpansive Mappings in Banach Spaces, Filomat, 36:10 (2022), 3279–3300 (2022: Scopus Q2) 
	\item M. Huang, \underline{P. Sunthrayuth}, A.A. Pasha and M.A. Khan, Numerical solution of stochastic and fractional competition model in Caputo derivative using Newton method, AIMS Mathematics, 2022, Volume 7, Issue 5: 8933-8952 (2022: Scopus Q1) 
	\item J. Yang, P. Cholamjiak and \underline{P. Sunthrayuth}, Weak and strong convergence results for solving monotone variational inequalities in reflexive Banach spaces, Optimization (2022), https://doi.org/10.1080/02331934.2022.2069568 (2022: Scopus Q1)	
\end{enumerate} 
&  
&   \\ \hline
&&
\begin{enumerate}[resume*=tar]
	\item \underline{P. Sunthrayuth} and T.M. Tuyen, A Generalized Self-Adaptive Algorithm for the Split Feasibility Problem in Banach Spaces, Bulletin of the Iranian Mathematical Society, (2022) 48:1869–1893 (2022: Scopus Q3)
	\item T.M. Tuyen, \underline{P. Sunthrayuth}, and N.M. Trang, An inertial self-adaptive algorithm for the generalized split common null point problem in Hilbert spaces, Rendiconti del Circolo Matematico di Palermo Series 2 (2022) 71:537–557 (2022: Scopus Q1)
	\item \underline{P. Sunthrayuth}, L.O. Jolaoso and P. Cholamjiak, New Bregman projection methods for solving pseudo-monotone variational inequality problem, Journal of Applied Mathematics and Computing, volume 68, pages1565–1589 (2022) (2022: Scopus Q1) 
	\item M.I. Asjad, \underline{P. Sunthrayuth}, M.D. Ikrama, T. Muhammad and A.S. Alshomrani, Analysis of non-singular fractional bioconvection and thermal memory with generalized Mittag-Leffler kernel, Chaos, Solitons and Fractals, 159 (2022) 112090 (2022: Scopus Q1) 
\end{enumerate}
&  
&   \\ \hline
&&
\begin{enumerate}[resume*=tar]
	\item J. Yang, P. Cholamjiak and \underline{P. Sunthrayuth}, Modified Tseng’s splitting algorithms for the sum of two monotone operators in Banach spaces, AIMS Mathematics, 6(5): 4873–4900, (2021) (2022: Scopus Q1)
	\item T.M. Tuyen, R. Promkam and \underline{P. Sunthrayuth}, Strong convergence of a generalized forward–backward splitting method in reflexive Banach spaces, Optimization, Volume 71, 2022, 1483-1508 (2022: Scopus Q1) 
	\item P. Cholamjiak, N. Pholasa, S. Suantai and \underline{P. Sunthrayuth}, The generalized viscosity explicit rules for solving variational inclusion problems in Banach spaces, Optimization, Volume 70, 2021, 2607-2633 (2022: Scopus Q1)
	\item \underline{P. Sunthrayuth} and P. Cholamjiak, A modified extragradient method for variational inclusion and fixed point problems in Banach spaces, Applicable Analysis, Volume 100, 2021, 2049-2068 (2022: Scopus Q2) 
	\item P. Cholamjiak, S. Suantai and \underline{P. Sunthrayuth}, An explicit parallel algorithm for solving variational inclusion problem and fixed point problem in Banach spaces, Banach Journal of Mathematical Analysis, January 2020, Volume 14, Issue 1, 20–40 (2022: Scopus Q2) \newline
\end{enumerate} 
&  
& \\ \hline
&&
\begin{enumerate}[resume*=tar]
	\item P. Cholamjiak, S. Suantai and \underline{P. Sunthrayuth}, An iterative method with residual vectors for solving the fixed point and the split inclusion problems in Banach spaces, Computational and Applied Mathematics (2019) 38:12 (2022: Scopus Q1)
	\item S. Suantai, P. Cholamjiak and \underline{P. Sunthrayuth}, Iterative methods with perturbations for the sum of two accretive operators in q-uniformly smooth Banach spaces, RACSAM (2019) 113:203–223 (2022: Scopus Q1) 
\end{enumerate}
&  
& \\ \hline

2. &
นายวงศ์วิศรุต เขื่องสตุ่ง \newline 
รองศาสตราจารย์	\newline
ปร.ด.(คณิตศาสตร์ประยุกต์) \newline สถาบันเทคโนโลยีพระจอมเกล้าเจ้าคุณทหารลาดกระบัง 2559 \newline
วท.ม.(คณิตศาสตร์ประยุกต์) \newline สถาบันเทคโนโลยีพระจอมเกล้าเจ้าคุณทหารลาดกระบัง 2555  \newline
วท.บ.(คณิตศาสตร์) \newline สถาบันเทคโนโลยีพระจอมเกล้าเจ้าคุณทหารลาดกระบัง 2553

& 
\begin{enumerate}[series=note]
	\item \underline{W. Khuangsatung}, A.G. Gebrie, C. Suanooma,  2024. “Some New Results on Fixed Points for 𝜛-Distances in Complex-Valued Metric Spaces” Science and Technology Asia Vol.29 No.2 (April-June 2024) 
	\item A. Kheawborisut, \underline{W. Khuangsatung}, 2024. “A modified krasnoselskii-type subgradient extragradient algorithm with inertial effects for solving variational inequality problems and fixed point problem” Nonlinear Functional Analysis and Applications Vol. 29, No. 2(2024), pp. 393-418.
	\item \underline{W. Khuangsatung}, A. Singta, A., and A. Kangtunyakarn, 2024. “A regularization method for solving the G-variational inequality problem and fixed-point problems in Hilbert spaces endowed with graphs” Journal of Inequalities and Applications, 2024(15): 1-25. 
	\item P. Jailoka, C. Suanoom, \underline{W. Khuangsatung} and S. Suantai. Self-adaptive CQ-type algorithms for the split feasibility problem involving two bounded linear operators in Hilbert spaces. Carpathian Journal of Mathematics. 40(1): 77-98, (2024). 
\end{enumerate} 
& 9  
& 12\\ \hline
&&
\begin{enumerate}[resume*=note]
	\item \underline{W. Khuangsatung}, A. Kangtunyakarn. An intermixed method for solving the combination of mixed variational inequality problems and fixed-point problems. J Inequal Appl 2023, 1 (2023). https://doi.org/10.1186/s13660-022-02908-8.
	\item C. Suanoom, \underline{W. Khuangsatung}, T. Bantaojai. On an Open Problem in Complex Valued Rectangular b-Metric Spaces with an Application. Science  Technology Asia. 27(2), 78-83, (2022).  
	\item \underline{W. Khuangsatung}, A. Kangtunyakarn. Strong Convergence for the Modified Split Monotone Variational Inclusion and Fixed Point Problem. Thai Journal of Mathematics.  20(2),  889–904, (2022).
	\item \underline{W. Khuangsatung}, A. Kangtunyakarn. A Method for Solving the Variational Inequality Problem and Fixed Point Problems in Banach Spaces. Tamkang Journal of Mathematics. 53(1), 23-36,  (2022).
	\item P. Sukprasert, V. Yang, R. Khunprasert, \underline{W. Khuangsatung}, Convergence results for modified SP-iteration in uniformly convex metric spaces, Journal of Mathematics and Computer Science, 26(2), 162-171, (2021).
\end{enumerate}
&   
& \\ \hline
&&
\begin{enumerate}[resume*=note]
	\item C. Suanoom, \underline{W. Khuangsatung}. The Convergence Results for an AK-Generalized Nonexpansive Mapping in Hilbert Spaces. Thai Journal of Mathematics. 19(2), 623–634 (2021).
	\item \underline{W. Khuangsatung}, S. Suantai, A. Kangtunyakarn. The Modification of Generalized Mixed Equilibrium Problems for Convergence Theorem of Variational Inequalities Problems and Fixed Point Problems. Thai Journal of Mathematics. 19 (1), 271-296. (2021).
	\item \underline{W. Khuangsatung}, S. Suwannaut. Fixed Point Theorems for a Demicontractive Mapping and Equilibrium Problems in Hilbert Spaces. Communications in Mathematics and Applications 11 (2), 181-198. (2020)
	\item T. Bantaojai, C. Suanoom, \underline{W. Khuangsatung}. The Convergence Theorem for a Square  -Nonexpansive Mapping in a Hyperbolic Space. Thai Journal of Mathematics. 18(3), 1597–1609 (2020).
	\item \underline{W. Khuangsatung}, S. Chan-iam, P. Muangkarn, C. Suanoom, The Rectangular Quasi-Metric Space and Common Fixed Point Theorem for -Contraction and -Kannan Mappings. Thai Journal of Mathematics. 89-101 (2020).
\end{enumerate}
&  
& \\ \hline
&&
\begin{enumerate}[resume*=note]
	\item \underline{W. Khuangsatung}, A. Kangtunyakarn, The Method for Solving Fixed Point Problem of G-Nonexpansive Mapping in Hilbert Spaces Endowed with Graphs and Numerical Example. Indian J Pure Appl Math. 51, 155–170 (2020).
	\item \underline{W. Khuangsatung}, P. Jailoka, S. Suantai, An iterative method for solving proximal split feasibility problems and fixed point problems. Comp. Appl. Math. 38, 177 (2019). 
	\item C. Suanoom, K. Sriwichai, C. Klin-Eam, \underline{W. Khuangsatung}. The Finite Family L-Lipschitzian Suzuki-Generalized Nonexpansive Mappings. Communications in Mathematics and Applications. 10(1), 55–69. (2019).
	\item C. Suanoom, K. Sriwichai, C. Klin-Eam, \underline{W. Khuangsatung}. The Generalized -Nonexpansive Mappings and Related Convergence Theorems in Hyperbolic Spaces. Journal of Informatics and Mathematical Sciences. 11 (1), 1-17 (2019)
\end{enumerate}
&  
& \\ \hline
3. &
นายรัฐพรหม พรหมคำ \newline 
อาจารย์	\newline
Dr.rer.nat (Mathematik) \newline Universi\"{a}t W\"{u}rzburg 2562 \newline
วท.ม.(คณิตศาสตร์ประยุกต์) \newline ม.ธรรมศาสตร์ 2552  \newline
วท.บ.(คณิตศาสตร์) \newline ม.ธรรมศาสตร์ 2550

& 
\begin{enumerate}[series=dear]
	\item P. Sunthrayuth, K. Kankam,  \underline{R.Promkam} and S. Srisawat, (2023). Three novel inertial subgradient extragradient methods for quasi-monotone variational inequalities in Banach spaces, Computational and Applied Mathematics (2024) 43:421 https://doi.org/10.1007/s40314-024-02929-7, (2024: Scopus Q1)
	\item P. Sunthrayuth, K. Kankam, \underline{R. Promkam} and S. Srisawat, (2023). Novel inertial methods for fixed point problems in reflexive Banach spaces with applications, Rendiconti del Circolo Matematico di Palermo Series 2, 10.1007/s12215-023-00976-3 (2023: Scopus Q1)
	\item \underline{R. Promkam}, P. Sunthrayuth, S. Kesornprom and E. Tanprayoon, (2023). New inertial self-adaptive algorithms for the split common null-point problem: application to data classifications, Journal of Inequalities and Applications, Article number: 136, 2023 (2022: Scopus Q1)
	\item C.C. Okeke, A. Adamu, \underline{R. Promkam} and P. Sunthrayuth, Two-step inertial method for solving split common null point problem with multiple output sets in Hilbert spaces, AIMS Mathematics, 2023, Volume 8, Issue 9: 20201-20222 (2022: Scopus Q1)
\end{enumerate}
& 9 
& 12 \\ \hline
&&
\begin{enumerate}[resume*=dear]
	\item T.M. Tuyen, \underline{R. Promkam} and P. Sunthrayuth, Strong convergence of a generalized forward–backward splitting method in reflexive Banach spaces, Optimization, Volume 71, 2022, 1483-1508 (2022: Scopus Q1)
	\item Y. Tang, \underline{R. Promkam}, P. Cholamjiak, and P. Sunthrayuth, “Convergence Results of Iterative Algorithms for the Sum of Two Monotone Operators in Reflexive Banach Spaces,” Appl Math, Sep. 2021, doi: 10.21136/AM.2021.0108-20.
\end{enumerate}
&  
& \\ \hline
\end{longtable}
\end{center}

\subsubsection{อาจารย์ผู้สอน}

{
\begin{center}
\begin{longtable}{|p{0.05\textwidth}|p{0.32\textwidth}|p{0.35\textwidth}|
	C{0.08\textwidth}|
	C{0.08\textwidth}|}
	\hline
	
	\multirow{6}{0.05\textwidth}{\textbf{ลำดับ}} &
	\multirow{6}{0.32\textwidth}{ \textbf{ชื่อ-นามสกุล\newline ตำแหน่งทางวิชาการ\newline คุณวุฒิ-สาขาวิชา\newline ชื่อสถาบัน, ปี พ.ศ.ที่สำเร็จการศึกษา }} &
	\multirow{6}{0.35\textwidth}{\hfill\textbf{ผลงานทางวิชาการ}\hfill\,} &
	\multicolumn{2}{C{0.16\textwidth}|}{\textbf{ภาระการสอน ชม./สัปดาห์/ปีการศึกษา}} \\ \cline{4-5}	
& &	& \textbf{ปัจจุบัน} & \textbf{เมื่อเปิดหลักสูตรนี้แล้ว} \\
	\hline
\endhead	
 

%====================================================

1. &
นางสาววรรณา ศรีปราชญ์ \newline 
ผู้ช่วยศาสตราจารย์ (คณิตศาสตร์)	\newline
ปร.ด.(คณิตศาสตร์) \newline มหาวิทยาลัยนเรศวร 2558 \newline
วท.ม.(คณิตศาสตร์) \newline  มหาวิทยาลัยนเรศวร 2554 \newline
คบ.(คณิตศาสตร์) \newline สถาบันราชภัฏพระนครศรีอยุธยา 2541
& 
\begin{enumerate}[series=na]
	\item S. Srisawat, \underline{W. Sriprad}. (2024)	Some identities of (s, t)-Pell and (s, t)-Pell-Lucas polynomials by matrix methods International Journal of Mathematics and Computer Science, 19(2024), no. 4, 1183–1188 (Scopus Q2)
\end{enumerate} 
& 9 
& 12 \\ \hline
2. &
นางสาวกมลรัตน์ สมบุตร \newline 
ผู้ช่วยศาสตราจารย์ (คณิตศาสตร์)	\newline
ปร.ด.(คณิตศาสตร์) \newline มหาวิทยาลัยนเรศวร 2557 \newline
คบ.(คณิตศาสตร์) \newline มหาวิทยาลัยราชภัฏอุตรดิตถ์ 2550
& 
\begin{enumerate}[series=jee]
	\item \underline{K. Sombut}, K. Khammahawong, P. Borisut, N. Makate. (2024)	Existence and uniqueness of solutions of a coupled system of $\psi$-Hilfer fractional differential equations under uncoupled non-local multi point conditions involving fixed point theorems Journal of Nonlinear Functional Analysis 2024 (2024) 21. (Scopus Q1)
	\item K. Amnuaykarn, P. Kumam,  \underline{K. Sombut}, J. Nantadilok. (2024) Best proximity points of generalized α-ψ-Geraghty proximal contractions in generalized metric spaces Fixed Point Theory, Volume 25, No. 1, 2024, 15-30, February 1st, 2024. (Scopus Q2)
	\item \underline{K. Sombut}, T. Seangwattana, K. Sitthithakerngkiet, A. Arunchai. (2023)	An Inertial Forward-Backward Splitting Method for Solving Modified Variational Inclusion Problems and Its Application Mathematics 2023, 11, 2107 (Scopus Q1)
	\item A. Arunchai, \underline{K. Sombut}, T. Seangwattana, K. Sitthithakerngkiet. (2023)	Image restoration by using a modified proximal point algorithm AIMS Mathematics 2023, Volume 8, Issue 4: 9557-9575. doi: 10.3934/math.2023482 (Scopus Q1)
\end{enumerate} 
& 9 
& 12 \\ \hline
&&
\begin{enumerate}[resume*=jee]
	\item K. Khammahawong, \underline{K. Sombut}, P. Chaipunya. (2022)	Approximating Common Fixed Points of Nonexpansive Mappings on Hadamard Manifolds with Applications Mathematics Volume 10 Issue 21 10.3390/math10214080 (ISI Q1)
	\item H. ur Rehman, W. Kumam, \underline{K. Sombut}. (2022)	Inertial Modification Using Self-Adaptive Subgradient Extragradient Techniques for Equilibrium Programming Applied to Variational Inequalities and Fixed-Point Problems  Mathematics, 10, 1751, pp.1-29, 20 May 2022. (Scopus Q1)
	\item T. Seangwattana, K. Sitthithakerngkiet, \underline{K. Sombut}, A. Arunchai.(2022)	On Strengthened Extragradient Methods Non-Convex Combination with Adaptive Step Size Rule for Equilibrium Problems Symmetry, 2022, 14, 1045. (Scopus Q1)
\end{enumerate}
&  
&  \\ \hline
3. &
นางสาวนนธิยา มากะเต \newline 
อาจารย์(คณิตศาสตร์ประยุกต์)	\newline
วท.ด.(คณิตศาสตร์ประยุกต์) \newline มหาวิทยาลัยเทคโนโลยีสุรนารี 2556 \newline
วท.ม.(คณิตศาสตร์) \newline  มหาวิทยาลัยเชียงใหม่ 2545  \newline
วท.บ.(คณิตศาสตร์) \newline มหาวิทยาลัยนเรศวร 2543
& 
\begin{enumerate}[series=non]
	\item \underline{Makate, N.}, Rattanajak, P. and Mongkhol, B. (2024) Bi-Periodic k-Pell Sequence, International Journal of Mathematics and Computer Science, Vol 19, No. 1, pp. 103-109. future-in-tech.net/Volume19.1.htm (Scopus Q2) 
	\item K. Sombut, K. Khammahawong, P. Borisut, \underline{N. Makate}. (2024)	Existence and uniqueness of solutions of a coupled system of $\psi$-Hilfer fractional differential equations under uncoupled non-local multi point conditions involving fixed point theorems Journal of Nonlinear Functional Analysis 2024 (2024) 21. (Scopus Q1)
	\item \underline{N. Makate}, P. Rattanajak, B. Mongkhol. (2023)	Bi-Periodic k-Pell Sequence. International Journal of Mathematics and Computer Science 19(2024), no. 1 (Scopus Q2)
	\item Srimud, K., \underline{Makate, N.}, Ampawa, T. and Jantree, T., (2022) On the Diophantine Equation 2/x+3/y+4/z=1/2, Progress in Applied Science and Technology, Vol 12, No. 1, pp. 11-16. https://ph02.tci-thaijo.org/index.php/past/article/view/246172 (TCI 1)
\end{enumerate} 
& 9 
& 12 \\ \hline
4. &
นางสาวภคีตา สุขประเสริฐ \newline 
ผู้ช่วยศาสตราจารย์ (คณิตศาสตร์ประยุกต์)	\newline
ปร.ด.(คณิตศาสตร์ประยุกต์) \newline มหาวิทยาลัยเทคโนโลยีพระจอมเกล้าธนบุรี 2560 \newline
วท.ม.(คณิตศาสตร์) \newline  มหาวิทยาลัยธรรมศาสตร์ 2552 \newline
วท.บ.(คณิตศาสตร์) \newline มหาวิทยาลัยธรรมศาสตร์ 2550
& 
\begin{enumerate}[series=tik]
	\item A. Padcharoen, \underline{P. Sukprasert}. (2024) Ciric-contraction type via wt-distance Advances in Fixed Point Theory Vol 14 (2024)(Scopus Q4)
	\item A. Padcharoen, \underline{P. Sukprasert}. (2022)	Convergence of Iterative Scheme for Asymptotically Nonexpansive Mapping in Hadamard Spaces WSEAS Transactions on Mathematics Vol.22, 2023 (Scopus Q3)
	\item C. Mungkala, \underline{P. Sukprasert}, A. Padcharoen.(2022) Coincidence Point Results in Hausdorff Rectangular Metric Spaces with an Application to Lebesgue Integral Function WSEAS Transactions on Mathematics, ISSN / E-ISSN: 1109-2769 / 2224-2880, Volume 21, 2022, Art. (Scopus Q3)
\end{enumerate} 
& 9 
& 12 \\ \hline
5. &
นายอลงกต สุวรรณมณี \newline 
อาจารย์(คณิตศาสตร์ประยุกต์)	\newline
วท.ม.(คณิตศาสตร์ประยุกต์ (นานาชาติ)) \newline  มหาวิทยาลัยมหิดล 2549  \newline
วท.บ.(คณิตศาสตร์) \newline มหาวิทยาลัยมหิดล 2546

& 
\begin{enumerate}[series=kot]
	\item Ampawa , T. and Suvarnamani, A. (2023). Sustainable Tourism Route Management in Group of Pathum Thani, Nakhon Nayok, Prachinburi, Chachoengsao and Sa Kaeo Province. Journal of Thai Hospitality and Tourism, 18(1), 49–60. https://so04.tci-thaijo.org/index.php/tourismtaat/article/view/253390.  (TCI 1)
	\item Ampawa , T. and Suvarnamani, A. (2021). Some properties of new multiplicative pulsating 3-Fibonacci sequence. วารสารวิทยาศาสตร์และเทคโนโลยี      หัวเฉียวเฉลิมพระเกียรติ, 7(2), 8–15. https://ph02.tci-thaijo.org/index.php/scihcu/article/view/244603.(TCI 1) 
\end{enumerate} 
& 9 
& 12 \\ \hline
6. &
นายปริญญวัฒน์  ชูสุวรรณ \newline 
ผู้ช่วยศาสตรจารย์ (คณิตศาสตร์)	\newline
วท.ด.(คณิตศาสตร์) \newline จุฬาลงกรณ์มหาวิทยาลัย 2561 \newline
วท.ม.(คณิตศาสตร์) \newline  จุฬาลงกรณ์มหาวิทยาลัย 2557  \newline
วท.บ.(คณิตศาสตร์) \newline มหาวิทยาลัยสงขลานครินทร์ 2555
& 
\begin{enumerate}[series=new]
	\item Choosuwan P., Sangsawang P., Matwangsang C., Thongsupol S. and Sirisuk S. (2024) Generalized Order Divisor of Finite Groups. International Journal of Group Theory, 13(1), 31-45. 
\end{enumerate} 
& 9 
& 12 \\ \hline
7. &
นางสาวปฤณท์ธพร สงวนสุทธิกุล \newline 
อาจารย์ (คณิตศาสตร์ประยุกต์)	\newline
ปร.ด.(คณิตศาสตร์ประยุกต์) \newline มหาวิทยาลัยเทคโนโลยีพระจอมเกล้าธนบุรี 2563 \newline
วท.ม.(คณิตศาสตร์ประยุกต์) \newline  มหาวิทยาลัยเทคโนโลยีพระจอมเกล้าธนบุรี 2560 \newline
วท.บ.(คณิตศาสตร์) \newline มหาวิทยาลัยศรีนครินทรวิโรฒ 2558
& 
\begin{enumerate}[series=phing]
	\item \underline{Sanguansuttigul, P.}, Chayawatto, N. and Chaipunya, P. (2024) A Bilevel QP-PLP Approach to Demand Response Modulation between Consumers and a Single Electricity Seller, Science and Technology Asia, Vol 29, No. 2,  pp. 32-44. (WoS ESCI, SJR Q4, Scopus Q3)
	\item \underline{Sanguansuttigul, P.} (2023) An Optimal Control Technique for Epidemiological Model with Limited Vaccination Supply, Thai Journal of Mathematics, Vol 21, No. 3, pp. 657-669. (WoS ESCI, SJR Q4, Scopus Q4)
	\item Chaipunya, P., Chuensupantharat, N. and \underline{Sanguansuttigul, P.} (2023) Graphical Ekeland’s variational principle with a generalized w-distance and a new approach to quasi- equilibrium problems, Carpathian Journal of Mathematics, Vol 39, No. 1, pp. 95-107. (WoS SCIE, SJR Q2, Scopus Q1)
	
\end{enumerate} 
& 9 
& 12 \\ \hline
8. &
นายโอม สถิตยนาค \newline 
อาจารย์ (คณิตศาสตร์)	\newline
วท.ม.(คณิตศาสตร์) \newline  จุฬาลงกรณ์มหาวิทยาลัย 2551  \newline
วท.บ.(คณิตศาสตร์) \newline มหาวิทยาลัยธรรมศาสตร์, 2547
& 
\begin{enumerate}[series=ohm]
	\item \underline{W. Thongkamhaeng}, S. Wongwai. (2023) Pseudo NQ-principally Projective Modules International Journal of Mathematics and Computer Science 19(2024), no. 1, 49–56. (Scopus Q2)
\end{enumerate} 
& 9 
& 12 \\ \hline
9. &
นางสาววาสนา ทองกำแหง \newline 
อาจารย์ (คณิตศาสตร์)	\newline
วท.ม.(คณิตศาสตร์) \newline  มหาวิทยาลัยรามคำแหง 2551  \newline
วท.บ.(คณิตศาสตร์) \newline มหาวิทยาลัยศรีนครินทรวิโรฒประสานมิตร, 2543
& 
\begin{enumerate}[series=pui]
	\item A. Bumpendee, \underline{W. Thongkamhaeng}, S. Wongwai. (2023) Pseudo NQ-principally Projective Modules International Journal of Mathematics and Computer Science 19(2024), no. 1, 49–56. (Scopus Q2)
\end{enumerate} 
& 9 
& 12 \\ \hline
10. &
นางสาวอมราภรณ์ บำเพ็ญดี \newline 
อาจารย์ (คณิตศาสตร์)	\newline
วท.ม.(คณิตศาสตร์) \newline  มหาวิทยาลัยรามคำแหง 2550  \newline
วท.บ.(คณิตศาสตร์) \newline มหาวิทยาลัยศรีนครินทรวิโรฒประสานมิตร, 2543
& 
\begin{enumerate}[series=nok]
	\item \underline{A. Bumpendee}, W. Thongkamhaeng, S. Wongwai. (2023) Pseudo NQ-principally Projective Modules International Journal of Mathematics and Computer Science 19(2024), no. 1, 49–56. (Scopus Q2)
\end{enumerate} 
& 9 
& 12 \\ \hline
12. &
นายอัคเรศ สิงห์ทา \newline 
อาจารย์ (คณิตศาสตร์)	\newline
วท.ม.(คณิตศาสตร์) \newline  มหาวิทยาลัยรามคำแหง 2551  \newline
วท.บ.(คณิตศาสตร์) \newline มหาวิทยาลัยศรีนครินทรวิโรฒประสานมิตร, 2543
& 
\begin{enumerate}[series=pui]
	\item W. Khuangsatung, \underline{A. Singta}, A., and A. Kangtunyakarn, 2024. “A regularization method for solving the G-variational inequality problem and fixed-point problems in Hilbert spaces endowed with graphs” Journal of Inequalities and Applications, 2024(15): 1-25.
\end{enumerate} 
& 9 
& 12 \\ \hline
13. &
นายนิพัทธ์ จงสวัสดิ์ \newline 
ผู้ช่วยศาสตราจารย์(เทคโนโลยีสารสนเทศ)	\newline
ปร.ด.(เทคโนโลยีสารสนเทศเชิงธุรกิจ) \newline มหาวิทยาลัยสยาม 2554 \newline
วท.ม.(ระบบสารสนเทศคอมพิวเตอร์) \newline  มหาวิทยาลัยอัสสัมชัญ 2545  \newline
วท.บ.(วิศวกรรมไฟฟ้า) \newline มหาวิทยาลัยอัสสัมชัญ 2542
& 
\begin{enumerate}[series=nipat]
	\item Y. Thwe, A. Tungkasthan, \underline{N. Jongsawat}, Accurate fashion and accessories detection for mobile application based on deep learningInternational Journal of Electrical and Computer Engineering (IJECE) Vol. 13, No. 4, August2023, pp. 4347~4356 DOI: 10.11591/ijece.v13i4.pp4347-4356(2023: Scopus Q2) 
	\item A. Tungkasthan, \underline{N. Jongsawat}, Development of Portable Air Quality Monitor Devices and Real Time Monitoring Cloud-Based System, Sripatum Review of Science and Technology Vol.14 January-December 2022, pp 95-110(2022: TCI1)
	\item Y. Thwe, \underline{N. Jongsawat}, A. Tungkasthan, A Semi-Supervised Learning Approach for Automatic Detection and Fashion Product Category Prediction with Small Training Dataset Using FC-YOLOv4 Applied Sciences Volume 12 Issue 16(2022: Scopus Q2) 
\end{enumerate} 
& 9 
& 12 \\ \hline


\end{longtable}
\end{center}


\newpage
\subsubsection{อาจารย์พิเศษ}
{
\begin{center}
\renewcommand{\arraystretch}{1.2}
\begin{longtable}{|C{0.05\textwidth}|p{0.3\textwidth}|p{0.2\textwidth}|
	p{0.35\textwidth}|}
	\hline
	\multicolumn{1}{|c|}{\textbf{ลำดับ}} &
	\multicolumn{1}{c|}{\textbf{ชื่อ-นามสกุล}} &
	\multicolumn{1}{c|}{\textbf{ตำแหน่ง}} &
	\multicolumn{1}{c|}{\textbf{สถานที่ทำงาน}}  \\
	\hline
\endhead	
%====================================================
1. 
& ดร.ธนพงศ์ อินทระ
& ผู้ช่วยศาสตราจารย์
& มหาวิทยาลัยขอนแก่น
\\ \hline
2. 
& ดร.วีรวัฒน์ สุทธ์สุทัศน์
& ผู้ช่วยศาสตราจารย์
& มหาวิทยาลัยรามคำแหง  
 \\ \hline
3. 
& นายเอกพงษ์ บุญเซ็น 
& กรรมการผู้จัดการ
& บริษัท เอมเมอรอล เรียลเอสเตท จำกัด 
 \\ \hline
4. 
& นายสืบพงษ์ สิทธิมาลัยรัตน์
& กรรมการผู้จัดการ
& บริษัท ซิมพลิโค จำกัด 
 \\ \hline
\end{longtable}
\end{center}

\section{องค์ประกอบเกี่ยวกับประสบการณ์ภาคสนาม}

\subsection{มาตรฐานผลการเรียนรู้ของประสบการณ์ภาคสนาม}
ไม่ม่
\subsection{ช่วงเวลา}
ไม่มี
\subsection{การจัดเวลาและตารางสอน}
ไม่มี

\section{ข้อกำหนดเกี่ยวกับการทำโครงงานหรืองานวิจัย}
ข้อกำหนดในการทำงานวิจัยของนักศึกษาต้องเป็นหัวข้อที่เกี่ยวข้องการพัฒนาแบบจำลอง อัลกอริทึม และตัวปรับแต่ง (Optimizers)  ทางด้านการเรียนรู้ของเครื่อง (Machine Learning) และ การเรียนรู้เชิงลึก (Deep Learning) ภายใต้การดูแลและการให้คำปรึกษาจากอาจารย์ผู้ควบคุม มีขอบเขตการทำงานที่ชัดเจน มีการรายงานความก้าวหน้าทุกภาคการศึกษา การเขียนวิทยานิพนธ์ตามรูปแบบที่กำหนดและทดสอบความรู้ต่อคณะกรรมการสอน
\subsection{คำอธิบายโดยย่อ}
นักศึกษาต้องสามารถวิเคราะห์ปัญหา เพื่อกำหนดหัวข้อวิจัย ทำการศึกษา สืบค้นและรวบรวมข้อมูล วางแผนการวิจัย วิเคราะห์และอภิปรายผลการวิจัย เสนอผลงาน เขียนรายงานผลการวิจัยในเนื้อหาที่เกี่ยวข้อง
\subsection{มาตรฐานผลการเรียนรู้}
\begin{enumerate}
	\item นักศึกษาผ่านการสอบเค้าโครงวิทยานิพนธ์
	\item นักศึกษาสามารถดำเนินการวิจัยภายใต้การดูแลของอาจารย์ที่ปรึกษา
	\item นักศึกษาจัดทำเล่มวิทยานิพนธ์และสอบป้องกันวิทยานิพนธ์ตามที่มหาวิทยาลัยกำหนด
	\item ผลงานวิทยานิพนธ์จะต้องได้รับการตีพิมพ์หรืออย่างน้อยต้องดำเนินการให้ผลงานหรือส่วนหนึ่งของผลงานได้รับการยอมรับให้ตีพิมพ์ตามประกาศมหาวิทยาลัยเทคโนโลยีราชมงคลธัญบุรี เรื่อง การตีพิมพ์บทความวิจัยเพื่อการสำเร็จการศึกษาระดับบัณฑิตศึกษา พ.ศ.2565
\end{enumerate}

\subsection{ช่วงเวลา}
ภาคการศึกษาที่ 1-2 ของปีการศึกษาที่ 
2

\subsection{จำนวนหน่วยกิต}
12 หน่วยกิต


\subsection{การเตรียมการ}

มีการจัดการเรียนการสอนรายวิชาในหมวดวิชาบังคับ และหมวดวิชาเลือกในกลุ่มวิชาการวิเคราะห์เชิงคณิตศาสตร์ กลุ่มวิชาคณิตศาสตร์เชิงคำนวณและการประยุกต์ และกลุ่มวิชาวิทยาการข้อมูลและการเรียนรู้ของเครื่อง เพื่อเสนอหัวข้อในรูปแบบที่นักศึกษาสนใจ มีการแต่งตั้งอาจารย์ที่ปรึกษาวิทยานิพนธ์ กำหนดชั่วโมงในการให้คำปรึกษา อาจารย์ที่ปรึกษาให้คำปรึกษาในการเลือกหัวข้อและกระบวนการวิจัย มีตัวอย่างงานวิจัยเพื่อศึกษาค้นคว้า

\subsection{กระบวนการประเมินผล}
ประเมินผลจากการเสนอหัวข้อและโครงร่างวิทยานิพนธ์ โดยนักศึกษานำเสนอผลการศึกษาต่อคณะกรรมการสอบตามรูปแบบและระยะเวลาตามที่ได้กำหนด ติดตามความก้าวหน้าในการทำวิทยานิพนธ์ 
การตีพิมพ์ผลงานวิจัยที่เป็นส่วนหนึ่งของวิทยานิพนธ์ในวารสารหรือสิ่งพิมพ์วิชาการ และการสอบวิทยานิพนธ์ให้เป็นไปตามแผนการศึกษา และตามที่มหาวิทยาลัยกำหนด





















