\chapter{ปรัชญา วัตถุประสงค์ และผลการเรียนรู้ของหลักสูตร}

\section{ปรัชญา ความสำคัญ และวัตถุประสงค์ของหลักสูตร}

\subsection{ปรัชญาการศึกษาของมหาวิทยาลัยฯ}

นวัตกรรมสร้างชาติ ราชมงคลธัญบุรีสร้างนวัตกรรม

\subsection{ปรัชญาหลักสูตร}

ผสานความรู้ด้านคณิตศาสตร์ สถิติ และคอมพิวเตอร์ เพื่อสร้างนวัตกรรมเชิงคำนวณที่ตอบโจทย์เศรษฐกิจและสังคมดิจิทัล และผลิตบัณฑิตที่มีศักยภาพในการคิด วิเคราะห์ และพัฒนานวัตกรรมอย่างยั่งยืน

\subsection{ความสำคัญ} 

หลักสูตรวิทยาศาสตรมหาบัณฑิต สาขาวิทยาการเชิงคำนวณถูกออกแบบมาให้สอดคล้องกับเป้าหมายการพัฒนาที่ยั่งยืนของสหประชาชาติ (SDGs) หลายข้อ โดยเฉพาะอย่างยิ่ง เป้าหมายที่ 9 อุตสาหกรรม นวัตกรรม และโครงสร้างพื้นฐาน, เป้าหมายที่ 8 การจ้างงานที่มีคุณค่าและการเติบโตทางเศรษฐกิจ, เป้าหมายที่ 11 เมืองและชุมชนที่ยั่งยืน, และเป้าหมายที่ 17 ความร่วมมือเพื่อการพัฒนาที่ยั่งยืน หลักสูตรมุ่งเน้นในการเสริมสร้างทักษะการคำนวณขั้นสูงให้กับนักศึกษา เพื่อส่งเสริมการนวัตกรรมและความก้าวหน้าทางเทคโนโลยี ซึ่งเป็นสิ่งจำเป็นสำหรับการสร้างโครงสร้างพื้นฐานที่ยืดหยุ่นและส่งเสริมการอุตสาหกรรมที่ครอบคลุมและยั่งยืน บัณฑิตที่สำเร็จการศึกษาจะพร้อมที่จะขับเคลื่อนการเติบโตทางเศรษฐกิจโดยการพัฒนากระบวนการแก้ปัญหาที่มีประสิทธิภาพต่อปัญหาที่ซับซ้อนในอุตสาหกรรมต่าง ๆ ซึ่งนำไปสู่การสร้างงานที่มีคุณค่าและการเพิ่มผลิตภาพ

ในบริบทของเมืองและชุมชนที่ยั่งยืน (เป้าหมายที่ 11) หลักสูตรเสริมสร้างความสามารถให้นักศึกษาสามารถประยุกต์ใช้วิธีการคำนวณและการคิดวิพากษ์ผ่านการสร้างแบบจำลองทางคณิตศาสตร์เพื่อนำไปประยุกต์ใช้การแก้ปัญหาสิ่งแวดล้อม การวางผัง ตลอดจนการจัดการทรัพยากร ซึ่งช่วยให้เกิดการพัฒนาเมืองอัจฉริยะที่ใช้ทรัพยากรอย่างมีประสิทธิภาพ ลดผลกระทบต่อสิ่งแวดล้อม และปรับปรุงคุณภาพชีวิตของผู้อยู่อาศัย โดยการผสานเทคโนโลยีการคำนวณเข้ากับการลงมือจริงจริงในห้องปฏิบัติการ บัณฑิตจะมีส่วนร่วมในการสร้างสภาพแวดล้อมเมืองที่น่าอยู่

นอกจากนี้ หลักสูตรยังเน้นความสำคัญของความร่วมมือระดับโลก (เป้าหมายที่ 17) โดยส่งเสริมการทำงานร่วมกันระหว่างสถาบันการศึกษา อุตสาหกรรม และหน่วยงานรัฐบาล ผ่านโครงการสหวิทยาการและความคิดริเริ่มด้านการวิจัย นักศึกษาจะได้มีส่วนร่วมในการแลกเปลี่ยนความรู้และการแก้ปัญหาแบบร่วมมือกันในระดับโลก ซึ่งไม่เพียงแต่เพิ่มพูนประสบการณ์การศึกษา แต่ยังมีส่วนในการสร้างความร่วมมือที่แข็งแกร่งที่จำเป็นสำหรับการบรรลุ SDGs ซึ่งขยายผลกระทบของหลักสูตรต่อการพัฒนาที่ยั่งยืนทั่วโลก

หลักสูตรวิทยาศาสตรมหาบัณฑิต สาขาวิทยาการเชิงคำนวณ มีความสำคัญอย่างยิ่งในการแก้ไขและลดช่องว่างในสถานการณ์ปัจจุบันของโลก โดยเฉพาะอย่างยิ่งในประเทศไทย ในยุคที่ปัญญาประดิษฐ์ (AI) และการเรียนรู้ของเครื่อง (Machine Learning) มีบทบาทสำคัญในการขับเคลื่อนนวัตกรรมและเศรษฐกิจ การค้นหาอัลกอริทึมและตัวปรับแต่งใหม่ ๆ โดยเฉพาะในแนวทางการเรียนรู้เชิงลึก (Deep Learning) เป็นสิ่งจำเป็นในการพัฒนาระบบที่มีประสิทธิภาพและชาญฉลาด

หลักสูตรนี้มุ่งเน้นการผลิตบัณฑิตที่มีความเชี่ยวชาญในการพัฒนาแบบจำลอง อัลกอริทึม และตัวปรับแต่ง (Optimizers) ในสาขาการเรียนรู้ของเครื่องซึ่งจะช่วยในการแก้ไขปัญหาที่ซับซ้อนและท้าทายที่ประเทศไทยและโลกกำลังเผชิญ เช่น การวิเคราะห์ข้อมูลขนาดใหญ่ การประมวลผลภาษาธรรมชาติ และการประยุกต์ใช้ปัญญาประดิษฐ์ในภาคอุตสาหกรรมต่าง ๆ ด้วยการเสริมสร้างความรู้และทักษะในด้านนี้ บัณฑิตจะสามารถสร้างสรรค์นวัตกรรมที่มีผลกระทบสูง และสนับสนุนการตัดสินใจที่มีข้อมูลเป็นฐาน

นอกจากนี้ หลักสูตรยังช่วยเสริมสร้างศักยภาพของประเทศไทยในการเป็นผู้พัฒนานวัตกรรมด้านปัญญาประดิษฐ์ โดยการส่งเสริมการวิจัยและพัฒนาในสาขาที่กำลังเติบโตนี้ ด้วยการสนับสนุนให้นักศึกษามีความคิดสร้างสรรค์และความสามารถในการพัฒนาเทคโนโลยีใหม่ ๆ ประเทศไทยจะสามารถเพิ่มขีดความสามารถในการแข่งขันบนเวทีโลก และตอบสนองต่อความต้องการของตลาดแรงงานที่ต้องการบุคลากรที่มีทักษะสูงในด้านนี้

เพื่อเสริมสร้างความสำคัญของหลักสูตรนี้ต่อแนวโน้มการเลือกศึกษาของนักศึกษา หลักสูตรวิทยาศาสตรมหาบัณฑิต สาขาวิทยาการเชิงคำนวณ ตอบสนองต่อความสนใจที่เพิ่มขึ้นของนักศึกษาในด้านเทคโนโลยีขั้นสูงในด้านปัญญาประดิษฐ์ (AI) และการเรียนรู้ของเครื่อง (Machine Learning)

แนวโน้มการเลือกศึกษาของนักศึกษาแสดงถึงความสนใจในสาขาเทคโนโลยีและการคำนวณที่เพิ่มขึ้น รายงานจากหลายแหล่งระบุว่ามีนักศึกษาสมัครเข้าเรียนในสาขาวิทยาศาสตร์คอมพิวเตอร์ วิศวกรรมคอมพิวเตอร์ และสาขาที่เกี่ยวข้องกับ AI และการเรียนรู้ของเครื่องมากขึ้น เนื่องจากเห็นถึงโอกาสในการทำงานที่กว้างขวางและความต้องการบุคลากรในตลาดแรงงานที่เพิ่มขึ้น

นักศึกษามองหาหลักสูตรที่มีความเกี่ยวข้องกับตลาดงานและมีโอกาสการทำงานสูง สาขาวิชาที่เกี่ยวข้องกับ AI การเรียนรู้ของเครื่อง และวิทยาการข้อมูล (Data Science) ถูกจัดอันดับให้เป็นสาขาที่มีศักยภาพสูง ทั้งในด้านเงินเดือนและการเติบโตในสายอาชีพ หลักสูตรที่เน้นการพัฒนาอัลกอริทึมและตัวปรับแต่งใหม่ ๆ ในการเรียนรู้ของเครื่องตรงกับความต้องการของนักศึกษาที่ต้องการความท้าทายและการสร้างสรรค์นวัตกรรม

หลักสูตรนี้ยังสอดคล้องกับแนวโน้มการศึกษาระดับสูงที่มุ่งเน้นการวิจัยและพัฒนาเทคโนโลยีใหม่ นักศึกษาที่สนใจในการแก้ไขปัญหาที่ซับซ้อนและการมีส่วนร่วมในการพัฒนาเทคโนโลยีที่มีผลกระทบสูง จะถูกดึงดูดโดยหลักสูตรที่ให้โอกาสในการทำวิจัยและพัฒนาอัลกอริทึมใหม่ ๆ


\clearpage
\subsection{ความต้องการและความคาดหวังของผู้มีส่วนได้ส่วนเสีย}

\begin{longtable}{|p{0.2\textwidth} | p{0.75\textwidth}|}
\hline
\textbf{ผู้มีส่วนได้ส่วนเสีย} & \textbf{ความต้องการ/ความคาดหวังของผู้มีส่วนได้ส่วนเสีย} \\ \hline
\endhead

\multicolumn{2}{|l|}{\textbf{1. ผู้มีส่วนได้เสียภายนอกหน่วยงาน}} \\ \hline

ผู้ใช้บัณฑิต & 
\begin{enumerate}
  \item \textbf{Western Digital Storage Technologies (Thailand) Ltd.:} 
    \begin{enumerate}
      \item ความรู้และทักษะด้าน Machine Learning, Deep Learning
      \item การเขียนโปรแกรมภาษา Python
      \item ความรู้ภาษา SQL
      \item การสื่อสารภาษาอังกฤษ
      \item ทักษะการนำเสนอ (Presentation skill)
    \end{enumerate}
  \item \textbf{ธนาคารทหารไทยธนชาต จำกัด (มหาชน):}
    \begin{enumerate}
      \item ทักษะ Machine Learning
      \item ความรู้ด้านสถิติและการเขียนโปรแกรมวิเคราะห์ข้อมูล (Stat Programming)
      \item ความรู้เชิงธุรกิจ (Business)
      \item ทักษะโปรแกรม R
      \item โปรแกรม Statistical Analysis System (SAS)
      \item โปรแกรมภาษา Python
      \item ความเข้าใจโครงสร้างข้อมูล (Data Structure)
    \end{enumerate}
  \item \textbf{ธนาคารกรุงศรีอยุธยา จำกัด (มหาชน):}
    \begin{enumerate}
      \item เขียนโปรแกรมภาษา Python
      \item Query ข้อมูลด้วย SQL, Oracle
      \item ใช้เครื่องมือ Power BI, Power Query in Excel, Excel
      \item ความรู้เชิงธุรกิจ
      \item การสร้าง Dashboard เพื่อสรุปและนำเสนอข้อมูลให้ผู้มีส่วนได้ส่วนเสีย
      \item ทักษะการนำเสนอ (Presentation skill)
      \item การวิเคราะห์ข้อมูลขนาดใหญ่ (Big Data Analysis)
    \end{enumerate}
\end{enumerate} \\ \hline

ศิษย์เก่า & 
\begin{enumerate}
  \item พัฒนาทักษะด้าน Web Technology ทั้ง Front-end และ Back-end เพื่อนำไปประยุกต์ใช้กับ AI/ML
  \item พัฒนาทักษะการนำเสนอและการเล่าเรื่องด้วยข้อมูล (Data Storytelling)
  \item พัฒนาทักษะการสร้างเอกสารทางวิชาการด้วย LaTeX
\end{enumerate} \\ \hline

หน่วยงานราชการและผู้กำกับดูแลหลักสูตร &
\begin{enumerate}
  \item หลักสูตรสอดคล้องกับมาตรฐานคุณวุฒิระดับอุดมศึกษาแห่งชาติ
  \item ปรับปรุงเนื้อหาให้ทันสมัยและตรงตามแนวโน้มเทคโนโลยีและตลาดแรงงาน
  \item ผลงานวิจัย/นวัตกรรมของนักศึกษาเป็นประโยชน์ต่อสังคมและประเทศ
\end{enumerate} \\ \hline

องค์กรวิชาชีพหรือสมาคมวิชาการที่เกี่ยวข้อง &
\begin{enumerate}
  \item หลักสูตรได้รับการสนับสนุนหรือรับรองจากองค์กรวิชาชีพในสาขาคอมพิวเตอร์ วิทยาการข้อมูล หรือปัญญาประดิษฐ์
  \item ผู้สำเร็จการศึกษามีคุณสมบัติเพียงพอสำหรับเข้าทำงานในสาขาอุตสาหกรรมดิจิทัลและวิจัย
\end{enumerate} \\ \hline

\multicolumn{2}{|l|}{\textbf{2. ผู้มีส่วนได้เสียภายในหน่วยงาน}} \\ \hline

มหาวิทยาลัย &
\begin{enumerate}
  \item หลักสูตรส่งเสริมการสร้างและพัฒนานวัตกรรม
  \item นักศึกษามีผลงานตีพิมพ์ในวารสารที่อยู่ในฐานข้อมูล TCI หรือ Scopus
  \item นักศึกษาหรือบุคลากรสามารถสร้างนวัตกรรมที่ได้รับอนุสิทธิบัตรหรือสิทธิบัตร
\end{enumerate} \\ \hline

อาจารย์ &
\begin{enumerate}
  \item สามารถตีพิมพ์งานวิจัยด้านคณิตศาสตร์หรือคณิตศาสตร์ประยุกต์ในวารสารฐานข้อมูล Scopus
  \item เพิ่มจำนวนนักศึกษาระดับบัณฑิตศึกษา
  \item สร้างงานวิจัยร่วมกับภาคอุตสาหกรรมหรือหน่วยงานวิจัยภายนอก
\end{enumerate} \\ \hline

นักศึกษาปัจจุบัน &
\begin{enumerate}
  \item ต้องการความรู้และทักษะด้าน Machine Learning, Deep Learning
  \item ต้องการทักษะการนำเสนอ (Presentation Skill) และทักษะการสื่อสารภาษาอังกฤษ
  \item ต้องการเรียนรู้การจัดการและวิเคราะห์ข้อมูลขนาดใหญ่
\end{enumerate} \\ \hline

หน่วยงานประกันคุณภาพการศึกษาภายในมหาวิทยาลัย &
\begin{enumerate}
  \item หลักสูตรผ่านการประเมินภายในตามเกณฑ์ประกันคุณภาพระดับบัณฑิตศึกษา
  \item ผู้สำเร็จการศึกษามีผลลัพธ์การเรียนรู้ที่สอดคล้องกับมาตรฐานคุณภาพ
\end{enumerate} \\ \hline

คณะกรรมการบริหารหลักสูตร &
\begin{enumerate}
  \item มีการปรับปรุงและพัฒนาหลักสูตรอย่างต่อเนื่องให้ทันสมัย
  \item ตอบสนองต่อข้อเสนอแนะของผู้มีส่วนได้ส่วนเสียทุกกลุ่ม
  \item สร้างเครือข่ายความร่วมมือกับภาคอุตสาหกรรมและสถาบันการศึกษาอื่น ๆ
\end{enumerate} \\ \hline

\end{longtable}





%\subsection{วัตถุประสงค์} 
%\begin{enumerate}
%	\item ...
%\end{enumerate}
%
%\subsection{ผลลัพธ์การเรียนรู้ที่คาดหวังระดับหลักสูตร (Program Learning Outcomes, PLOs)}
%\begin{enumerate}[label=PLOs \arabic*.]
%	\item ...
%	\item ...
%\end{enumerate}
%
%\subsection{ผลลัพธ์การเรียนรู้ที่คาดหวังระดับชั้นปี  (Year Learning Outcomes, YLOs)}
%\begin{tabular}{|c | p{12cm}|}
%\hline	
%ชั้นปีที่  &  ผลลัพธ์การเรียนรู้ที่คาดหวังระดับชั้นปี  (Year Learning Outcomes, YLOs) \\\hline
%1 & \\\hline
%2 & \\\hline
%3 & \\\hline
%\end{tabular}

%\newpage
%\begin{landscape}
%ตารางความสัมพันธ์ระหว่างวัตถุประสงค์ของหลักสูตรและผลลัพธ์การเรียนรู้ที่คาดหวังระดับหลักสูตร
%
%
%
%
%
%\begin{tabular}{ |m{4cm}|p{1.5cm}|p{1.5cm}|p{1.5cm}|p{1.5cm}|p{1.5cm}|p{1.5cm}| } 
%\hline
%\multirow{2}{6cm}{วัตถุประสงค์ของหลักสูตร} & \multicolumn{6}{c|}{\textbf{ผลลัพธ์การเรียนรู้ที่คาดหวังระดับหลักสูตร}} \\ \cline{2-7}
%& PLOs 1 & PLOs 2 & PLOs 3 & PLOs 4  & PLOs 5 & PLOs 6\\ \hline
%1. &   &   &   &     &  &   \\ 
%\hline
%2. &   &   &  &     &   &   \\\hline
%3. &   &   &   &     &   &   \\ \hline
%\end{tabular}
%
%\end{landscape}
%
%\newpage
%\begin{landscape}
%ตารางความเชื่อมโยงระหว่างผลลัพธ์การเรียนรู้ที่คาดหวังของหลักสูตร (PLOs) และผลลัพธaการเรียนรู้ ตามมาตรฐานคุณวุฒิระดับอุดมศึกษา พ.ศ. 2565
%
%\begin{tabular}{|p{5cm} | c| c |c |c | c| c |c |c | c| c |c |c | c| c |c |}
%\hline	
%  \multirow{3}{15em}{ผลลัพธ์การเรียนรู้ที่คาดหวังของหลักสูตร PLOs} & \multicolumn{15}{c|}{\textbf{ผลลัพธ์การเรียนรู้ที่คาดหวังระดับหลักสูตร}}  \\ \cline{2-16} 
%  &\multicolumn{3}{c|}{1.ด้านความรู้ } & \multicolumn{3}{c|}{2.ด้านทักษะ  } & \multicolumn{3}{c|}{3.ด้านจริยธรรม  } & \multicolumn{3}{c|}{4.ลักษณะบุคคล  } & \multicolumn{3}{c|}{5.ด้าน........ }  \\ \cline{2-16}
%  &1.1 &1.2 &1.3 & 2.1 & 2.2& 2.3 &3.1 &3.2 & 3.3 & 4.1 & 4.2& 4.3 & 5.1 &  5.2 & 5.3\\ \hline\hline
%1.  & & & & & & & & & & & & & & &\\ \hline
%2.  & & & & & & & & & & & & & & &\\ \hline
%3.  & & & & & & & & & & & & & & &\\ \hline
%4.  & & & & & & & & & & & & & & &\\ \hline
%\end{tabular}
%
%\end{landscape}


%\newpage
%\section{แผนพัฒนาปรับปรุง}
%
%\begin{tabular}{|p{4cm} | p{4cm}| p{4cm} |}
%\hline	
%แผนการพัฒนา/เปลี่ยนแปลง   &  กลยุทธ์ &  หลักฐาน/ตัวบ่งชี้\\\hline
%1 &  &\\\hline
%2 &  &\\\hline
%3 & & \\\hline
%\end{tabular}



















