\chapter{ปรัชญา วัตถุประสงค์ และผลการเรียนรู้ของหลักสูตร}

\section{ปรัชญา ความสำคัญ และวัตถุประสงค์ของหลักสูตร}

\subsection{ปรัชญาการศึกษาของมหาวิทยาลัยฯ}

นวัตกรรมสร้างชาติ ราชมงคลธัญบุรีสร้างนวัตกรรม

\subsection*{\hspace{10mm} ปรัชญาหลักสูตร}

ผสานความรู้ด้านคณิตศาสตร์ สถิติ และคอมพิวเตอร์ เพื่อสร้างนวัตกรรมเชิงคำนวณที่ตอบโจทย์เศรษฐกิจและสังคมดิจิทัล และผลิตบัณฑิตที่มีศักยภาพในการคิด วิเคราะห์ และพัฒนานวัตกรรมอย่างยั่งยืน

\subsection{ความสำคัญ} 

หลักสูตรวิทยาศาสตรมหาบัณฑิต สาขาวิทยาการเชิงคำนวณถูกออกแบบมาให้สอดคล้องกับเป้าหมายการพัฒนาที่ยั่งยืนของสหประชาชาติ (SDGs) หลายข้อ โดยเฉพาะอย่างยิ่ง เป้าหมายที่ 9 อุตสาหกรรม นวัตกรรม และโครงสร้างพื้นฐาน, เป้าหมายที่ 8 การจ้างงานที่มีคุณค่าและการเติบโตทางเศรษฐกิจ, เป้าหมายที่ 11 เมืองและชุมชนที่ยั่งยืน, และเป้าหมายที่ 17 ความร่วมมือเพื่อการพัฒนาที่ยั่งยืน หลักสูตรมุ่งเน้นในการเสริมสร้างทักษะการคำนวณขั้นสูงให้กับนักศึกษา เพื่อส่งเสริมการนวัตกรรมและความก้าวหน้าทางเทคโนโลยี ซึ่งเป็นสิ่งจำเป็นสำหรับการสร้างโครงสร้างพื้นฐานที่ยืดหยุ่นและส่งเสริมการอุตสาหกรรมที่ครอบคลุมและยั่งยืน บัณฑิตที่สำเร็จการศึกษาจะพร้อมที่จะขับเคลื่อนการเติบโตทางเศรษฐกิจโดยการพัฒนากระบวนการแก้ปัญหาที่มีประสิทธิภาพต่อปัญหาที่ซับซ้อนในอุตสาหกรรมต่าง ๆ ซึ่งนำไปสู่การสร้างงานที่มีคุณค่าและการเพิ่มผลิตภาพ

ในบริบทของเมืองและชุมชนที่ยั่งยืน (เป้าหมายที่ 11) หลักสูตรเสริมสร้างความสามารถให้นักศึกษาสามารถประยุกต์ใช้วิธีการคำนวณและการคิดวิพากษ์ผ่านการสร้างแบบจำลองทางคณิตศาสตร์เพื่อนำไปประยุกต์ใช้การแก้ปัญหาสิ่งแวดล้อม การวางผัง ตลอดจนการจัดการทรัพยากร ซึ่งช่วยให้เกิดการพัฒนาเมืองอัจฉริยะที่ใช้ทรัพยากรอย่างมีประสิทธิภาพ ลดผลกระทบต่อสิ่งแวดล้อม และปรับปรุงคุณภาพชีวิตของผู้อยู่อาศัย โดยการผสานเทคโนโลยีการคำนวณเข้ากับการลงมือจริงจริงในห้องปฏิบัติการ บัณฑิตจะมีส่วนร่วมในการสร้างสภาพแวดล้อมเมืองที่น่าอยู่

นอกจากนี้ หลักสูตรยังเน้นความสำคัญของความร่วมมือระดับโลก (เป้าหมายที่ 17) โดยส่งเสริมการทำงานร่วมกันระหว่างสถาบันการศึกษา อุตสาหกรรม และหน่วยงานรัฐบาล ผ่านโครงการสหวิทยาการและความคิดริเริ่มด้านการวิจัย นักศึกษาจะได้มีส่วนร่วมในการแลกเปลี่ยนความรู้และการแก้ปัญหาแบบร่วมมือกันในระดับโลก ซึ่งไม่เพียงแต่เพิ่มพูนประสบการณ์การศึกษา แต่ยังมีส่วนในการสร้างความร่วมมือที่แข็งแกร่งที่จำเป็นสำหรับการบรรลุ SDGs ซึ่งขยายผลกระทบของหลักสูตรต่อการพัฒนาที่ยั่งยืนทั่วโลก

หลักสูตร\thdegree{} สาขาวิชา\thdegreebranch{} มีความสำคัญอย่างยิ่งในการแก้ไขและลดช่องว่างในสถานการณ์ปัจจุบันของโลก โดยเฉพาะอย่างยิ่งในประเทศไทย ในยุคที่ปัญญาประดิษฐ์ (AI) และการเรียนรู้ของเครื่อง (Machine Learning) มีบทบาทสำคัญในการขับเคลื่อนนวัตกรรมและเศรษฐกิจ การค้นหาอัลกอริทึมและตัวปรับแต่งใหม่ ๆ โดยเฉพาะในแนวทางการเรียนรู้เชิงลึก (Deep Learning) เป็นสิ่งจำเป็นในการพัฒนาระบบที่มีประสิทธิภาพและชาญฉลาด

หลักสูตรนี้มุ่งเน้นการผลิตบัณฑิตที่มีความเชี่ยวชาญในการพัฒนาแบบจำลอง อัลกอริทึม และตัวปรับแต่ง (Optimizers) ในสาขาการเรียนรู้ของเครื่องซึ่งจะช่วยในการแก้ไขปัญหาที่ซับซ้อนและท้าทายที่ประเทศไทยและโลกกำลังเผชิญ เช่น การวิเคราะห์ข้อมูลขนาดใหญ่ การประมวลผลภาษาธรรมชาติ และการประยุกต์ใช้ปัญญาประดิษฐ์ในภาคอุตสาหกรรมต่าง ๆ ด้วยการเสริมสร้างความรู้และทักษะในด้านนี้ บัณฑิตจะสามารถสร้างสรรค์นวัตกรรมที่มีผลกระทบสูง และสนับสนุนการตัดสินใจที่มีข้อมูลเป็นฐาน

นอกจากนี้ หลักสูตรยังช่วยเสริมสร้างศักยภาพของประเทศไทยในการเป็นผู้พัฒนานวัตกรรมด้านปัญญาประดิษฐ์ โดยการส่งเสริมการวิจัยและพัฒนาในสาขาที่กำลังเติบโตนี้ ด้วยการสนับสนุนให้นักศึกษามีความคิดสร้างสรรค์และความสามารถในการพัฒนาเทคโนโลยีใหม่ ๆ ประเทศไทยจะสามารถเพิ่มขีดความสามารถในการแข่งขันบนเวทีโลก และตอบสนองต่อความต้องการของตลาดแรงงานที่ต้องการบุคลากรที่มีทักษะสูงในด้านนี้

เพื่อเสริมสร้างความสำคัญของหลักสูตรนี้ต่อแนวโน้มการเลือกศึกษาของนักศึกษา หลักสูตรวิทยาศาสตรมหาบัณฑิต สาขาวิทยาการเชิงคำนวณ ตอบสนองต่อความสนใจที่เพิ่มขึ้นของนักศึกษาในด้านเทคโนโลยีขั้นสูงในด้านปัญญาประดิษฐ์ (AI) และการเรียนรู้ของเครื่อง (Machine Learning)

แนวโน้มการเลือกศึกษาของนักศึกษาแสดงถึงความสนใจในสาขาเทคโนโลยีและการคำนวณที่เพิ่มขึ้น รายงานจากหลายแหล่งระบุว่ามีนักศึกษาสมัครเข้าเรียนในสาขาวิทยาศาสตร์คอมพิวเตอร์ วิศวกรรมคอมพิวเตอร์ และสาขาที่เกี่ยวข้องกับ AI และการเรียนรู้ของเครื่องมากขึ้น เนื่องจากเห็นถึงโอกาสในการทำงานที่กว้างขวางและความต้องการบุคลากรในตลาดแรงงานที่เพิ่มขึ้น

นักศึกษามองหาหลักสูตรที่มีความเกี่ยวข้องกับตลาดงานและมีโอกาสการทำงานสูง สาขาวิชาที่เกี่ยวข้องกับ AI การเรียนรู้ของเครื่อง และวิทยาการข้อมูล (Data Science) ถูกจัดอันดับให้เป็นสาขาที่มีศักยภาพสูง ทั้งในด้านเงินเดือนและการเติบโตในสายอาชีพ หลักสูตรที่เน้นการพัฒนาอัลกอริทึมและตัวปรับแต่งใหม่ ๆ ในการเรียนรู้ของเครื่องตรงกับความต้องการของนักศึกษาที่ต้องการความท้าทายและการสร้างสรรค์นวัตกรรม

หลักสูตรนี้ยังสอดคล้องกับแนวโน้มการศึกษาระดับสูงที่มุ่งเน้นการวิจัยและพัฒนาเทคโนโลยีใหม่ นักศึกษาที่สนใจในการแก้ไขปัญหาที่ซับซ้อนและการมีส่วนร่วมในการพัฒนาเทคโนโลยีที่มีผลกระทบสูง จะถูกดึงดูดโดยหลักสูตรที่ให้โอกาสในการทำวิจัยและพัฒนาอัลกอริทึมใหม่ ๆ


\clearpage
\subsection{ความต้องการและความคาดหวังของผู้มีส่วนได้ส่วนเสีย}

\begin{longtable}{|p{0.1\textwidth} | p{0.4\textwidth}| p{0.4\textwidth} |}
\hline
\textbf{ผู้มีส่วนได้ส่วนเสีย} & \textbf{ความต้องการ/ความคาดหวังของผู้มีส่วนได้ส่วนเสีย} & \textbf{การพัฒนาไปสู่ผลลัพธ์การเรียนรู้ที่คาดหวังระดับหลักสูตร}\\ \hline
\endhead

\hline
\endfoot

\multicolumn{3}{|l|}{\textbf{1. ผู้มีส่วนได้เสียภายในหน่วยงาน}} \\ \hline

คณะวิทยาศาสตร์และเทคโนโลยี & 

1.ผลิตผลงานวิจัย สร้างสรรค์เทคโนโลยีและนวัตกรรมเพื่อการพัฒนาประเทศ
& PLO4: สร้างหรือพัฒนางานวิจัยด้านการเรียนรู้ของเครื่องหรือการคำนวณเชิงคณิตศาสตร์ที่เกี่ยวข้อง (สายวิชาการ) / ประมวลงานวิจัยด้านการเรียนรู้ของเครื่อง หรือการคำนวณเชิงคณิตศาสตร์ที่เกี่ยวข้อง (สายวิชาชีพ)\newline 
PLO7: ใช้เทคโนโลยี และอัลกอริทึม สำหรับการเรียนรู้ของเครื่องในการสร้างสรรค์หรือพัฒนานวัตกรรม \\ \cline{2-3}
&2. ยกระดับการดำเนินการและการพัฒนาหลักสูตรให้สอดคล้องกับมาตรฐานหลักสูตรระดับบัณฑิตศึกษา โดยมุ่งพัฒนาหลักสูตรตามแนว Outcome Based Education
& PLO4: สร้างหรือพัฒนางานวิจัยด้านการเรียนรู้ของเครื่องหรือการคำนวณเชิงคณิตศาสตร์ที่เกี่ยวข้อง (สายวิชาการ) / ประมวลงานวิจัยด้านการเรียนรู้ของเครื่อง หรือการคำนวณเชิงคณิตศาสตร์ที่เกี่ยวข้อง (สายวิชาชีพ)\newline 
PLO7: ใช้เทคโนโลยี และอัลกอริทึม สำหรับการเรียนรู้ของเครื่องในการสร้างสรรค์หรือพัฒนานวัตกรรม \\ \hline

อาจารย์/ผู้รับผิดชอบหลักสูตร & 
1. การผลิตกำลังคนด้านการเรียนรู้ของเครื่องหรือการคำนวณเชิงคณิตศาสตร์ที่มีสมรรถนะตรงตามมาตรฐานที่สอดรับกับบัณฑิตศึกษา
& PLO4: สร้างหรือพัฒนางานวิจัยด้านการเรียนรู้ของเครื่องหรือการคำนวณเชิงคณิตศาสตร์ที่เกี่ยวข้อง (สายวิชาการ) / ประมวลงานวิจัยด้านการเรียนรู้ของเครื่อง หรือการคำนวณเชิงคณิตศาสตร์ที่เกี่ยวข้อง (สายวิชาชีพ)\\ \cline{2-3}
&2. มีความรู้เกี่ยวกับการเรียนรู้ของเครื่อง & PLO1: อธิบายแนวคิด ทฤษฎี และหลักการของการเรียนรู้ของเครื่องได้ \\ \cline{2-3}
&3. มีความรู้เกี่ยวกับกระบวนการคณิตศาสตร์เชิงคำนวณ 
นักศึกษาปัจจุบัน & PLO2: อธิบายแนวคิด ทฤษฎี และหลักการของกระบวนการคณิตศาสตร์เชิงคำนวณได้ \\ \cline{2-3}
&4. มีความสามารถในเลือกใช้เทคโนโลยีสารสนเทศที่เกี่ยวข้องกับการเรียนรู้ของเครื่องและการคำนวณเชิงคณิตศาสตร์ได้อย่างเหมาะสม & PLO6: ใช้เทคโนโลยีสารสนเทศในการรวบรวม วิเคราะห์ สังเคราะห์ และนำเสนอข้อมูลได้ \newline PLO7: ใช้เทคโนโลยี และอัลกอริทึม สำหรับการเรียนรู้ของเครื่องในการสร้างสรรค์หรือพัฒนานวัตกรรม \\ \cline{2-3}
&5. มีความสามารถในถ่ายทอดองค์ความรู้เพื่อพัฒนาบุคลากรให้มีทักษะและความเชี่ยวชาญทางด้านการเรียนรู้ของเครื่องหรือการคำนวณเชิงคณิตศาสตร์ & PLO8: สื่อสาร และนำเสนอ องค์ความรู้จากการศึกษาด้านการเรียนรู้ของเครื่องหรือการคำนวณเชิงคณิตศาสตร์ที่เกี่ยวข้องได้ \\ \cline{2-3}
&6.  มีความสามารถในการออกแบบการวิจัยทางด้านการเรียนรู้ของเครื่องหรือการคำนวณเชิงคณิตศาสตร์ได้อย่างถูกต้อง เพื่อพัฒนาศักยภาพนักวิจัยรุ่นใหม่ให้มีความเชี่ยวชาญในการดำเนินงานวิจัยเชิงลึกได้อย่างมีประสิทธิภาพ & PLO3: ออกแบบการวิจัยได้ถูกต้องตามกระบวนการการทำวิจัยทางวิทยาศาสตร์ได้ \\ \cline{2-3}
&7. มีระเบียบวินัยและรับผิดชอบ	 & PLO10: รู้จักบทบาทหน้าที่และมีความรับผิดชอบ \\ \cline{2-3}
&8. มีจริยธรรมทางวิชาการ &	PLO9: ผลิตผลงานทางวิชาการ หรือพัฒนานวัตกรรม ด้วยจริยธรรมทางวิชาการ \\ \hline


\multicolumn{3}{|l|}{\textbf{2. ผู้มีส่วนได้เสียภายนอกหน่วยงาน}} \\ \hline

สถานประกอบการ/ \newline อุตสาหกรรม\newline /หน่วยงานภาครัฐ	
&1. สร้างหรือพัฒนาแบบจำลองทางคณิตศาสตร์สำหรับการเรียนรู้ของเครื่องที่เหมาะสมกับชุดข้อมูลที่กำหนด โดยคำนึงถึงประสิทธิภาพและความแม่นยำในการทำนาย
&PLO4: สร้างหรือพัฒนางานวิจัยด้านการเรียนรู้ของเครื่องหรือการคำนวณเชิงคณิตศาสตร์ที่เกี่ยวข้อง (สายวิชาการ) / ประมวลงานวิจัยด้านการเรียนรู้ของเครื่อง หรือการคำนวณเชิงคณิตศาสตร์ที่เกี่ยวข้อง (สายวิชาชีพ) \\ \cline{2-3}
&2. มีความสามารถในการประยุกต์ใช้เทคโนโลยีสารสนเทศที่เกี่ยวข้องกับการเรียนรู้ของเครื่อง ทั้งในด้านการจัดการข้อมูล การวิเคราะห์ และการนำเสนอข้อมูลได้อย่างมีประสิทธิภาพ	 & PLO6: ใช้เทคโนโลยีสารสนเทศในการรวบรวม วิเคราะห์ สังเคราะห์ และนำเสนอข้อมูลได้ \\ \cline{2-3}
&3. มีความสามารถในการสื่อสารและนำเสนอข้อมูลทางด้านการเรียนรู้ของเครื่องหรือการคำนวณเชิงคณิตศาสตร์ได้อย่างถูกต้อง	& PLO8: สื่อสาร และนำเสนอ องค์ความรู้จากการศึกษาด้านการเรียนรู้ของเครื่องหรือการคำนวณเชิงคณิตศาสตร์ที่เกี่ยวข้องได้ \\ \cline{2-3}
&4. มีความสามารถในการประยุกต์ใช้เครื่องมือด้านการเรียนรู้ของเครื่องหรือการคำนวณเชิงคณิตศาสตร์ที่เหมาะสมในการแก้ไขปัญหาจริงได้อย่างมีประสิทธิภาพ & PLO5: ประยุกต์ความรู้ด้านการเรียนรู้ของเครื่อง หรือการคำนวณเชิงคณิตศาสตร์ที่เกี่ยวข้อง ในการแก้ปัญหาจริงได้อย่างเหมาะสม \\ \cline{2-3}
&5. มีความสามารถในการวิเคราะห์และเลือกใช้อัลกอริทึมการเรียนรู้ของเครื่องที่เหมาะสมกับลักษณะและโครงสร้างของชุดข้อมูลได้อย่างมีประสิทธิภาพ &	 PLO7: ใช้เทคโนโลยี และอัลกอริทึม สำหรับการเรียนรู้ของเครื่องในการสร้างสรรค์หรือพัฒนานวัตกรรม \\ \cline{2-3}
&6. การทำงานร่วมกับผู้อื่น &	PLO10: รู้จักบทบาทหน้าที่และมีความรับผิดชอบ \\ \cline{2-3} 
&7. มีระเบียบวินัยและรับผิดชอบ	 & PLO10: รู้จักบทบาทหน้าที่และมีความรับผิดชอบ \\ \hline

ศิษย์เก่า & 
1.มีความรู้ทันสมัยและเชี่ยวชาญในศาสตร์การเรียนรู้ของเครื่องหรือการคำนวณเชิงคณิตศาสตร์	& PLO4: สร้างหรือพัฒนางานวิจัยด้านการเรียนรู้ของเครื่องหรือการคำนวณเชิงคณิตศาสตร์ที่เกี่ยวข้อง (สายวิชาการ) / ประมวลงานวิจัยด้านการเรียนรู้ของเครื่อง หรือการคำนวณเชิงคณิตศาสตร์ที่เกี่ยวข้อง (สายวิชาชีพ) 
\\ \cline{2-3}
&2. มีความสามารถในการใช้เทคโนโลยีสารสนเทศในการวิเคราะห์และนำเสนอข้อมูลได้	&PLO6: ใช้เทคโนโลยีสารสนเทศในการรวบรวม วิเคราะห์ สังเคราะห์ และนำเสนอข้อมูลได้ \\ 

\end{longtable}


\subsection{วัตถุประสงค์}
	เพื่อผลิตมหาบัณฑิตที่มีคุณสมบัติ ดังนี้
\begin{enumerate}
%	\item มีความรับผิดชอบต่อตนเอง รู้จักบทบาทหน้าที่ มีคุณธรรมและจริยธรรม
%	\item มีความรู้ความเข้าใจในหลักการของการพัฒนาแบบจำลอง อัลกอริทึม และตัวปรับแต่ง (Optimizers)  ทางด้านการเรียนรู้ของเครื่อง (Machine Learning)และ การเรียนรู้เชิงลึก (Deep Learning)
%	\item มีความคิดริเริ่มในการพัฒนา สร้างองค์ความรู้ใหม่ เพื่อพัฒนางานวิจัยทางด้านวิทยาการเชิงคำนวณทางด้านการเรียนรู้ของเครื่อง (Machine Learning) และการเรียนรู้เชิงลึก (Deep Learning)
%	\item สามารถประยุกต์ใช้ความรู้ทางด้านการเรียนรู้ของเครื่อง (Machine Learning)และการเรียนรู้เชิงลึก (Deep Learning)เพื่อหาคำตอบการวิจัยได้ถูกต้องตามหลักวิชาการ
%	\item สามารถเรียนรู้ได้ด้วยตนเอง สามารถคิดและแก้ปัญหาอย่างเป็นระบบและเป็นเหตุเป็นผล
%	\item มีทักษะทางสังคมทั้งในด้านของการสื่อสาร สามารถทำงานร่วมกับผู้อื่นได้อย่างมีประสิทธิภาพ
	\item มีความรู้ ความสามารถ และทักษะในการประยุกต์ พัฒนาคิดค้น และวิจัยด้านด้านการเรียนรู้ของเครื่องหรือการคำนวณเชิงคณิตศาสตร์อย่างมีคุณภาพ และมีประสิทธิภาพ ตามความต้องการของภาครัฐและภาคเอกชนในปัจจุบัน ให้มีคุณภาพในระดับสากล
	\item สามารถวิเคราะห์ สังเคราะห์ และนำเสนอข้อมูลเชิงลึก โดยใช้เทคนิคการคำนวณเชิงคณิตศาสตร์ เทคโนโลยีสารสนเทศและการเรียนรู้ของเครื่องเพื่อพัฒนาความรู้และนำไปใช้ประโยชน์ได้ในภาครัฐและภาคเอกชน
	\item มีคุณธรรม จริยธรรม และจรรยาบรรณทางวิชาการ/วิชาชีพ 
\end{enumerate}


\subsection{ผลลัพธ์การเรียนรู้ที่คาดหวังระดับหลักสูตร (Program Learning Outcomes, PLOs)}
\begin{enumerate}[label=PLO\arabic*., leftmargin=3\parindent]
	\item อธิบายแนวคิด ทฤษฎี และหลักการ ของการเรียนรู้ของเครื่องได้
	\item อธิบายแนวคิด ทฤษฎี และหลักการ ของกระบวนการคณิตศาสตร์เชิงคำนวณได้
	\item ออกแบบการวิจัยได้ถูกต้องตามกระบวนการการทำวิจัยทางวิทยาศาสตร์ 
	\item สร้างหรือพัฒนางานวิจัยด้านการเรียนรู้ของเครื่องหรือการคำนวณเชิงคณิตศาสตร์ที่เกี่ยวข้อง (สายวิชาการ)/ ประมวลงานวิจัยด้านการเรียนรู้ของเครื่อง หรือการคำนวณเชิงคณิตศาสตร์ที่เกี่ยวข้อง (สายวิชาชีพ) 
	\item ประยุกต์ความรู้ด้านการเรียนรู้ของเครื่อง หรือการคำนวณเชิงคณิตศาสตร์ที่เกี่ยวข้อง ในการแก้ปัญหาจริงได้อย่างเหมาะสม
	\item ใช้เทคโนโลยีสารสนเทศในการรวบรวม วิเคราะห์ สังเคราะห์ และนำเสนอข้อมูลได้
	\item ใช้เทคโนโลยี และอัลกอริทึม สำหรับการเรียนรู้ของเครื่องในการสร้างสรรค์หรือพัฒนานวัตกรรม
	\item สื่อสาร และนำเสนอ องค์ความรู้จากการศึกษาด้านการเรียนรู้ของเครื่องหรือการคำนวณเชิงคณิตศาสตร์ที่เกี่ยวข้องได้
	\item ผลิตผลงานทางวิชาการ หรือพัฒนานวัตกรรม ด้วยจริยธรรมทางวิชาการ
	\item รู้จักบทบาทหน้าที่และมีความรับผิดชอบ
\end{enumerate}

\subsection{ผลลัพธ์การเรียนรู้ที่คาดหวังระดับชั้นปี  (Year Learning Outcomes, YLOs)}

\begin{longtable}{|C{0.08\textwidth}|p{0.7\textwidth}|C{0.12\textwidth} |}
\hline	
ชั้นปีที่  &  ผลลัพธ์การเรียนรู้ที่คาดหวังระดับชั้นปี  (Year Learning Outcomes, YLOs) & ผลลัพธ์การเรียนรู้ของหลักสูตร \\ \hline
\endhead

\hline
\endfoot
%------------------ Contents
1 & YLO1.1 : อธิบายแนวคิด ทฤษฎี และหลักการของอัลกอริทึมสำหรับการเรียนรู้ของเครื่องได้อย่างถูกต้อง & PLO1 \\
 &	YLO1.2 : อธิบายแนวคิด ทฤษฎี และหลักการทางคณิตศาสตร์เชิงคำนวณ & PLO2 \\
& YLO1.3 : มีทักษะในการใช้เทคโนโลยีสารสนเทศเพื่อการรวบรวม วิเคราะห์ สังเคราะห์และนำเสนอข้อมูลด้านการเรียนรู้ของเครื่องหรือการคำนวณเชิงคณิตศาสตร์ได้อย่างถูกต้อง & PLO6 \\
& YLO1.4 : มีความสามารถในการสื่อสารทั้งการพูด การเขียน และนำเสนอองค์ความรู้จากการศึกษาด้านการเรียนรู้ของเครื่องหรือการคำนวณเชิงคณิตศาสตร์ที่เกี่ยวข้องได้อย่างถูกต้องตามหลักวิชาการ & PLO8 \\
& YLO1.5 : ตระหนักถึงบทบาทหน้าที่ของตนเอง และมีความรับผิดชอบต่องานที่ได้รับมอบหมาย & PLO10 \\
\hline
2 & YLO2.1 : อธิบายแนวคิด ทฤษฎี และหลักการของอัลกอริทึมสำหรับการเรียนรู้ของเครื่องได้อย่างถูกต้อง  & PLO1 \\
& YLO2.2 : อธิบายแนวคิด ทฤษฎี และหลักการทางคณิตศาสตร์เชิงคำนวณ ในด้านการเรียนรู้ของเครื่องและการหาค่าเหมาะที่สุดเชิงคำนวณได้อย่างถูกต้อง  & PLO2 \\
& YLO2.3 : สามารถวางแผน ออกแบบ และดำเนินการวิจัยที่เกี่ยวข้องกับการเรียนรู้ของเครื่องหรือการคำนวณเชิงคณิตศาสตร์ได้อย่างเป็นระบบ โดยใช้ระเบียบวิธีวิจัยทางวิทยาศาสตร์อย่างถูกต้อง& PLO3 \\
& YLO2.4 : สร้างหรือพัฒนาองค์ความรู้ใหม่ผ่านการวิจัยด้านการเรียนรู้ของเครื่องหรือการคำนวณเชิงคณิตศาสตร์ที่เกี่ยวข้องได้อย่างถูกต้อง (สายวิชาการ) / ประมวลงานวิจัยด้านการเรียนรู้ของเครื่อง หรือการคำนวณเชิงคณิตศาสตร์ที่เกี่ยวข้องได้อย่างถูกต้อง (สายวิชาชีพ) & PLO4 \\
& YLO2.5 : สามารถนำหลักการและเทคนิคของการเรียนรู้ของเครื่องหรือการคำนวณเชิงคณิตศาสตร์มาประยุกต์ใช้ในการวิเคราะห์และแก้ปัญหาจริงที่ซับซ้อนได้อย่างมีประสิทธิภาพและเหมาะสมกับบริบทของปัญหานั้น ๆ & PLO5 \\
& YLO2.6 : มีทักษะในการใช้เทคโนโลยีสารสนเทศเพื่อการรวบรวม วิเคราะห์ สังเคราะห์ และนำเสนอข้อมูลด้านการหาค่าเหมาะที่สุดเชิงคำนวณและการเรียนรู้ของเครื่องได้อย่างถูกต้อง & PLO6 \\
& YLO2.7 สามารถเลือกใช้เครื่องมือและอัลกอริทึมด้านการเรียนรู้ของเครื่องเพื่อการสร้างสรรค์หรือพัฒนานวัตกรรมได้อย่างมีประสิทธิภาพ & PLO7 \\
& YLO2.8 : มีความสามารถในการสื่อสารทั้งการพูด การเขียน  และนำเสนอองค์ความรู้จากการศึกษาด้านการเรียนรู้ของเครื่องหรือการคำนวณเชิงคณิตศาสตร์ที่เกี่ยวข้องได้อย่างถูกต้องตามหลักวิชาการ & PLO8 \\
& YLO2.9 : สามารถทำวิจัย เขียนบทความวิชาการ เขียนบทความวิจัยหรือพัฒนานวัตกรรมในด้านการเรียนรู้ของเครื่องและการคำนวณเชิงคณิตศาสตร์ตามหลักจริยธรรมทางวิชาการ & PLO9 \\
& YLO2.10 : ตระหนักถึงบทบาทหน้าที่ของตนเอง และมีความรับผิดชอบต่อการทำวิทยานิพนธ์ & PLO10 \\
\end{longtable}

\newpage
\begin{landscape}
\begin{center}
\textbf{ตารางความสัมพันธ์ระหว่างวัตถุประสงค์ของหลักสูตรและผลลัพธ์การเรียนรู้ที่คาดหวังระดับหลักสูตร}

\par\noindent\bigskip

\begin{tabular}{ |m{0.6\textwidth}|p{0.055\textwidth}|p{0.055\textwidth}|p{0.055\textwidth}|p{0.055\textwidth}|p{0.055\textwidth}|p{0.055\textwidth}| p{0.055\textwidth}|p{0.055\textwidth}|p{0.055\textwidth}|p{0.06\textwidth}|} 
\hline
\multirow{2}{0.4\textwidth}{วัตถุประสงค์ของหลักสูตร} & \multicolumn{10}{c|}{\textbf{ผลลัพธ์การเรียนรู้ที่คาดหวังระดับหลักสูตร}} \\ \cline{2-11}
& PLO1 & PLO2 & PLO3 & PLO4  & PLO5 & PLO6& PLO7& PLO8& PLO9& PLO10 \\ \hline
1. มีความรู้ ความสามารถ และทักษะในการประยุกต์ พัฒนาคิดค้น และวิจัยด้านด้านการเรียนรู้ของเครื่องหรือการคำนวณเชิงคณิตศาสตร์อย่างมีคุณภาพ และมีประสิทธิภาพ ตามความต้องการของภาครัฐและภาคเอกชนในปัจจุบัน ให้มีคุณภาพในระดับสากล \newline 
	 & \ding{51}  & \ding{51}  & \ding{51}  & \ding{51}    & \ding{51} & & & & &  \\ 
\hline
2. สามารถวิเคราะห์ สังเคราะห์ และนำเสนอข้อมูลเชิงลึก โดยใช้เทคนิคการคำนวณเชิงคณิตศาสตร์ เทคโนโลยีสารสนเทศและการเรียนรู้ของเครื่องเพื่อพัฒนาความรู้และนำไปใช้ประโยชน์ได้ในภาครัฐและภาคเอกชน \newline 
	&   &   &  &     &   & \ding{51} &\ding{51} &\ding{51} & &  \\\hline
3. มีคุณธรรม จริยธรรม และจรรยาบรรณทางวิชาการ/วิชาชีพ \newline &   &   &   &     &   &  & & & \ding{51}& \ding{51} \\ \hline
\end{tabular}
\end{center}
\end{landscape}

\newpage
\begin{landscape}
\begin{center}
\textbf{ตารางความเชื่อมโยงระหว่างผลลัพธ์การเรียนรู้ที่คาดหวังของหลักสูตร (PLOs) และผลลัพธaการเรียนรู้ ตามมาตรฐานคุณวุฒิระดับอุดมศึกษา พ.ศ. 2565}

\phantom{x}
\par\noindent\bigskip
\renewcommand{\arraystretch}{1.4}
\begin{tabular}{|C{0.2\linewidth} | C{0.05\linewidth}| C{0.05\linewidth} |C{0.05\linewidth} |C{0.05\linewidth} | C{0.05\linewidth}| C{0.05\linewidth} |C{0.05\linewidth} |C{0.05\linewidth} | C{0.05\linewidth}| C{0.05\linewidth} |}
\hline	
  \multirow{3}{0.2\textwidth}{ผลลัพธ์การเรียนรู้ที่คาดหวังของหลักสูตร PLOs} & \multicolumn{10}{C{0.5\textwidth}|}{\textbf{ผลลัพธ์การเรียนรู้ที่คาดหวังระดับหลักสูตร}}  \\ \cline{2-11} 
  &\multicolumn{5}{C{0.25\textwidth}|}{1.ด้านความรู้ } & \multicolumn{3}{C{0.15\textwidth}|}{2.ด้านทักษะ  } & 3.ด้านจริยธรรม  & 4.ลักษณะบุคคล     \\ \cline{2-11}
  &1.1 &1.2 &1.3 &1.4 &1.5 &2.1 &2.2 &2.3 &3.1 & 4.1 \\ \hline
PLO1.  &\ding{51} & & & & & & & & & \\ \hline
PLO2.  & &\ding{51} & & & & & & & & \\ \hline
PLO3.  & & &\ding{51} & & & & & & & \\ \hline
PLO4.  & & & &\ding{51} & & & & & & \\ \hline
PLO5.  & & & & &\ding{51} & & & & & \\ \hline
PLO6.  & & & & & &\ding{51} & & & & \\ \hline
PLO7.  & & & & & & &\ding{51} & & & \\ \hline
PLO8.  & & & & & & & &\ding{51} & & \\ \hline
PLO9.  & & & & & & & & &\ding{51} & \\ \hline
PLO10.  & & & & & & & & & &\ding{51} \\ \hline
\end{tabular}	
\end{center}	
\end{landscape}



\newpage
\section{แผนพัฒนาปรับปรุง}
หลักสูตร\thdegree\,สาขาวิชา\thdegreebranch มีแผนการพัฒนาปรับปรุงเป็นประจำทุกปี ดังรายละเอียดแผนการพัฒนา/เปลี่ยนแปลง กลยุทธ์ และตัวบ่งชี้การพัฒนาปรับปรุง ซึ่งคาดว่าจะดำเนินการแล้วเสร็จภายในระยะเวลา 5 ปี นับจากเปิดการเรียนการสอนตามหลักสูตรดังนี้

\begin{longtable}{|p{0.2\textwidth} | p{0.25\textwidth}| p{0.45\textwidth} |}
\hline	
\multicolumn{1}{|p{0.2\textwidth}|}{\textbf{แผนการพัฒนา/เปลี่ยนแปลง}} &
	\multicolumn{1}{p{0.25\textwidth}|}{\textbf{กลยุทธ์}} &
	\multicolumn{1}{p{0.45\textwidth}|}{\textbf{หลักฐาน/ตัวบ่งชี้}}\\
	\hline
\endhead

1. ปรับปรุงหลักสูตรให้เหมาะสมและสอดคล้องกับความต้องการของผู้ใช้บัณฑิต และเป็นไปตามเกณฑ์มาตรฐานที่ สป.อว. กำหนด
&1. สร้างเครือข่ายความร่วมมือด้านวิชาการและวิจัยกับหน่วยงานภาครัฐและเอกชน ทั้งในประเทศและต่างประเทศ  
&\underline{ตัวบ่งชี้} \newline 
1. จำนวนสถาบันอุดมศึกษาที่เข้าร่วมเครือข่ายทั้งในและต่างประเทศ/หน่วยงานภาครัฐและเอกชน ไม่น้อยกว่า 3 หน่วยงาน \newline 
2. จำนวนครั้งในการประชุมร่วมกันไม่น้อยกว่า 1 ครั้ง/ปี \newline 
\underline{หลักฐาน} \newline
1. รายงานการประชุม \newline
2. เอกสารการลงนามความร่วมมือ \\\hline
& 2. สำรวจความต้องการมหาบัณฑิต สาขาวิชา\thdegreebranch ของตลาดแรงงานจากสถานประกอบการต่าง ๆ
&\underline{ตัวบ่งชี้} \newline
1. จำนวนครั้งในการสำรวจไม่น้อยกว่า 1 ครั้ง ภายในรอบ 5 ปี \newline
2. รายงานการสำรวจความคิดเห็น แสดงข้อมูลอย่างน้อย 3 ประเด็น คือ \newline
   2.1 ความต้องการของหน่วยงานต่อแผนที่จะรับผู้สำเร็จการศึกษา หลักสูตร\thdegree\,สาขาวิชา\thdegreebranch \newline
   2.2 ความคิดเห็นของหน่วยงานต่อเนื้อหาของหลักสูตร\thdegree\,สาขาวิชา\thdegreebranch \newline
   2.3 ความคิดเห็นของหน่วยงานต่อคุณลักษณะมหาบัณฑิต หลักสูตร\thdegree\,สาขาวิชา\thdegreebranch \newline
\underline{หลักฐาน} \newline
รายงานสรุปผลการสำรวจความคิดเห็นของสถานประกอบการต่อเนื้อหา คุณลักษณะ และความต้องการต่อหลักสูตร\thdegree\,สาขาวิชา\thdegreebranch
\\\hline
& 3. สำรวจความพึงพอใจของอาจารย์และนักศึกษาต่อหลักสูตร\thdegree\,สาขาวิชา\thdegreebranch 
&\underline{ตัวบ่งชี้} \newline
1. จำนวนครั้งในการสำรวจไม่น้อยกว่า 1 ครั้ง ภายในรอบ 1 ปี \newline
2. รายงานการสำรวจความพึงพอใจต่อหลักสูตรอย่างน้อย 2 ประเด็นคือ \newline
   2.1 ด้านเนื้อหาของหลักสูตร \newline
   2.2 ด้านการจัดการเรียนการสอน \newline
\underline{หลักฐาน} \newline
รายงานสรุปการสำรวจความพึงพอใจของอาจารย์และนักศึกษาต่อหลักสูตร\thdegree\,สาขาวิชา\thdegreebranch
  \\ \hline
  
& 4. สำรวจความพึงพอใจของผู้บังคับบัญชา/หัวหน้างานของผู้สำเร็จการศึกษาหลักสูตร\thdegree\,สาขาวิชา\thdegreebranch 
&\underline{ตัวบ่งชี้} \newline
1. จำนวนครั้งในการสำรวจไม่น้อยกว่า 1 ครั้ง ภายในรอบ 1 ปี \newline
2. รายงานการสำรวจความคิดเห็นเกี่ยวกับการปฏิบัติงานและคุณลักษณะของมหาบัณฑิต อย่างน้อย 3 ประเด็น คือ \newline
   2.1 ด้านความรู้ความสามารถทางวิชาการและการปฏิบัติงาน \newline
   2.2 ด้านบุคลิกภาพในการปฏิบัติงาน \newline
   2.3 ด้านวุฒิภาวะ คุณธรรม และจริยธรรม \newline
\underline{หลักฐาน} \newline
รายงานแบบสอบถามผู้บังคับบัญชา/หัวหน้างานของผู้สำเร็จการศึกษา  \\\hline

2. ปรับปรุงปัจจัยสนับสนุนการเรียนการสอน
& 1. สำรวจความต้องการของนักศึกษาและอาจารย์ผู้สอนเกี่ยวกับปัจจัยสนับสนุนการเรียนการสอน
&\underline{ตัวบ่งชี้} \newline
1. จำนวนครั้งในการสำรวจไม่น้อยกว่า 1 ครั้ง/ปี \newline
2. รายงานความต้องการโดยแสดงข้อมูลอย่างน้อย 5 ประเด็น คือ \newline
   2.1 บริการด้านสิ่งอำนวยความสะดวกที่เอื้อต่อการเรียนรู้ \newline
   2.2 บริการด้านกายภาพเพื่อส่งเสริมคุณภาพชีวิต \newline
   2.3 บริการด้านให้คำปรึกษา \newline
   2.4 บริการข้อมูลข่าวสารที่เป็นประโยชน์ \newline
   2.5 บริการเพื่อพัฒนาประสบการณ์ทางวิชาชีพ \newline
\underline{หลักฐาน} \newline
รายงานความต้องการของนักศึกษาและอาจารย์ผู้สอนเกี่ยวกับปัจจัยสนับสนุนการเรียนการสอน
  \\\hline
& 2. จัดหาและจัดสรรทุนเพื่อปรับปรุงปัจจัยสนับสนุนการเรียนการสอน เช่น วัสดุ ครุภัณฑ์ โสตทัศนูปกรณ์ อาคาร และห้องสมุด ให้มีความทันสมัยและมีประสิทธิภาพยิ่งขึ้น
&\underline{ตัวบ่งชี้} \newline
จัดทำคำเสนอขอครุภัณฑ์ในแต่ละปี \newline
\underline{หลักฐาน} \newline
1. คำเสนอขอครุภัณฑ์ในแต่ละปี \newline
2. จำนวนครุภัณฑ์ที่ได้รับจัดสรร \newline
  \\\hline
\end{longtable}






