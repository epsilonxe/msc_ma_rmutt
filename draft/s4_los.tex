
\chapter{ผลการเรียนรู้ กลยุทธ์การสอนและการประเมินผล}

\section{การพัฒนาคุณลักษณะพิเศษของนักศึกษา}

\renewcommand{\arraystretch}{1.2}
\begin{tabular}{|p{0.45\textwidth}|p{0.45\textwidth}|}
\hline
\textbf{คุณลักษณะพิเศษ} & \textbf{กลยุทธ์หรือกิจกรรมของนักศึกษา} \\ 
\hline	

 มีทักษะทางด้านวิชาชีพ & 1.ส่งเสริมให้นักศึกษามีการพัฒนาทักษะการเรียนรู้ การวิเคราะห์ข้อมูล การนำความรู้ด้านต่าง ๆ มาบูรณาการร่วมกับวิชาชีพ\\
 & 2.มีรายวิชาที่ส่งเสริมให้นักศึกษาได้ฝึกทักษะให้เกิดความเชี่ยวชาญทางด้านวิทยาการเชิงคำนวณเพื่อนำมาประยุกต์ใช้ให้เกิดประโยชน์ต่อสังคมและการประกอบอาชีพ  \\  \hline
 มีทักษะทางด้านคุณธรรม จริยธรรม และมีภาวะความเป็นผู้นำและผู้ตาม & 1.ให้ความรู้ในการทำงานวิจัยที่ดี ส่งเสริมให้นักศึกษามีความซื่อสัตย์ต่อการรายงานผลงานวิจัย มีความเสียสละ มีคุณธรรม จริยธรรม และจรรยาบรรณในการประกอบวิชาชีพ \\
 & 2.มีกิจกรรมที่ส่งเสริมการแสดงออกและฝึกทักษะความเป็นผู้นำและผู้ตามของนักศึกษาในสถานการณ์ต่าง ๆ \\ \hline
 มีทักษะทางด้านการสื่อสาร การใช้ภาษา และการใช้เทคโนโลยีสารสนเทศ & 1.จัดให้มีการเรียนการสอนที่ฝึกทักษะการนำเสนอผลงานโดยใช้ทั้งภาษาไทยและภาษาอังกฤษ \\
 & 2.ฝึกทักษะการนำเสนอผลงานและการเขียนรายงานทางวิชาการอย่างต่อเนื่องผ่านวิชาเรียนและการทำวิทยานิพนธ์ \\
 & 3.มอบหมายให้มีการสืบค้นข้อมูลโดยใช้เทคโนโลยีสารสนเทศ สังเคราะห์ วิเคราะห์ข้อมูลเพื่อใช้ในงานวิจัยและนำเสนอได้อย่างเหมาะสม \\
\hline
 
\end{tabular}

 \newpage
 \begin{landscape}
 \section{การพัฒนาผลลัพธ์การเรียนรู้}
 \subsection{การพัฒนาผลลัพธ์การเรียนรู้ตามมาตรฐานคุณวุฒิระดับอุดมศึกษา พ.ศ. 2565 ในแต่ละด้าน}
 
 \par\noindent\bigskip
 \renewcommand{\arraystretch}{1.3}
 \begin{tabular}{|p{0.3\linewidth}|p{0.3\linewidth}|p{0.3\linewidth}|}
\hline
\textbf{ผลลัพธ์การเรียนรู้ด้านความรู้} & \textbf{กลยุทธ์การสอนที่ใช้พัฒนาการเรียนรู้ด้านความรู้} & \textbf{กลยุทธ์การวัดผลและการประเมินผลการเรียนรู้ด้านความรู้} \\
\hline 
1. มีความเข้าใจในหลักการทางด้านการเรียนรู้ของเครื่อง(Machine Learning)และ การเรียนรู้เชิงลึก (Deep Learning)และทฤษฎีทางคณิตศาสตร์ ที่เกี่ยวข้องกับวิทยาการเชิงคำนวณ & 1.บรรยายหลักการทางด้านการเรียนรู้ของเครื่อง(Machine Learning)และ การเรียนรู้เชิงลึก (Deep Learning)และทฤษฎีทางคณิตศาสตร์ ที่เกี่ยวข้องกับวิทยาการเชิงคำนวณ & 1.ทดสอบโดยการสอบเขียนกลางภาคและปลายภาค \\ 
2. สามารถวิเคราะห์ปัญหา เข้าใจ รวมทั้งประยุกต์ความรู้ และทักษะที่เหมาะสมกับการแก้ไขปัญหา & 2.ยกตัวอย่างการประยุกต์ใช้หลักการและทฤษฎีทางวิทยาการเชิงคำนวณในสถานการณ์จริง & 2.ประเมินผลจากการทำงานที่ได้รับมอบหมายและรายงานที่ให้ค้นคว้า \\ 
3. สามารถออกแบบงานวิจัยที่เชื่อมโยงองค์ความรู้ต่าง ๆ เพื่อแก้ปัญหา และสร้างองค์ความรู้ใหม่ได้อย่างเหมาะสม & 3.การเปิดโอกาสให้ผู้เรียนได้มีการแสดงความคิดเห็นทางวิชาการและซักถามข้อสงสัย & 3.ประเมินจากพฤติกรรมการเรียนรู้ และการเข้าร่วมกิจกรรมการเรียนการสอนทั้งในและนอกสถานที่ \\ 
& 4. มอบหมายงานที่เกี่ยวข้องให้ค้นคว้า & 4.การสอบวัดคุณสมบัติ \\ 
& 5. การสรุปและสังเคราะห์องค์ความรู้ใหม่จากกรณีศึกษาและงานวิจัยของนักศึกษา & 5.การสอบหัวข้อวิทยานิพนธ์\\  
& 6. อภิปรายการออกแบบงานวิจัยที่เชื่อมโยงองค์ความรู้ในด้านต่าง ๆ จากกรณีศึกษา & 6.ความก้าวหน้าวิทยานิพนธ์ \\  
& 7. มีการจัดทำร่างผลงานวิจัยเพื่อขอรับการตีพิมพ์เผยแพร่ & 7.การสอบป้องกันวิทยานิพนธ์ \\  
& & 8. ประเมินต้นฉบับผลงานตีพิมพ์ และประเมินจากคุณภาพของผลงานตีพิมพ์ในฐานข้อมูลตามที่มหาวิทยาลัยกำหนด \\ \hline
 \end{tabular}
 
 \par\noindent\bigskip
 \renewcommand{\arraystretch}{1.3}
 \begin{tabular}{|p{0.3\linewidth}|p{0.3\linewidth}|p{0.3\linewidth}|}
\hline
\textbf{ผลลัพธ์การเรียนรู้ด้านทักษะ} & \textbf{กลยุทธ์การสอนที่ใช้พัฒนาการเรียนรู้ด้านทักษะ} & \textbf{กลยุทธ์การวัดผลและการประเมินผลการเรียนรู้ด้านทักษะ} \\
\hline 
1. สามารถนำองค์ความรู้ทางวิชาชีพ มาพัฒนาแนวคิด ริเริ่มและสร้างสรรค์ เพื่อตอบสนองประเด็นหรือปัญหาทางวิชาการและวิชาชีพได้ & 1.บรรยาย หรือยกตัวอย่างเกี่ยวกับการนำความรู้ทางวิชาชีพมาพัฒนาและต่อยอดเพื่อตอบสนองประเด็นหรือปัญหาทางวิชาการและวิชาชีพได้ & 1.ประเมินจากงานที่มอบหมาย \\ \hline
2. สามารถใช้เทคโนโลยีดิจิทัลเพื่อการสืบค้นข้อมูล นำเสนอ เขียนรายงานการวิจัย และตีพิมพ์ผลงานวิจัยได้อย่างเหมาะสม & 2.ยกตัวอย่างและอภิปราย การประยุกต์ใช้ความรู้ทางวิชาชีพเพื่อแก้ปัญหาในสถานการณ์จริง & 2.ประเมินจากพฤติกรรมการปฏิบัติของนักศึกษาระหว่างการเรียนการสอน \\ \hline
 & 3. อธิบายและแนะนำการเลือกใช้เทคโนโลยีดิจิทัลเพื่อการสืบค้นข้อมูล นำเสนอ เขียนรายงานการวิจัย และตีพิมพ์ผลงานวิจัยได้ & 3.ประเมินจากการสอบหัวข้อและเค้าโครงวิทยานิพนธ์ ความก้าวหน้าวิทยานิพนธ์ และการสอบป้องกันวิทยานิพนธ์ \\ \hline
 & 4. สาธิตการใช้เทคโนโลยีดิจิทัลที่เกี่ยวข้องกับการทำงานวิจัย ยกตัวอย่างกรณีศึกษา & 4.ประเมินจากงานที่ได้รับมอบหมาย กรณีศึกษา และงานที่เกี่ยวข้องกับวิทยานิพนธ์ของนักศึกษา \\ \hline
 & 5. แนะนำการประยุกต์ใช้เทคโนโลยีดิจิทัลสมัยใหม่ที่เกี่ยวข้องกับการทำงานวิจัย &  \\ \hline
 \end{tabular}
 
 
 \par\noindent\bigskip
 \renewcommand{\arraystretch}{1.3}
 \begin{tabular}{|p{0.3\linewidth}|p{0.3\linewidth}|p{0.3\linewidth}|}
\hline
\textbf{ผลลัพธ์การเรียนรู้ด้านจริยธรรม} & \textbf{กลยุทธ์การสอนที่ใช้พัฒนาการเรียนรู้ด้านจริยธรรม} & \textbf{กลยุทธ์การวัดผลและการประเมินผลการเรียนรู้ด้านจริยธรรม} \\
\hline 
1. ตระหนักและให้ความสำคัญในจรรยาบรรณทางวิชาการและวิชาชีพ & 1.สอดแทรกความสำคัญเรื่องจรรยาบรรณทางวิชาการและวิชาชีพในระหว่างการจัดการเรียนการสอน & 1.ประเมินจากพฤติกรรมของนักศึกษาระหว่างร่วมกิจกรรมการเรียนการสอน เช่น การส่งเสริมให้ผู้อื่นเห็นความสำคัญของจรรยาบรรณทางวิชาการและวิชาชีพ และร่วมการอภิปรายชั้นเรียน การไม่คัดลอกผลงานทางวิชาการ การปฏิบัติตามมาตรฐานและข้อกำหนดที่เกี่ยวข้องกับการทำวิทยานิพนธ์ \\ \hline
2. ปฏิบัติตามจริยธรรม มีระเบียบวินัย มีความซื่อสัตย์ รักษาความลับและผลประโยชน์ขององค์กร และมีจิตสาธารณะ & 2.บรรยาย หรือยกตัวอย่างเกี่ยวกับความสำคัญของจริยธรรม ระเบียบวินัย ความซื่อสัตย์ การรักษาความลับและผลประโยชน์ขององค์กร และจิตสาธารณะระหว่างการเรียนการสอน & 2.ประเมินจากพฤติกรรมการปฏิบัติระหว่างการเรียนการสอน \\ \hline
 & 3. นำเสนอและแลกเปลี่ยนความคิดเห็นเกี่ยวกับความสำคัญของการมีระเบียบวินัย มีความซื่อสัตย์ รักษาความลับและผลประโยชน์ขององค์กร และมีจิตสาธารณะ & 3.ประเมินจากการแสดงออกถึงความมีระเบียบวินัย ความซื่อสัตย์ และจิตสาธารณะระหว่างการเรียนการสอน \\ \hline
 &  & 4. ประเมินจากงานที่มอบหมาย \\ \hline
 \end{tabular}
 
 \par\noindent\bigskip
 \renewcommand{\arraystretch}{1.3}
 \begin{tabular}{|p{0.3\linewidth}|p{0.3\linewidth}|p{0.3\linewidth}|}
\hline
\textbf{ผลลัพธ์การเรียนรู้ด้านลักษณะบุคคล} & \textbf{กลยุทธ์การสอนที่ใช้พัฒนาการเรียนรู้ด้านลักษณะบุคคล} & \textbf{กลยุทธ์การวัดผลและการประเมินผลการเรียนรู้ด้านลักษณะบุคคล} \\
\hline 
1. สามารถสื่อสารและนำเสนอเชิงวิชาการด้วยภาษาไทยและภาษาอังกฤษ และมีบุคลิกภาพที่เหมาะสม & 1.การนำเสนอด้วยวาจาด้วยภาษาไทยและภาษาอังกฤษ & 1.ประเมินจากการนำเสนอเชิงวิชาการด้วยภาษาไทยและภาษาอังกฤษ \\ \hline
2. สามารถสื่อสารเพื่อทำงานร่วมกับผู้อื่น ปรับตัวเข้ากับสถานการณ์และวัฒนธรรมองค์กรได้ & 2.การนำเสนอในชั้นเรียนและกรณีศึกษา & 2.ประเมินจากบุคลิกภาพในการสื่อสาร \\ \hline
 & 3.การจัดการเรียนการสอนที่เน้นการสื่อสารระหว่างการทำงานร่วมกับผู้อื่น ทั้งการพูด การฟัง และการเขียน & 3.ประเมินจากกิจกรรมการเรียนการสอนที่จัดในห้องเรียน เช่น การสังเกตพฤติกรรม การสื่อสารระหว่างการเรียนและการทำงานร่วมกับผู้อื่น \\ \hline
 \end{tabular} 
 
 
 \newpage
\subsection{การพัฒนาผลลัพธ์การเรียนรู้ที่คาดหวังของหลักสูตร (PLOs)}
 
 \par\noindent\bigskip
 \renewcommand{\arraystretch}{1.3}
 \begin{longtable}{|p{0.3\linewidth}|p{0.3\linewidth}|p{0.3\linewidth}|}
 \hline
 \textbf{PLO 1} & \textbf{กลยุทธ์การสอนที่ใช้พัฒนาการเรียนรู้} & \textbf{กลยุทธ์การวัดผลและการประเมินผลการเรียนรู้} \\ 
 \hline
 อธิบายแนวคิด ทฤษฎี และหลักการ ของการเรียนรู้ของเครื่องได้ & 1.บรรยายหลักการทางด้านการเรียนรู้ของเครื่อง(Machine Learning)และ การเรียนรู้เชิงลึก(Deep Learning)& 1.ทดสอบโดยการสอบข้อเขียนกลางภาคและปลายภาค \\ \hline
& 2.ยกตัวอย่างการประยุกต์ใช้หลักการทางด้านการเรียนรู้ของเครื่อง(Machine Learning)และการเรียนรู้เชิงลึก (Deep Learning)& 2.ประเมินผลจากการทำงานท่ี่ได้รับมอบหมายและรายงานที่ให้ค้นคว้า \\ \hline
& 3. การเปิดโอกาสให้ผู้เรียนได้มีการแสดงความคิดเห็นและซักถามข้อสงสัย& 3. ประเมินจากพฤติกรรมการเรียนรู้ และการเข้าร่วมกิจกรรมการเรียนการสอนทั้งในและนอกสถานที่ \\ \hline
&&4. ประเมินจากการสอบวัดคุณสมบัติ \\ \hline
 \textbf{PLO 2} & \textbf{กลยุทธ์การสอนที่ใช้พัฒนาการเรียนรู้} & \textbf{กลยุทธ์การวัดผลและการประเมินผลการเรียนรู้} \\ 
 \hline
 อธิบายแนวคิด ทฤษฎี และหลักการ ของกระบวนการคณิตศาสตร์เชิงคำนวณได้ & 1.บรรยายหลักการทางด้านคณิตศาสตร์เชิงคำนวณ& 1.ทดสอบโดยการสอบข้อเขียนกลางภาคและปลายภาค \\ \hline
 & 2.ยกตัวอย่างการประยุกต์ใช้หลักการทางด้านคณิตศาสตร์เชิงคำนวณ& 2.ประเมินผลจากการทำงานท่ี่ได้รับมอบหมายและรายงานที่ให้ค้นคว้า \\ \hline
& 3. การเปิดโอกาสให้ผู้เรียนได้มีการแสดงความคิดเห็นและซักถามข้อสงสัย& 3. ประเมินจากพฤติกรรมการเรียนรู้ และการเข้าร่วมกิจกรรมการเรียนการสอนทั้งในและนอกสถานที่ \\ \hline
&&4. ประเมินจากการสอบวัดคุณสมบัติ \\ \hline
 \textbf{PLO 3} & \textbf{กลยุทธ์การสอนที่ใช้พัฒนาการเรียนรู้} & \textbf{กลยุทธ์การวัดผลและการประเมินผลการเรียนรู้} \\ 
 \hline
 ออกแบบการวิจัยได้ถูกต้องตามกระบวนการการทำวิจัยทางวิทยาศาสตร์ & 1. ยกตัวอย่างการประยุกต์ใช้ตัวแบบสำหรับวิเคราะห์ ข้อมูลในสถานการณ์ต่าง ๆ & 1. ทดสอบโดยการสอบข้อเขียนกลางภาคและปลายภาค \\
 \hline
 & 2. การเปิดโอกาสให้ผู้เรียนได้มีการแสดงความคิดเห็น วิพากษ์และซักถามข้อสงสัย & 2. ประเมินผลจากการทำงานที่ได้รับมอบหมายและรายงานที่ให้ค้นคว้า \\
 \hline
 & 3. มอบหมายงานค้นคว้าที่เกี่ยวข้องกับการพัฒนา/สร้างตัวแบบ & 3. ประเมินจากพฤติกรรมการเรียนรู้ และการเข้าร่วมกิจกรรมการเรียนการสอนทั้งในและนอกสถานที่ \\
 \hline
 && 4. ประเมินจากการสอบวัดคุณสมบัติ \\
 \hline
 \textbf{PLO 4} & \textbf{กลยุทธ์การสอนที่ใช้พัฒนาการเรียนรู้} & \textbf{กลยุทธ์การวัดผลและการประเมินผลการเรียนรู้} \\ 
 \hline
 สร้างหรือพัฒนางานวิจัยด้านการเรียนรู้ของเครื่องหรือการคำนวณเชิงคณิตศาสตร์ที่เกี่ยวข้อง (สายวิชาการ) / ประมวลงานวิจัยด้านการเรียนรู้ของเครื่อง หรือการคำนวณเชิงคณิตศาสตร์ที่เกี่ยวข้อง (สายวิชาชีพ) & 1. ยกตัวอย่างการประยุกต์ใช้ตัวแบบสำหรับวิเคราะห์ ข้อมูลในสถานการณ์ต่าง ๆ & 1. ทดสอบโดยการสอบข้อเขียนกลางภาคและปลายภาค \\
 \hline
 & 2. การเปิดโอกาสให้ผู้เรียนได้มีการแสดงความคิดเห็น วิพากษ์และซักถามข้อสงสัย & 2. ประเมินผลจากการทำงานที่ได้รับมอบหมายและรายงานที่ให้ค้นคว้า \\
 \hline
 & 3. มอบหมายงานค้นคว้าที่เกี่ยวข้องกับการพัฒนา/สร้างตัวแบบ & 3. ประเมินจากพฤติกรรมการเรียนรู้ และการเข้าร่วมกิจกรรมการเรียนการสอนทั้งในและนอกสถานที่ \\
 \hline
 && 4. ประเมินจากการสอบวัดคุณสมบัติ \\
 \hline
  \textbf{PLO 5} & \textbf{กลยุทธ์การสอนที่ใช้พัฒนาการเรียนรู้} & \textbf{กลยุทธ์การวัดผลและการประเมินผลการเรียนรู้} \\ 
 \hline
 ประยุกต์ความรู้ด้านการเรียนรู้ของเครื่อง หรือการคำนวณเชิงคณิตศาสตร์ที่เกี่ยวข้อง ในแก้ปัญหาจริงได้อย่างเหมาะสม & 1.จัดกิจกรรมการนำเสนอเชิงวิชาการด้วยภาษาไทยและภาษาอังกฤษ & 1.ประเมินจากกิจกรรมการสื่อสารและการนำเสนอเชิงวิชาการด้วยภาษาไทยและภาษาอังกฤษ \\
 \hline
 & 2.ฝึกบุคลิกภาพในการสื่อสาร & 2.ประเมินจากบุคลิกภาพในการสื่อสาร \\
 \hline
 & 3.จัดกิจกรรมการเรียนการสอนที่เน้นการสื่อสารระหว่างการทำงานร่วมกับผู้อื่น ทั้งการพูด การฟัง และการเขียน & 3.ประเมินจากกิจกรรมการเรียนการสอนที่จัดในห้องเรียน เช่น การสังเกตพฤติกรรม การสื่อสารระหว่างการเรียน \\
\hline
 \textbf{PLO 6} & \textbf{กลยุทธ์การสอนที่ใช้พัฒนาการเรียนรู้} & \textbf{กลยุทธ์การวัดผลและการประเมินผลการเรียนรู้} \\ 
 \hline
 ใช้เทคโนโลยีสารสนเทศในการรวบรวม วิเคราะห์ สังเคราะห์ และนำเสนอข้อมูลได้ & 1. อธิบายและแนะนำการเลือกใช้เทคโนโลยีสารสนเทศเพื่อการรวบรวม วิเคราะห์ สังเคราะห์ และนำเสนอข้อมูลได้ & 1. ประเมินผลจากการทำงานที่ได้รับมอบหมายและรายงานที่ให้ค้นคว้า \\
 \hline
 & 2. สาธิตการใช้เทคโนโลยีสารสนเทศที่เกี่ยวข้องกับการทำงานวิจัย ยกตัวอย่างกรณีศึกษา & 2. ประเมินจากพฤติกรรมการเรียนรู้ และการเข้าร่วมกิจกรรมการเรียนการสอน \\
 \hline
 & 3. แนะนำการประยุกต์ใช้เทคโนโลยีสารสนเทศสมัยใหม่ที่เกี่ยวข้องกับการทำงานวิจัย & 3.ประเมินจากการสอบวัดคุณสมบัติ\\
 \hline
 & & 4.ประเมินจากการสอบหัวข้อวิทยานิพนธ์\\
 \hline
 & & 5.ประเมินจากความก้าวหน้าของวิทยานิพนธ์\\
 \hline
 & & 6.ประเมินจากการสอบป้องกันวิทยานิพนธ์\\
 \hline
 & & 7.ประเมินต้นฉบับผลงานตีพิมพ์และประเมินจากคุณภาพของผลงานตีพิมพ์เผยแพร่\\
 \hline
 \textbf{PLO 7} & \textbf{กลยุทธ์การสอนที่ใช้พัฒนาการเรียนรู้} & \textbf{กลยุทธ์การวัดผลและการประเมินผลการเรียนรู้} \\ 
 \hline
 ใช้เทคโนโลยี และอัลกอริทึม สำหรับการเรียนรู้ของเครื่องในการสร้างสรรค์หรือพัฒนานวัตกรรม & 1.ฝึกปฏิบัติใช้เครื่องมือเกี่ยวกับการเรียนรู้ของเครื่องในการประมวลผลข้อมูล จากข้อมูลจริง และกรณีศึกษา & 1.ประเมินจากการสอบข้อเขียนและสอบปฏิบัติกลางภาคและปลายภาค \\
 \hline
  & 2.การเปิดโอกาสให้ผู้เรียนได้มีการแสดงความคิดเห็น วิพากษ์และซักถามข้อสงสัย & 2.ประเมินจากพฤติกรรมการเรียนรู้ และการเข้าร่วมกิจกรรมการเรียนการสอน \\
 \hline
 & & 3.ประเมินจากการทำงานที่ได้รับมอบหมายและรายงานที่ให้ค้นคว้า \\
 \hline
 & & 4.ประเมินจากการสอบวัดคุณสมบัติ \\
 \hline
 \textbf{PLO 8} & \textbf{กลยุทธ์การสอนที่ใช้พัฒนาการเรียนรู้} & \textbf{กลยุทธ์การวัดผลและการประเมินผลการเรียนรู้} \\ 
 \hline
 สื่อสาร และนำเสนอ องค์ความรู้จากการศึกษาด้านการเรียนรู้ของเครื่องหรือการคำนวณเชิงคณิตศาสตร์ที่เกี่ยวข้องได้ & 1.จัดกิจกรรมการนำเสนอเชิงวิชาการด้วยภาษาไทยและภาษาอังกฤษ & 1.ประเมินจากกิจกรรมการสื่อสารและการนำเสนอเชิงวิชาการด้วยภาษาไทยและภาษาอังกฤษ \\
 \hline
 & 2.ฝึกบุคลิกภาพในการสื่อสาร & 2.ประเมินจากบุคลิกภาพในการสื่อสาร \\
 \hline
 & 3.จัดกิจกรรมการเรียนการสอนที่เน้นการสื่อสารระหว่างการทำงานร่วมกับผู้อื่น ทั้งการพูด การฟัง และการเขียน & 3.ประเมินจากกิจกรรมการเรียนการสอนที่จัดในห้องเรียน เช่น การสังเกตพฤติกรรม การสื่อสารระหว่างการเรียน \\ \hline

 \textbf{PLO 9} & \textbf{กลยุทธ์การสอนที่ใช้พัฒนาการเรียนรู้} & \textbf{กลยุทธ์การวัดผลและการประเมินผลการเรียนรู้} \\ 
 \hline
 ผลิตผลงานทางวิชาการ หรือพัฒนานวัตกรรม ด้วยจริยธรรมทางวิชาการ & 1.สอดแทรกความสำคัญเรื่องจรรยาบรรณทางวิชาการและวิชาชีพในระหว่างการจัดการเรียนการสอน & 1.ประเมินจากพฤติกรรมของนักศึกษาระหว่างร่วมกิจกรรมการเรียนการสอน เช่น การส่งเสริมให้ผู้อื่นเห็นความสำคัญของจรรยาบรรณทางวิชาการและวิชาชีพ และร่วมการอภิปรายในชั้นเรียน \\
 \hline
 & 2.ยกตัวอย่างและอภิปรายจากสถานการณ์จริง กรณีศึกษา และผลกระทบที่เกิดขึ้นทั้งในเชิงบวกและเชิงลบ & 2.การไม่คัดลอกผลงานทางวิชาการ การปฏิบัติตามมาตรฐานและข้อกำหนดที่เกี่ยวข้องกับการทำวิทยานิพนธ์ \\
 \hline

 \textbf{PLO 10} & \textbf{กลยุทธ์การสอนที่ใช้พัฒนาการเรียนรู้} & \textbf{กลยุทธ์การวัดผลและการประเมินผลการเรียนรู้} \\ 
 \hline
 รู้จักบทบาทหน้าที่และมีความรับผิดชอบ & 1.นำเสนอและแลกเปลี่ยนความคิดเห็นเกี่ยวกับความสำคัญของการมีระเบียบวินัย มีความซื่อสัตย์ รักษาความลับและผลประโยชน์ขององค์กร และมีจิตสาธารณะ & 1.ประเมินจากพฤติกรรมการปฏิบัติระหว่างการเรียนการสอน \\
 \hline
 && 2.ประเมินจากการแสดงออกถึงระเบียบวินัย ความซื่อสัตย์ และจิตสาธารณะระหว่างการเรียนการสอน \\
 \hline
 && 3.ประเมินจากงานที่ได้รับมอบหมาย \\
 \hline
 \end{longtable}
 \end{landscape}
 
 \section{แผนที่แสดงการกระจายความรับผิดชอบมาตรฐานผลการเรียนรู้จากหลักสูตรสู่รายวิชา (Curriculum Mapping)}
 ผลการเรียนรู้ในตารางมีความหมายดังนี้
\subsection{ด้านความรู้}
\begin{enumerate}
	\item อธิบายแนวคิด ทฤษฎี และหลักการ ของการเรียนรู้ของเครื่องได้
	\item อธิบายแนวคิด ทฤษฎี และหลักการ ของกระบวนการคณิตศาสตร์เชิงคำนวณได้
	\item ออกแบบการวิจัยได้ถูกต้องตามกระบวนการการทำวิจัยทางวิทยาศาสตร์
	\item สร้างหรือพัฒนางานวิจัยด้านการเรียนรู้ของเครื่องหรือการคำนวณเชิงคณิตศาสตร์ที่เกี่ยวข้อง (สายวิชาการ) / ประมวลงานวิจัยด้านการเรียนรู้ของเครื่อง หรือการคำนวณเชิงคณิตศาสตร์ที่เกี่ยวข้อง (สายวิชาชีพ)
	\item ประยุกต์ความรู้ด้านการเรียนรู้ของเครื่อง หรือการคำนวณเชิงคณิตศาสตร์ที่เกี่ยวข้อง ในแก้ปัญหาจริงได้อย่างเหมาะสม
\end{enumerate}
\subsection{ด้านทักษะ}
\begin{enumerate}
	\item ใช้เทคโนโลยีสารสนเทศในการรวบรวม วิเคราะห์ สังเคราะห์ และนำเสนอข้อมูลได้
	\item ใช้เทคโนโลยี และอัลกอริทึม สำหรับการเรียนรู้ของเครื่องในการสร้างสรรค์หรือพัฒนานวัตกรรม
	\item สื่อสาร และนำเสนอ องค์ความรู้จากการศึกษาด้านการเรียนรู้ของเครื่องหรือการคำนวณเชิงคณิตศาสตร์ที่เกี่ยวข้องได้
\end{enumerate}
\subsection{ด้านจริยธรรม}
\begin{enumerate}
	\item ผลิตผลงานทางวิชาการ หรือพัฒนานวัตกรรม ด้วยจริยธรรมทางวิชาการ
\end{enumerate}
\subsection{ด้านบุคคล}
\begin{enumerate}
	\item รู้จักบทบาทหน้าที่และมีความรับผิดชอบ
\end{enumerate}

\newpage
\begin{landscape}
\begin{center}

\textbf{\large แผนที่แสดงการกระจายความรับผิดชอบมาตรฐานผลลัพธ์การเรียนรู้จากหลักสูตรสู่รายวิชา (Curriculum Mapping)}
\par\bigskip
\begin{longtable}{|p{0.08\linewidth}|p{0.22\linewidth}|C{0.04\linewidth}|C{0.04\linewidth}|C{0.04\linewidth}|C{0.04\linewidth}|C{0.04\linewidth}|C{0.04\linewidth}|C{0.04\linewidth}|C{0.04\linewidth}|C{0.04\linewidth}|C{0.08\linewidth}|}
\hline
\multicolumn{2}{|C{0.32\linewidth}|}{ \multirow{2}{*}{\textbf{รายวิชา}}} &
\multicolumn{5}{C{0.28\linewidth}|}{\textbf{1. ความรู้}} &
\multicolumn{2}{C{0.08\linewidth}|}{\textbf{2. ทักษะ}} &
\multicolumn{2}{C{0.08\linewidth}|}{\textbf{3. จริยธรรม}} &
\textbf{4. บุคคล} \\ \cline{3-12}
\multicolumn{2}{|p{0.27\textwidth}|}{} &  1 & 2 & 3 & 4 & 5 & 1 & 2 & 1 & 2 &1 \\ \hline
\endhead09-110-601 & ระเบียบวิธีวิจัยการหาค่าเหมาะที่สุดเชิงคำนวณและการเรียนรู้ของเครื่อง & & & \ding{108}& & & & & \ding{108}& & \ding{108}\\ \hline
09-110-602 & สัมมนา & & & & & & & & \ding{108}& & \ding{108}\\ \hline
09-111-601 & สถิติและความน่าจะเป็นสำหรับการเรียนรู้ของเครื่อง & & \ding{108}& & & & & & & & \ding{108}\\ \hline
09-111-602 & คณิตศาสตร์สำหรับการเรียนรู้ของเครื่อง & \ding{108}& \ding{108}& & & & & & & & \ding{108}\\ \hline
09-111-603 & การเรียนรู้ของเครื่อง & \ding{108}& \ding{108}& & & & & & & & \ding{108}\\ \hline
09-111-704 & การหาค่าเหมาะที่สุดสำหรับการเรียนรู้ของเครื่อง & \ding{108}& & & & & \ding{108}& & & & \ding{108}\\ \hline
09-111-705 & โครงสร้างข้อมูลและอัลกอริทึมสำหรับการเรียนรู้ของเครื่อง & \ding{108}& & & & & \ding{108}& & & & \ding{108}\\ \hline
09-112-601 & การวิเคราะห์เชิงฟังก์ชัน & & \ding{108}& & & & & & & & \ding{108}\\ \hline
09-112-702 & ทฤษฎีจุดตรึงและการประยุกต์ & & \ding{108}& & & & & & & & \ding{108}\\ \hline
09-113-601 & คณิตศาสตร์ขั้นสูงสำหรับการเรียนรู้ของเครื่อง & & \ding{108}& & & & & & & & \ding{108}\\ \hline
09-113-702 & ขั้นตอนวิธีเชิงตัวเลขสำหรับค่าเหมาะที่สุด  & \ding{108}& \ding{108}& & & & \ding{108}& & & & \ding{108}\\ \hline
09-113-603 & การตัดสินใจอย่างชาญฉลาดด้วยกำหนดการวิจัยดำเนินงาน & \ding{108}& & & & & \ding{108}& & & & \ding{108}\\ \hline
09-113-704 & หัวข้อพิเศษของคณิตศาสตร์เชิงคำนวณ  & \ding{108}& \ding{108}& & & & \ding{108}& & & & \ding{108}\\ \hline
09-114-701 & การเรียนรู้เชิงลึกและการประยุกต์   & \ding{108}& & & & & \ding{108}& & & & \ding{108}\\ \hline
09-114-702 & วิศวกรรมการเรียนรู้ของเครื่องและการดึงข้อมูล & \ding{108}& & & & & \ding{108}& & & & \ding{108}\\ \hline
09-114-603 & การวิเคราะห์ข้อมูล & \ding{108}& & & & & \ding{108}& & & & \ding{108}\\ \hline
09-114-604 & การทำให้เห็นข้อมูล & \ding{108}& & & & & \ding{108}& & & & \ding{108}\\ \hline
09-114-705 & การประยุกต์ใช้การเรียนรู้ของเครื่องในงานด้านการประมวลภาษาธรรมชาติ & \ding{108}& \ding{108}& & & & \ding{108}& \ding{108}& & & \ding{108}\\ \hline
09-114-706 & การประยุกต์ใช้การเรียนรู้ของเครื่องในงานด้านประมวลผลภาพและสัญญาณ & \ding{108}& \ding{108}& & & & \ding{108}& \ding{108}& & & \ding{108}\\ \hline
09-114-707 & การประยุกต์ใช้การเรียนรู้ของเครื่องในงานด้านการแพทย์  & \ding{108}& \ding{108}& & & & \ding{108}& \ding{108}& & & \ding{108}\\ \hline
09-114-708 & การประยุกต์ใช้การเรียนรู้ของเครื่องในด้านธุรกิจและการเงิน & \ding{108}& \ding{108}& & & & \ding{108}& \ding{108}& & & \ding{108}\\ \hline
09-114-709 & หัวข้อพิเศษของการเรียนรู้ของเครื่อง  & \ding{108}& \ding{108}& & & & \ding{108}& & & & \ding{108}\\ \hline
09-115-701 & สารนิพนธ์ & \ding{108}& \ding{108}& \ding{108}& \ding{108}& \ding{108}& \ding{108}& \ding{108}& \ding{108}& \ding{108}& \ding{108}\\ \hline
09-115-702 & วิทยานิพนธ์ & \ding{108}& \ding{108}& \ding{108}& \ding{108}& \ding{108}& \ding{108}& \ding{108}& \ding{108}& \ding{108}& \ding{108}\\ \hline
09-115-703 & วิทยานิพนธ์ & \ding{108}& \ding{108}& \ding{108}& \ding{108}& \ding{108}& \ding{108}& \ding{108}& \ding{108}& \ding{108}& \ding{108}\\ \hline
\end{longtable}
	
\end{center}
\end{landscape}


\newpage
\begin{center}
\textbf{\large แผนที่แสดงการกระจายความรับผิดชอบมาตรฐานผลลัพธ์การเรียนรู้ระดับหลักสูตร (PLOs) \\ในตารางมีความหมายดังนี้}
\end{center}

\begin{enumerate}[label=PLO\arabic*., leftmargin=3\parindent]
	\item อธิบายแนวคิด ทฤษฎี และหลักการ ของการเรียนรู้ของเครื่องได้
	\item อธิบายแนวคิด ทฤษฎี และหลักการ ของกระบวนการคณิตศาสตร์เชิงคำนวณได้
	\item ออกแบบการวิจัยได้ถูกต้องตามกระบวนการการทำวิจัยทางวิทยาศาสตร์ 
	\item สร้างหรือพัฒนางานวิจัยด้านการเรียนรู้ของเครื่องหรือการคำนวณเชิงคณิตศาสตร์ที่เกี่ยวข้อง (สายวิชาการ)/ ประมวลงานวิจัยด้านการเรียนรู้ของเครื่อง หรือการคำนวณเชิงคณิตศาสตร์ที่เกี่ยวข้อง (สายวิชาชีพ) 
	\item ประยุกต์ความรู้ด้านการเรียนรู้ของเครื่อง หรือการคำนวณเชิงคณิตศาสตร์ที่เกี่ยวข้อง ในการแก้ปัญหาจริงได้อย่างเหมาะสม
	\item ใช้เทคโนโลยีสารสนเทศในการรวบรวม วิเคราะห์ สังเคราะห์ และนำเสนอข้อมูลได้
	\item ใช้เทคโนโลยี และอัลกอริทึม สำหรับการเรียนรู้ของเครื่องในการสร้างสรรค์หรือพัฒนานวัตกรรม
	\item สื่อสาร และนำเสนอ องค์ความรู้จากการศึกษาด้านการเรียนรู้ของเครื่องหรือการคำนวณเชิงคณิตศาสตร์ที่เกี่ยวข้องได้
	\item ผลิตผลงานทางวิชาการ หรือพัฒนานวัตกรรม ด้วยจริยธรรมทางวิชาการ
	\item รู้จักบทบาทหน้าที่และมีความรับผิดชอบ
\end{enumerate}


\newpage
\begin{landscape}
\begin{center}
\textbf{\large แผนที่แสดงการกระจายความรับผิดชอบมาตรฐานผลลัพธ์การเรียนรู้ระดับหลักสูตรสู่รายวิชา (Curriculum Maopping) หมวดวิชาเฉพาะ}
\par\bigskip
\begin{longtable}{|p{0.08\linewidth}p{0.2\linewidth}|C{0.05\linewidth}|C{0.05\linewidth}|C{0.05\linewidth}|C{0.05\linewidth}|C{0.05\linewidth}|C{0.05\linewidth}|C{0.05\linewidth}|C{0.05\linewidth}|C{0.05\linewidth}|C{0.06\linewidth}|}
\hline
\multicolumn{2}{|C{0.28\linewidth}|}{ \textbf{ผลการเรียนรู้} }
 &  \textbf{PLO1} & \textbf{PLO2} & \textbf{PLO3} & \textbf{PLO4} & \textbf{PLO5} & \textbf{PLO6} & \textbf{PLO7} & \textbf{PLO8} & \textbf{PLO9} &\textbf{PLO10} \\ \hline
\endhead09-110-601 & ระเบียบวิธีวิจัย & & & \ding{108}& & & & & \ding{108}& & \ding{108}\\ \hline
09-110-602 & สัมมนา & & & & & & & & \ding{108}& & \ding{108}\\ \hline
09-111-601 & สถิติและความน่าจะเป็นสำหรับการเรียนรู้ของเครื่อง & & \ding{108}& & & & & & & & \ding{108}\\ \hline
09-111-602 & คณิตศาสตร์สำหรับการเรียนรู้ของเครื่อง & \ding{108}& \ding{108}& & & & & & & & \ding{108}\\ \hline
09-111-603 & การเรียนรู้ของเครื่องแบบมีผู้สอน & \ding{108}& \ding{108}& & \ding{108}& \ding{108}& \ding{108}& \ding{108}& \ding{108}& \ding{108}& \ding{108}\\ \hline
09-111-704 & การหาค่าเหมาะที่สุดสำหรับการเรียนรู้ของเครื่อง & \ding{108}& & & & & \ding{108}& & & & \ding{108}\\ \hline
09-111-705 & โครงสร้างข้อมูลและอัลกอริทึมสำหรับการเรียนรู้ของเครื่อง & \ding{108}& & & & & \ding{108}& & & & \ding{108}\\ \hline
09-112-601 & การวิเคราะห์เชิงฟังก์ชัน & & \ding{108}& & & & & & & & \ding{108}\\ \hline
09-112-702 & ทฤษฎีจุดตรึงและการประยุกต์ & & \ding{108}& & & & & & & & \ding{108}\\ \hline
09-113-601 & คณิตศาสตร์ขั้นสูงสำหรับการเรียนรู้ของเครื่อง & & \ding{108}& & & & & & & & \ding{108}\\ \hline
09-113-702 & ขั้นตอนวิธีเชิงตัวเลขสำหรับค่าเหมาะที่สุด  & \ding{108}& \ding{108}& & & & \ding{108}& & & & \ding{108}\\ \hline
09-113-603 & การตัดสินใจอย่างชาญฉลาดด้วยกำหนดการวิจัยดำเนินงาน & \ding{108}& & & & & \ding{108}& & & & \ding{108}\\ \hline
09-113-704 & หัวข้อพิเศษของคณิตศาสตร์เชิงคำนวณ  & \ding{108}& \ding{108}& & & & \ding{108}& & & & \ding{108}\\ \hline
09-114-601 & การเรียนรู้ของเครื่องแบบไม่มีผู้สอน & \ding{108}& \ding{108}& & \ding{108}& \ding{108}& \ding{108}& \ding{108}& \ding{108}& \ding{108}& \ding{108}\\ \hline
09-114-702 & การเรียนรู้ของเครื่องแบบเสริมแรง & \ding{108}& \ding{108}& & \ding{108}& \ding{108}& \ding{108}& \ding{108}& \ding{108}& \ding{108}& \ding{108}\\ \hline
09-114-703 & การเรียนรู้เชิงลึกและการประยุกต์   & \ding{108}& & & & & \ding{108}& & & & \ding{108}\\ \hline
09-114-704 & วิศวกรรมการเรียนรู้ของเครื่อง & \ding{108}& & & & & \ding{108}& & & & \ding{108}\\ \hline
09-114-605 & การวิเคราะห์ข้อมูล & \ding{108}& & & & & \ding{108}& & & & \ding{108}\\ \hline
09-114-606 & การทำให้เห็นข้อมูล & \ding{108}& & & & & \ding{108}& & & & \ding{108}\\ \hline
09-114-707 & แบบจำลองภาษาขนาดใหญ่ & \ding{108}& \ding{108}& & & & \ding{108}& \ding{108}& & & \ding{108}\\ \hline
09-114-708 & การประยุกต์ใช้การเรียนรู้ของเครื่องในงานด้านประมวลผลภาพและสัญญาณ & \ding{108}& \ding{108}& & & & \ding{108}& \ding{108}& & & \ding{108}\\ \hline
09-114-709 & การประยุกต์ใช้การเรียนรู้ของเครื่องในงานด้านการแพทย์  & \ding{108}& \ding{108}& & & & \ding{108}& \ding{108}& & & \ding{108}\\ \hline
09-114-710 & การประยุกต์ใช้การเรียนรู้ของเครื่องในด้านธุรกิจและการเงิน & \ding{108}& \ding{108}& & & & \ding{108}& \ding{108}& & & \ding{108}\\ \hline
09-114-711 & หัวข้อพิเศษของการเรียนรู้ของเครื่อง  & \ding{108}& \ding{108}& & & & \ding{108}& & & & \ding{108}\\ \hline
09-115-701 & สารนิพนธ์ & \ding{108}& \ding{108}& \ding{108}& \ding{108}& \ding{108}& \ding{108}& \ding{108}& \ding{108}& \ding{108}& \ding{108}\\ \hline
09-115-702 & วิทยานิพนธ์ & \ding{108}& \ding{108}& \ding{108}& \ding{108}& \ding{108}& \ding{108}& \ding{108}& \ding{108}& \ding{108}& \ding{108}\\ \hline
09-115-703 & วิทยานิพนธ์ & \ding{108}& \ding{108}& \ding{108}& \ding{108}& \ding{108}& \ding{108}& \ding{108}& \ding{108}& \ding{108}& \ding{108}\\ \hline
\end{longtable}
	
\end{center}	
\end{landscape}




 
 
 
 
 
 
 
 
 
 