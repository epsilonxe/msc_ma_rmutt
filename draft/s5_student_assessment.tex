\chapter{หลักเกณฑ์ในการประเมินผลนักศึกษา}

\section{กฎระเบียบหรืิอหลักเกณฑ์ในการให้ระดับคะแนน (เกรด)}

ประกาศ\university \, เรื่อง เกณฑ์การวัดผลและประเมินผลการศึกษาระดับบัณฑิตศึกษา พ.ศ. 2559 (ภาคผนวก)

\section{กระบวนการทวนสอบมาตรฐานผลสัมฤทธิ์ของนักศึกษา}

\subsection{การทวนสอบมาตรฐานผลการเรียนรู้ขณะนักศึกษายังไม่สำเร็จการศึกษา}
   \begin{enumerate}
   	\item อาจารย์ประจำหลักสูตรและอาจารย์ผู้สอนเป็นกรรมการพิจารณาความเหมาะสมของข้อสอบ 
   	\item แต่งตั้งคณะกรรมการของสาขาวิชา ทวนสอบผลการประเมินทุกรายวิชา
   	\item นักศึกษากรอกแบบประเมินการสอนของอาจารย์ผู้สอน
   	 \end{enumerate}    
  
\subsection{การทวนสอบมาตรฐานผลการเรียนรู้หลังนักศึกษาสำเร็จการศึกษา}
การกำหนดวิธีการทวนสอบมาตรฐานผลการเรียนรู้ของนักศึกษา ควรเน้นการประเมินผลสัมฤทธิ์การประกอบอาชีพของมหาบัณฑิตอย่างต่อเนื่อง และนำผลการประเมินที่ได้ย้อนกลับมาพัฒนาและปรับปรุงกระบวนการเรียนการสอนและหลักสูตรการเรียนการสอน โดยดำเนินการดังนี้
\begin{enumerate}
   	\item ภาวะการได้งานทำของมหาบัณฑิต โดยประเมินจากมหาบัณฑิตในแต่ละรุ่นที่สำเร็จการศึกษา 
   	\item ตรวจสอบจากผู้ประกอบการ โดยการขอสัมภาษณ์หรือการจัดส่งแบบสอบถามไปยังสถานประกอบการ เพื่อประเมินความพึงพอใจในมหาบัณฑิตที่สำเร็จการศึกษาและเข้าทำงานในสถานประกอบการนั้น ๆ
   	\item การประเมินจากตำแหน่งและ/หรือความก้าวหน้าในสายงานของมหาบัณฑิต
   	\item ความคิดเห็นจากผู้ทรงคุณวุฒิภายนอกที่มาร่วมปรับปรุงหรือวิพากษ์หลักสูตร หรืออาจารย์พิเศษ ต่อความพร้อมของนักศึกษาในการเรียนและคุณสมบัติอื่น ๆ ที่เกี่ยวข้องกับกระบวนการเรียนรู้และการพัฒนาองค์ความรู้ของนักศึกษา
   	 \end{enumerate}  

\section{เกณฑ์การสำเร็จการศึกษาตามหลักสูตร}
\subsection{แผน ก แบบ ก 1 (ทำวิทยานิพนธ์อย่างเดียว)}
\begin{enumerate}
   	\item ศึกษารายวิชาในหมวดวิชาพื้นฐานครบตามที่กำหนดในหลักสูตร และมีค่าระดับคะแนนเฉลี่ยสะสมของรายวิชาตามหลักสูตรในหมวดนี้ไม่ต่ำกว่า $3.00$ จากระบบ $4.00$ ระดับคะแนน หรือเทียบเท่า
%   	\item  สอบผ่านการสอบประมวลความรู้ เมื่อสอบผ่านรายวิชาในหมวดวิชาพื้นฐานครบถ้วน โดยได้แต้มระดับคะแนนเฉลี่ยสะสมไม่ต่ำกว่า $3.00$ จากระบบ $4.00$ ระดับคะแนน หรือเทียบเท่า ด้วยข้อเขียนและ/หรือปากเปล่าในสาขาวิชา\thdegreebranch{} ซึ่งมีองค์ประกอบเป็นไปตามเกณฑ์มาตรฐานหลักสูตรระดับบัณฑิตศึกษา พ.ศ. 2565 (ภาคผนวก)
   	\item เสนอวิทยานิพนธ์ และสอบผ่านการสอบปากเปล่าขั้นสุดท้าย จนบรรลุผลลัพธ์การเรียนรู้ที่คาดหวังของหลักสูตร สำหรับสอบปากเปล่าให้ดำเนินการโดยคณะกรรมการสอบวิทยานิพนธ์ที่สถาบันอุดมศึกษาแต่งตั้ง ซึ่งมีองค์ประกอบเป็นไปตามเกณฑ์มาตรฐานหลักสูตรระดับบัณฑิตศึกษา พ.ศ. 2565 กำหนด โดยเป็นระบบเปิดให้ผู้สนใจเข้ารับฟังได้ สำหรับผลงานวิทยานิพนธ์ หรือส่วนหนึ่งของวิทยานิพนธ์ต้องได้รับการตีพิมพ์ หรืออย่างน้อยได้รับการยอมรับให้ตีพิมพ์ในวารสารระดับชาติหรือระดับนานาชาติที่มีคุณภาพตามประกาศคณะกรรมการการอุดมศึกษา เรื่องหลักเกณฑ์การพิจารณาวารสารทางวิชาการสำหรับการเผยแพร่ผลงานทางวิชาการ 
   		   หรือนำเสนอต่อที่ประชุมวิชาการ โดยบทความที่นำเสนอฉบับสมบูรณ์ (Full Paper) ได้รับการตีพิมพ์ในรายงานสืบเนื่องจากการประชุมวิชาการ (Proceedings) อย่างน้อย 1 เรื่อง ทั้งนี้ ข้อกำหนดอื่นใดจะต้องเป็นไปตามประกาศ \university{} เรื่อง การตีพิมพ์บทความวิจัยเพื่อสำเร็จการศึกษาระดับบัณฑิตศึกษา (ภาคผนวก)
   	\item สอบผ่านความรู้ภาษาต่างประเทศตามเงื่อนไขและหลักเกณฑ์ โดยให้เป็นไปตามประกาศ\university{} (ภาคผนวก)
   	\item เกณฑ์อื่นใดให้เป็นไปตามข้อบังคับ\university{} ว่าด้วยการศึกษาระดับบัณฑิต พ.ศ. 2559 และที่แก้ไขเพิ่มเติม (ภาคผนวก)   	 
   	\end{enumerate}
\subsection{แผน ก แบบ ก 2 (ศึกษารายวิชาและทำวิทยานิพนธ์)}
\begin{enumerate}
   	\item ศึกษารายวิชาครบตามที่กำหนดในหลักสูตร และมีค่าระดับคะแนนเฉลี่ยสะสมของรายวิชาตามหลักสูตรไม่ต่ำกว่า $3.00$ จากระบบ $4.00$ ระดับคะแนน หรือเทียบเท่า 
%   	\item สอบผ่านการสอบประมวลความรู้ ด้วยข้อเขียนและ/หรือปากเปล่าในสาขาวิชานั้น พร้อมทั้งเสนอรายงานการค้นคว้าอิสระและสอบผ่านการสอบปากเปล่าขั้นสุดท้าย จนบรรลุผลลัพธ์การเรียนรู้ที่คาดหวังของหลักสูตร สำหรับสอบปากเปล่าให้ดำเนินการโดยคณะกรรมการสอบวิทยานิพนธ์ที่สถาบันอุดมศึกษาแต่งตั้ง ซึ่งมีองค์ประกอบเป็นไปตามเกณฑ์มาตรฐานหลักสูตรระดับบัณฑิตศึกษา พ.ศ. 2565 กำหนด โดยเป็นระบบเปิดให้ผู่้สนใจเข้ารับฟังได้ (ภาคผนวก)
   	\item เสนอวิทยานิพนธ์ และสอบผ่านการสอบปากเปล่าขั้นสุดท้าย จนบรรลุผลลัพธ์การเรียนรู้ที่คาดหวังของหลักสูตร สำหรับสอบปากเปล่าให้ดำเนินการโดยคณะกรรมการสอบวิทยานิพนธ์ที่สถาบันอุดมศึกษาแต่งตั้ง ซึ่งมีองค์ประกอบเป็นไปตามเกณฑ์มาตรฐานหลักสูตรระดับบัณฑิตศึกษา พ.ศ. 2565 กำหนด โดยเป็นระบบเปิดให้ผู้สนใจเข้ารับฟังได้ สำหรับผลงานวิทยานิพนธ์ หรือส่วนหนึ่งของวิทยานิพนธ์ต้องได้รับการตีพิมพ์ หรืออย่างน้อยได้รับการยอมรับให้ตีพิมพ์ในวารสารระดับชาติหรือระดับนานาชาติที่มีคุณภาพตามประกาศคณะกรรมการการอุดมศึกษา เรื่องหลักเกณฑ์การพิจารณาวารสารทางวิชาการสำหรับการเผยแพร่ผลงานทางวิชาการ 
   		   หรือนำเสนอต่อที่ประชุมวิชาการ โดยบทความที่นำเสนอฉบับสมบูรณ์ (Full Paper) ได้รับการตีพิมพ์ในรายงานสืบเนื่องจากการประชุมวิชาการ (Proceedings) อย่างน้อย 1 เรื่อง ทั้งนี้ ข้อกำหนดอื่นใดจะต้องเป็นไปตามประกาศ \university{} เรื่อง การตีพิมพ์บทความวิจัยเพื่อสำเร็จการศึกษาระดับบัณฑิตศึกษา (ภาคผนวก)  	
   \item สอบผ่านความรู้ภาษาต่างประเทศตามเงื่อนไขและหลักเกณฑ์ โดยให้เป็นไปตามประกาศ\university{} (ภาคผนวก)
   	\item เกณฑ์อื่นใดให้เป็นไปตามข้อบังคับ\university{} ว่าด้วยการศึกษาระดับบัณฑิต พ.ศ. 2559 และที่แก้ไขเพิ่มเติม (ภาคผนวก)   
   	 \end{enumerate}
   	 
 \subsection{แผน ข }
\begin{enumerate}
   	\item ศึกษารายวิชาครบตามที่กำหนดในหลักสูตร และมีค่าระดับคะแนนเฉลี่ยสะสมของรายวิชาตามหลักสูตรไม่ต่ำกว่า $3.00$ จากระบบ $4.00$ ระดับคะแนน หรือเทียบเท่า
   	\item  สอบผ่านการสอบประมวลความรู้ เมื่อสอบผ่านรายวิชาครบตามที่กำหนดในหลักสูตร โดยได้แต้มระดับคะแนนเฉลี่ยสะสมไม่ต่ำกว่า $3.00$ จากระบบ $4.00$ ระดับคะแนน หรือเทียบเท่า ด้วยข้อเขียนและ/หรือปากเปล่าในสาขาวิชา\thdegreebranch{} ซึ่งมีองค์ประกอบเป็นไปตามเกณฑ์มาตรฐานหลักสูตรระดับบัณฑิตศึกษา พ.ศ. 2565 (ภาคผนวก)
   	\item เสนอสารนิพนธ์ และสอบผ่านการสอบปากเปล่าขั้นสุดท้าย จนบรรลุผลลัพธ์การเรียนรู้ที่คาดหวังของหลักสูตร สำหรับสอบปากเปล่าให้ดำเนินการโดยคณะกรรมการสอบสารนิพนธ์ที่สถาบันอุดมศึกษาแต่งตั้ง ซึ่งมีองค์ประกอบเป็นไปตามเกณฑ์มาตรฐานหลักสูตรระดับบัณฑิตศึกษา พ.ศ. 2565 กำหนด โดยเป็นระบบเปิดให้ผู้สนใจเข้ารับฟังได้
   	\item สอบผ่านความรู้ภาษาต่างประเทศตามเงื่อนไขและหลักเกณฑ์ โดยให้เป็นไปตามประกาศ\university{} (ภาคผนวก)
   	\item เกณฑ์อื่นใดให้เป็นไปตามข้อบังคับ\university{} ว่าด้วยการศึกษาระดับบัณฑิต พ.ศ. 2559 และที่แก้ไขเพิ่มเติม (ภาคผนวก)   	 
   	\end{enumerate}













