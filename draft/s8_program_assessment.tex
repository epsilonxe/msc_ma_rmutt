\chapter{การประเมิน และปรับปรุงการดำเนินการของหลักสูตร}

\section{การประเมินประสิทธิผลของการสอน}

\subsection{การประเมินกลยุทธ์การสอน}
\begin{enumerate}
   	\item ประเมินกลยุทธ์การสอนโดยแต่งตั้งคณะกรรมการสาขาวิชาสังเกตการสอนของอาจารย์หรืออาจารย์ผู้รับผิดชอบหลักสูตร  
   	\item ประเมินโดยการสัมภาษณ์นักศึกษาหรือการทำแบบประเมินในเรื่องการสอนของอาจารย์   
   	 \end{enumerate}  
\subsection{การประเมินทักษะของอาจารย์ในการใช้แผนกลยุทธ์การสอน}
\university ให้นักศึกษาได้ประเมินผลการสอนของอาจารย์ทั้งในด้านทักษะ กลยุทธ์การสอน และการใช้สื่อการเรียนการสอน ในทุกรายวิชา ทุกภาคการศึกษา โดยมีการประเมินผ่านเว็บไซต์ของ\university \,\, หรือประเมินโดยอาจารย์ผู้รับผิดชอบหลักสูตร

\section{การประเมินหลักสูตรในภาพรวม}
มีการประเมินหลักสูตรในภาพรวม โดยสำรวจข้อมูลจากผลความพึงพอใจต่อหลักสูตรจากนักศึกษาที่ใกล้สำเร็จการศึกษา มหาบัณฑิต ผู้ทรงคุณวุฒิภายนอก และสถานประกอบการต่าง ๆ
\section{การประเมินผลการดำเนินงานตามรายละเอียดหลักสูตร}
มีการประเมินผลการดำเนินงานตามหลักสูตร ตามดัชนีตัวบ่งชี้ผลการดำเนินงานที่ระบุในหมวดที่ 7 ข้อ 7 โดยการดำเนินการตามเกณฑ์มาตรฐานหลักสูตรระดับบัณฑิตศึกษา พ.ศ. 2565 และเกณฑ์การประกันคุณภาพการศึกษาระดับหลักสูตรตามแนวทาง AUN-QA

\section{การทบทวนผลประเมินและวางแผนปรับปรุงหลักสูตรและแผนกลยุทธ์การสอน}
คณะกรรมการประจำสาขาวิชาฯรวบรวมข้อมูลจากการประเมินผลการเรียนการสอนของนักศึกษาที่ใกล้สำเร็จการศึกษา มหาบัณฑิต ผู้ทรงคุณวุฒิภายนอก และสถานประกอบการต่าง ๆ และข้อมูลจาก มคอ.5 และ มคอ.7 เพื่อให้ทราบถึงปัญหาและข้อเสนอแยนะต่าง ๆ จากการดำเนินการหลักสูตรทั้งในภาพรวมและในแต่ละรายวิชา เพื่อนำไปสู่การปรับปรุงหลักสูตรทั้งที่เป็นการปรับปรุงเล็กน้อยและการปรับปรุงที่ครบรอบพัฒนา ซึ่งกระทำทุก ๆ 5 ปี










