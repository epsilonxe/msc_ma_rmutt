\chapter{การประกันคุณภาพหลักสูตร}
\section{การบริหารหลักสูตร}
\subsection{สาระของรายวิชาในหลักสูตร}
หลักสูตรวิทยาศาสตรมหาบัณฑิต สาขา\thdegreebranch (หลักสูตรใหม่ พ.ศ. 2568) มุ่งผลิตมหาบัณฑิตที่มีความรู้ด้านคณิตศาสตร์ สถิติ และคอมพิวเตอร์ เพื่อสร้างนวัตกรรมเชิงคำนวณที่ตอบโจทย์เศรษฐกิจและสังคมดิจิทัล และผลิตบัณฑิตที่มีศักยภาพในการคิด วิเคราะห์ และพัฒนานวัตกรรมอย่างยั่งยืน โดยหลักคิดในการออกแบบหลักสูตรนั้นได้พิจารณา
ให้สอดคล้องกับเป้าหมายการพัฒนาที่ยั่งยืนของสหประชาชาติ (SDGs) หลายข้อ โดยเฉพาะอย่างยิ่ง เป้าหมายที่ 9 อุตสาหกรรม นวัตกรรม และโครงสร้างพื้นฐาน, เป้าหมายที่ 8 การจ้างงานที่มีคุณค่าและการเติบโตทางเศรษฐกิจ, เป้าหมายที่ 11 เมืองและชุมชนที่ยั่งยืน, และเป้าหมายที่ 17 ความร่วมมือเพื่อการพัฒนาที่ยั่งยืน หลักสูตรมุ่งเน้นในการเสริมสร้างทักษะการคำนวณขั้นสูงให้กับนักศึกษา เพื่อส่งเสริมการนวัตกรรมและความก้าวหน้าทางเทคโนโลยี ซึ่งเป็นสิ่งจำเป็นสำหรับการสร้างโครงสร้างพื้นฐานที่ยืดหยุ่นและส่งเสริมการอุตสาหกรรมที่ครอบคลุมและยั่งยืน บัณฑิตที่สำเร็จการศึกษาจะพร้อมที่จะขับเคลื่อนการเติบโตทางเศรษฐกิจโดยการพัฒนากระบวนการแก้ปัญหาที่มีประสิทธิภาพต่อปัญหาที่ซับซ้อนในอุตสาหกรรมต่าง ๆ ซึ่งนำไปสู่การสร้างงานที่มีคุณค่าและการเพิ่มผลิตภาพ

ในบริบทของเมืองและชุมชนที่ยั่งยืน (เป้าหมายที่ 11) หลักสูตรเสริมสร้างความสามารถให้นักศึกษาสามารถประยุกต์ใช้วิธีการคำนวณและการคิดวิพากษ์ผ่านการสร้างแบบจำลองทางคณิตศาสตร์เพื่อนำไปประยุกต์ใช้การแก้ปัญหาสิ่งแวดล้อม การวางผัง ตลอดจนการจัดการทรัพยากร ซึ่งช่วยให้เกิดการพัฒนาเมืองอัจฉริยะที่ใช้ทรัพยากรอย่างมีประสิทธิภาพ ลดผลกระทบต่อสิ่งแวดล้อม และปรับปรุงคุณภาพชีวิตของผู้อยู่อาศัย โดยการผสานเทคโนโลยีการคำนวณเข้ากับการลงมือจริงจริงในห้องปฏิบัติการ บัณฑิตจะมีส่วนร่วมในการสร้างสภาพแวดล้อมเมืองที่น่าอยู่

นอกจากนี้ หลักสูตรยังเน้นความสำคัญของความร่วมมือระดับโลก (เป้าหมายที่ 17) โดยส่งเสริมการทำงานร่วมกันระหว่างสถาบันการศึกษา อุตสาหกรรม และหน่วยงานรัฐบาล ผ่านโครงการสหวิทยาการและความคิดริเริ่มด้านการวิจัย นักศึกษาจะได้มีส่วนร่วมในการแลกเปลี่ยนความรู้และการแก้ปัญหาแบบร่วมมือกันในระดับโลก ซึ่งไม่เพียงแต่เพิ่มพูนประสบการณ์การศึกษา แต่ยังมีส่วนในการสร้างความร่วมมือที่แข็งแกร่งที่จำเป็นสำหรับการบรรลุ SDGs ซึ่งขยายผลกระทบของหลักสูตรต่อการพัฒนาที่ยั่งยืนทั่วโลก

หลักสูตรวิทยาศาสตรมหาบัณฑิต สาขาวิทยาการเชิงคำนวณ มีความสำคัญอย่างยิ่งในการแก้ไขและลดช่องว่างในสถานการณ์ปัจจุบันของโลก โดยเฉพาะอย่างยิ่งในประเทศไทย ในยุคที่ปัญญาประดิษฐ์ (AI) และการเรียนรู้ของเครื่อง (Machine Learning) มีบทบาทสำคัญในการขับเคลื่อนนวัตกรรมและเศรษฐกิจ การค้นหาอัลกอริทึมและตัวปรับแต่งใหม่ ๆ โดยเฉพาะในแนวทางการเรียนรู้เชิงลึก (Deep Learning) เป็นสิ่งจำเป็นในการพัฒนาระบบที่มีประสิทธิภาพและชาญฉลาด

หลักสูตรนี้มุ่งเน้นการผลิตบัณฑิตที่มีความเชี่ยวชาญในการพัฒนาแบบจำลอง อัลกอริทึม และตัวปรับแต่ง (Optimizers) ในสาขาการเรียนรู้ของเครื่องซึ่งจะช่วยในการแก้ไขปัญหาที่ซับซ้อนและท้าทายที่ประเทศไทยและโลกกำลังเผชิญ เช่น การวิเคราะห์ข้อมูลขนาดใหญ่ การประมวลผลภาษาธรรมชาติ และการประยุกต์ใช้ปัญญาประดิษฐ์ในภาคอุตสาหกรรมต่าง ๆ ด้วยการเสริมสร้างความรู้และทักษะในด้านนี้ บัณฑิตจะสามารถสร้างสรรค์นวัตกรรมที่มีผลกระทบสูง และสนับสนุนการตัดสินใจที่มีข้อมูลเป็นฐาน

นอกจากนี้ หลักสูตรยังช่วยเสริมสร้างศักยภาพของประเทศไทยในการเป็นผู้พัฒนานวัตกรรมด้านปัญญาประดิษฐ์ โดยการส่งเสริมการวิจัยและพัฒนาในสาขาที่กำลังเติบโตนี้ ด้วยการสนับสนุนให้นักศึกษามีความคิดสร้างสรรค์และความสามารถในการพัฒนาเทคโนโลยีใหม่ ๆ ประเทศไทยจะสามารถเพิ่มขีดความสามารถในการแข่งขันบนเวทีโลก และตอบสนองต่อความต้องการของตลาดแรงงานที่ต้องการบุคลากรที่มีทักษะสูงในด้านนี้

เพื่อเสริมสร้างความสำคัญของหลักสูตรนี้ต่อแนวโน้มการเลือกศึกษาของนักศึกษา หลักสูตรวิทยาศาสตรมหาบัณฑิต สาขาวิทยาการเชิงคำนวณ ตอบสนองต่อความสนใจที่เพิ่มขึ้นของนักศึกษาในด้านเทคโนโลยีขั้นสูงในด้านปัญญาประดิษฐ์ (AI) และการเรียนรู้ของเครื่อง (Machine Learning)

แนวโน้มการเลือกศึกษาของนักศึกษาแสดงถึงความสนใจในสาขาเทคโนโลยีและการคำนวณที่เพิ่มขึ้น รายงานจากหลายแหล่งระบุว่ามีนักศึกษาสมัครเข้าเรียนในสาขาวิทยาศาสตร์คอมพิวเตอร์ วิศวกรรมคอมพิวเตอร์ และสาขาที่เกี่ยวข้องกับ AI และการเรียนรู้ของเครื่องมากขึ้น เนื่องจากเห็นถึงโอกาสในการทำงานที่กว้างขวางและความต้องการบุคลากรในตลาดแรงงานที่เพิ่มขึ้น

นักศึกษามองหาหลักสูตรที่มีความเกี่ยวข้องกับตลาดงานและมีโอกาสการทำงานสูง สาขาวิชาที่เกี่ยวข้องกับ AI การเรียนรู้ของเครื่อง และวิทยาการข้อมูล (Data Science) ถูกจัดอันดับให้เป็นสาขาที่มีศักยภาพสูง ทั้งในด้านเงินเดือนและการเติบโตในสายอาชีพ หลักสูตรที่เน้นการพัฒนาอัลกอริทึมและตัวปรับแต่งใหม่ ๆ ในการเรียนรู้ของเครื่องตรงกับความต้องการของนักศึกษาที่ต้องการความท้าทายและการสร้างสรรค์นวัตกรรม

หลักสูตรนี้ยังสอดคล้องกับแนวโน้มการศึกษาระดับสูงที่มุ่งเน้นการวิจัยและพัฒนาเทคโนโลยีใหม่ นักศึกษาที่สนใจในการแก้ไขปัญหาที่ซับซ้อนและการมีส่วนร่วมในการพัฒนาเทคโนโลยีที่มีผลกระทบสูง จะถูกดึงดูดโดยหลักสูตรที่ให้โอกาสในการทำวิจัยและพัฒนาอัลกอริทึมใหม่ ๆ

ในปัจจุบันมีมหาบัณฑิตสำเร็จการศึกษาระดับอุดมศึกษามีจำนวนเพิ่มขึ้นทุกปีแต่ไม่ตรงกับความต้องการของตลาดแรงงานและยังมีสมรรถนะหรือคุณะลักษะอื่น ๆ ที่ไม่ตรงตามความต้องการของสถานประกอบการ ทำให้มีผู้ว่างงานอยู่จำนวนมาก จึงจำเป็นต้องให้ความสำคัญกับการพัฒนาผู้เรียนและกำลังแรงงานที่มีทักษะและคุณลักษณะที่พร้อม เพื่ิอตอบสนองต่อความต้องการของภาคส่วนต่าง ๆ โดยจะต้องมีการวิเคราะห์ความต้องการกำลังคนเพื่อวางเป้าหมายการจัดการศึกษา ด้วยเหตุนี้หลักสูตรจึงมุ่งเน้นในการพัฒนากำลังคนด้านวิทยาศาสตร์และเทคโนโลยีให้เป็นบุคคลากรที่มีความรู้ความสามารถและปรับตัวได้ทันกับยุคดิจิทัล มีทักษะการเรียนรู้ที่จำเป็นที่โลกในอนาคตต้องการ มีความรู้ความสามารถในการประยุกต์องค์ความรู้ทางด้านคณิตศาสตร์ สถิติ และคอมพิวเตอร์ เพื่อสร้างนวัตกรรมเชิงคำนวณที่ตอบโจทย์เศรษฐกิจและสังคมดิจิทัล ในการคิด วิเคราะห์ และพัฒนานวัตกรรมอย่างยั่งยืน ซึ่งจะช่วยเพิ่มขีดความสามารถในการประกอบอาชีพและการแข่งขันภายหลังสำเร็จการศึกษา








\subsection{การวางระบบผู้สอนและกระบวนการจัดการเรียนการสอน}
\subsubsection{การพิจารณากำหนดผู้สอน}
อาจารย์ผู้รับผิดชอบหลักสูตร ดำเนินการประชุมร่วมกันเพื่อวางแผนการจัดการเรียนการสอนก่อนเปิดภาคการศึกษาเพื่อกำหนดรายวิชาที่เปิดสอนและกำหนดผู้สอน กำหนดผู้รับผิดชอบการจัดทำตารางสอน รวมทั้งผู้รับผิดชอบการจัดทำ มคอ.3 และ มคอ.5 มีการจัดทำผลลัพธ์การเรียนรู้ระดับรายวิชา (CLOs) และกำหนดเกณฑ์การจัดการเรียนการสอนวิชาต่าง ๆ และวัดผลประเมินผลตามผลลัพธ์การเรียนรู้ที่กำหนดไว้ ซึ่งกำหนดอาจารย์ผู้สอนที่มีคุณสมบัติตามประกาศคณะกรรมการมาตรฐานการอุดมศึกษา เรื่อง เกณฑ์มาตรฐานหลักสูตรระดับบัณฑิตศึกษา พ.ศ. 2565 โดยพิจารณาจากคุณวุฒิสาขาวิชาที่สำเร็จการศึกษา ประสบการณ์ด้านการสอน และการทำงานวิจัยที่สอดคล้องกับสาระสำคัญในรายวิชานั้น สำหรับอาจารย์ใหม่จะจัดให้มี On the Job Training มีอาจารย์พี่เลี้ยงคอยให้คำปรึกษาอาจารย์ใหม่จะเป็นอาจารย์ร่วมสอนโดยจะจับคู่หรือสอนเป็นทีมกับอาจารย์เก่าที่มีประสบการณ์อย่างน้อย 1 ภาคการศึกษา มอบหมายให้อาจารย์ผู้รับผิดชอบแต่ละรายวิชาดำเนินการจัดทำ มคอ.3 และ มคอ.5 ให้แล้วเสร็จตามกำหนด

\subsubsection{การกำกับติดตามและตรวจสอบการจัดทำแผนการเรียนรู้ (มคอ.3)}
การจัดการเรียนการสอนอาจารย์ผู้รับผิดชอบหลักสูตรดำเนินการกำกับติดตาม และตรวจสอบการจัดทำแผนการเรียนรู้ (มคอ.3) ให้คลอบคลุมด้านข้อมูลทั่วไป จุดมุ่งหมายและวัตถุประสงค์ การจัดทำผลลัพธ์การเรียนรู้ระดับรายวิชา (CLOs) ลักษะและการดำเนินการ การพัฒนาการเรียนรู้ของนักศึกษา แผนการสอนและการประเมินผลทรัพยากรประกอบการเรียนการสอน การประเมิน และปรับปรุงการดำเนินการของรายวิชา และมอบให้อาจารย์ผู้สอนทุกคนจัดทำ มคอ.3 ให้สอดคล้องกับ มคอ.2 และกำหนดให้จัดทำ มคอ.3 ให้แล้วเสร็จก่อนเปิดภาคการศึกษา โยมีหลักการสำคัญของกรอบมาตรฐานคุณวุฒิ ดังนี้

\begin{enumerate}
   	\item มุ่งประมวลกฎเกณฑ์ต่าง ๆ ที่เกี่ยวข้อง อาทิเช่น ประกาศคณะกรรมการมาตรฐานการอุดมศึกษา เรื่อง เกณฑ์มาตรฐานหลักสูตรระดับบัณฑิตศึกษา พ.ศ. 2565 โดยทุกรายวิชาจะต้องมีการกระจายความรับผิดชอบมาตรฐานผลลัพธ์การเรียนรู้จาก มคอ.2 สู่รายวิชา (Curriculum Mapping) อย่างถูกต้อง
  
   	  	\item มุ่งเน้นผลลัพธ์การเรียนรู้ (Learning Outcome) 4 ด้าน โดยมีการกำหนดผลลัพธ์การเรียนรู้ระดับหลักสูตร (PLOs) จำนวน 10 ข้อ
   	    	
   	\item  เป็เครื่องมือการสื่อสารที่มีประสิทธิภาพเพื่อให้การจัดการเรียนการสอนสอดคล้องตามแผนที่วางไว้ในรายละเอียดของหลักสูตร
   	\end{enumerate} 

\subsubsection{การวัดผลและประเมินผลการศึกษา}
การวัดผลและประเมินผลการศึกษาเป็นไปตามข้อบังคับ\university ว่าด้วย
การศึกษาระดับบัณฑิตศึกษา พ.ศ. 2559 และส่วนแก้ไขเพิ่มเติม (ภาคผนวก ค) และให้อาจารย์ผู้สอนรายงานผลการดำเนินงานของแต่ละวิชาใน มคอ.5 และเมื่อสิ้นสุดปีการศึกษาให้รวบรวมข้อมูลจัดทำรายงาน มคอ.7

\subsubsection{การควบคุมหัวข้อวิทยานิพนธ์ให้สอดคล้องกับสาขาวิชาและความก้าวหน้าของศาสตร์}

อาจารย์ประจำหลักสูตรกำหนดหลักเกณฑ์การกำหนดหัวข้อวิทยานิพนธ์ของนักศึกษาให้มีความสอดคล้องกับสาขาวิชาและความก้าวหน้าของศาสตร์ ดังนี้
  \begin{enumerate}
   	\item หัวข้อวิทยานิพนธ์ต้องมีความสอดคล้องกับสาขาวิชา\thdegreebranch
   	\item อาจารย์ที่ปรึกษาวิทยานิพนธ์จะต้องมีคุณวุฒิและมีความเชี่ยวชาญตรงกับกลุ่มวิชาที่เปิดสอนในหลักสูตร มีคุณวุฒิเหมาะสม มีผลงานตีพิมพ์ทางวิชาการ และมีประสบการณ์ในการทำวิจัย   	
   	\item  อาจารย์ที่ปรึกษาวิทยานิพนธ์แจ้งหัวข้อและขอบเขตวิทยานิพนธ์ของนักศึกษา เสนอต่อที่ประชุมอาจารย์ประจำหลักสูตรพิจารณาก่อนขอเสนอหัวข้อวิทยานิพนธ์  	
   	\end{enumerate}  

\subsubsection{การช่วยเหลือ กำกับ ติดตาม ในการทำวิทยานิพนธ์ การตีพิมพ์ผลงานในระดับบัณฑิตศึกษา}

 คณะกรรมการบริหารหลักสูตรกำหนดแผนการติดตามการทำวิทยานิพนธ์และการตีพิมพ์ผลงานในระดับบัณฑิตศึกษา ดังนี้ 

 \begin{enumerate}
   	\item มอบหมายให้อาจารย์ที่ปรึกษาแจกคู่มือการจัดทำวิทยานิพนธ์ให้นักศึกษา   
   	\item กำหนดให้นักศึกษาจัดทำแผนการทำวิทยานิพนธ์ภายใต้การควบคุมของอาจารย์ที่ปรึกษาและเข้าพบอาจารย์ที่ปรึกษาอย่างสม่ำเสมอ
   	\item กำหนดให้อาจารย์ที่ปรึกษารายงานความก้าวหน้าของนักศึกษาต่ออาจารย์ประจำหลักสูตร  	  
   	\item อาจารย์ประจำหลักสูตรประชาสัมพันธ์ข้อมูลข่าวสารการจัดประชุมวิชาการที่เกี่ยวข้องกับสาขาวิชาหรือสัมพันธ์กับหัวข้อวิทยานิพนธ์ของนักศึกษา
   	\item กำหนดให้นักศึกษาที่ลงทะเบียนวิชาวิทยานิพนธ์ นำเสนอความก้าวหน้าการทำวิทยานิพนธ์ให้แก่อาจารย์ประจำหลักสูตรและอาจารย์ผู้สอน พร้อมทั้งส่งรายงานให้อาจารย์ประจำหลักสูตรพิจารณาอย่างน้อยภาคการศึกษาละ 1 ครั้ง  	
   	\item  กำกับการสอบวิทยานิพนธ์เป็นไปตามข้อบังคับ\university ว่าด้วยการศึกษาระดับบัณฑิตศึกษา พ.ศ. 2559 และส่วนที่แก้ไขเพิ่มเติม (ภาคผนวก ค)   	
   	 \end{enumerate}


\subsection{การประเมินผลการเรียนการสอน}
หลักสูตรกำหนดให้มีการประเมินผลการเรียนรู้ตามมาตรฐานคุณวุฒิระดับอุดมศึกษาแห่งชาติตามผลการเรียนรู้ (Learning Outcome) ทั้ง 4 ด้าน ได้แก่ 1) ความรู้ 2) ทักษะ 3) จริยธรรม และ 4) คุณลักษณะส่วนบุคคล โดยทำการประเมินผลทุกรายวิชาที่เปิดสอน และนำผลการประเมินไปวางแผนปรับปรุงและพัฒนาการจัดการเรียนการสอนเพื่อให้เกิดผลสัมฤทธิ์ทางการเรียนตามที่ได้วางแผนไว้

\section{การบริหารทรัพยากรการเรียนการสอน}
\subsection{การบริหารงบประมาณ}
สาขา\thdegreebranch เสนอของบประมาณรายจ่ายประจำปีและเงินรายได้ งบประมาณยุทธศาสตร์การพัฒนา\university ผ่านคณะวิทยาศาสตร์และเทคโนโลยี เพื่อจัดซื้อทรัพยากรการเรียนการสอนหลักสูตรวิทยาศาสตร์มหาบัณฑิต เช่น หนังสือ ตำรา สื่อการเรียนการสอน โสตทัศนูปกรณ์ วัสดุและครุภัณฑ์วิชาชีพ เป็นต้น
\subsection{ทรัพยากรการเรียนการสอนที่มีอยู่เดิม}
\university \,\, มีห้องสมุดกลางและห้องสมุดของคณะวิทยาศาสตร์และเทคโนโลยี ที่มีความพร้อมด้านหนังสือ ตำราทั่วไป และตำราเฉพาะทางในประเทศไทยและต่างประเทศ และมีการจัดห้องเรียนรู้ด้วยตนเองสืบค้นจากฐานข้อมูลที่สามารถสืบค้นได้อย่างรวดเร็ว นอกจากนี้คณะฯ มีอาคารสถานที่ วัสดุและอุปกรณ์ที่ใช้สนับสนุนการเรียนการสอนอย่างเพียงพอ โดยหลักสูตรมีห้องปฏิบัติการ รวมถึงห้องให้คำปรึกษาคณิตศาสตร์และบริการวิชาการทางคณิตศาสตร์ และครุภัณฑ์หลักที่จำเป็นในการจัดการเรียนการสอน โดยมีห้องเรียนบรรยายสำหรับนักศึกษาในหลักสูตรจำนวน 3 ห้อง และห้องปฏิบัติการจำนวน 2 ห้อง แสดงรายละเอียดดังต่อไปนี้ 

\medskip\par\noindent
\begin{tabular}{|p{0.15\textwidth}|p{0.2\textwidth}|C{0.1\textwidth}|C{0.14\textwidth}|C{0.12\textwidth}|C{0.12\textwidth}|}
\hline
\multicolumn{1}{|p{0.15\textwidth}|}{ \multirow{2}{0.15\textwidth}{ชื่ออาคาร}}  & 
\multicolumn{1}{p{0.2\textwidth}|}{ \multirow{2}{0.2\textwidth}{ชื่อห้องเรียน/\newline{}ห้องปฏิบัติการ} }  & 
\multicolumn{2}{C{0.26\textwidth}|}{ประเภทห้อง}        & ขนาด     &  ขนาด   \\ \cline{3-4}
 &  & ห้องเรียน & ห้องปฏิบัติการ & (กว้างxยาว) & ความจุ\,(คน) \\ \hline


\multirow{2}{0.15\textwidth}{คณะวิทยาศาสตร์และเทคโนโลยี ชั้น 3}  & ห้องบรรยายรวม ST1301  & \ding{51}        &      & 7.7x15.4       & 70            \\ \cline{2-6}


&  ห้องพักอาจารย์ ST1302  &         &      & {7.7x5}       &             \\ \hline

\multirow{7}{0.15\textwidth}{คณะวิทยาศาสตร์และเทคโนโลยี ชั้น 9} & ห้องปฏิบัติการ ST1905  &         & \ding{51}     & {7.7x7.7}       & 30            \\ \cline{2-6}

& ห้องปฏิบัติการ ST1908  &        & \ding{51}     & {7.7x5}       & 15            \\ \cline{2-6}
 
& ห้องพักอาจารย์ ST1909  &         &      & {7.7x15.4}       &             \\ \cline{2-6}
& ห้องบรรยายรวม ST1910  &  \ding{51}       &      & {7.7x7.7}       &    30         \\ \cline{2-6} 
& ห้องบรรยายรวม ST1911  & \ding{51}        &      & {7.7x7.7}       &    30         \\ \hline 
 
 \end{tabular}
 
 \par\medskip

หลักสูตรมีการติดตั้งซอฟต์แวร์โอเพนซอร์ส และซอฟท์แวร์ลิขสิทธิ์ที่จำเป็นบนเครื่องคอมพิวเตอร์ส่วนบุคคลในห้องปฏิบัติการคอมพิวเตอร์ ST1905 และ ST1908 ซึ่งนักศึกษาระดับบัณฑิตศึกษาสามารถใช้สำหรับการเรียนรวมไปถึงการทำวิจัยได้อย่างเพียงพอ
\subsection{การจัดหาทรัพยากรการเรียนการสอนเพิ่มเติม}
คณะฯมีการประสานงานกับสำนักวิทยบริการและเทคโนโลยีดิจิตัล ของ\university \,\, ในส่วนของการเชื่อมโยงสืบค้นข้อมูล ให้บริการอาจารย์และนักศึกษาได้ค้นคว้าโดยให้อาจารย์ผู้สอนแต่ละรายวิชามีส่วนร่วมในการเสนอแนะรายชื่ิอหนังสือและสื่อต่าง ๆ ให้กับห้องสมุดของคณะและมหาลัยเพื่อจัดซื้อต่อไป รวมทั้งจัดซื้อวัสดุและครุภัณฑ์ขั้นสูงเพื่อการวิเคราะห์ข้อมูลจากงบประมาณรายจ่ายประจำปี งบประมาณเงินได้ และจากงบประมาณตามยุทธศาสตร์ต่าง ๆ ที่\university จัดสรรให้

\subsection{การประเมินความพอเพียงของทรัพยากร}
สาขาวิชา/ผู้รับผิดชอบหลักสูตร มีการจัดทำแบบสอบถามเพื่อสำรวจความพึงพอใจและความต้องการสิ่งสนับสนุนการเรียนรู้ต่าง ๆ ของอาจารย์และนักศึกษา เช่น หนังสือ ตำรา สื่อประกอบการเรียนการสอน วัสดุ อุปกรณ์ โปรแกรมสำเร็จรูปทางคอมพิวเตอร์ และคุรุภัณฑ์ในห้องปฏิบัติการ จากนั้นนำเข้าที่ประชุมสาขา\thdegreebranch เพื่อจัดสรรงบประมาณในการจัดหาสิ่งสนับสนุนการเรียนรู้ให้เพียงพอ


\section{การบริหารคณาจารย์}
\subsection{การรับอาจารย์ใหม่}
มีการคัดเลือกอาจารย์ใหม่ตามระเบียบและหลักเกณฑ์ของมหาวิทยาลัย โดยอาจารย์ใหม่จะต้องมีวุฒิการศึกษาไม่ต่ำกว่าระดับปริญญาเอกในสาขาวิชาคณิตศาสตร์ประยุกต์ การเรียนรู้ของเครื่อง ปัญญาประดิษฐ์ วิทยาการข้อมูล หรือสาขาอื่น ๆ ที่เกี่ยวข้อง และต้องมีคะแนนทดสอบความสามารถภาษาอังกฤษได้ตามเกณฑ์ที่กำหนดไว้ในประกาศของ\university เรื่องเกณฑ์มาตรฐานความสามารถภาษาอังกฤษของอาจารย์ประจำ

 \subsection{การมีส่วนร่วมของคณาจารย์ในการวางแผน การติดตามและการทบทวนหลักสูตร}
อาจารย์ผู้รับผิดชอบหลักสูตรและอาจารย์ผู้สอนในสาขาวิชามีการประชุมร่วมกันในการวางแผนการจัดการเรียนการสอน การประเมินผล ติดตามการดำเนินงานตามแผนงาน เก็บรวบรวมข้อมูล ตลอดจนปรึกษาหารือแนวทางในการพัฒนาหลักสูตร เพื่อการผลิตมหาบัณฑิตที่มีคุณลักษณะตรงตามความต้องการของตลาดแรงงาน รวมทั้งเชิญผู้ทรงคุณวุฒิจากภายนอกทั้งภาครัฐและเอกชนมาร่วมจัดทำและวิพากษ์หลักสูตรทุกครั้งที่มีการปรับปรุงหลักสูตรตามรอบระยะเวลาของหลักสูตรหรือทุกรอบ 5 ปี เพื่อให้ได้มหาบัณฑิตที่มีคุณลักษณะที่พึงประสงค์

\subsection{การแต่งตั้งคณาจารย์พิเศษ}
สาขาวิชาฯ มีการเชิญอาจารย์พิเศษ
มาสอนในบางรายวิชาหรือบางหัวข้อที่ต้องการความรู้ที่เป็นประสบการณ์ตรงจากสถานประกอบการ หน่วยงานภาครัฐและเอกชน โดยอาจารย์พิเศษต้องมีคุณสมบัติเป็นไปตามประกาศคณะกรรมการประกันคุณภาพภายในระดัับอุดมศึกษา เรื่องหลักเกณฑ์และแนวทางปฏิบัติเกี่ยวกับการประกันคุณภาพภายในระดัับอุดมศึกษา พ.ศ. 2557 และประกาศคณะกรรมการมาตรฐานการอุดมศึกษา เรื่องเกณฑ์มาตรฐานหลักสูตรระดับบัณฑิตศึกษา พ.ศ. 2565

\section{การบริหารบุคลากรสนับสนุนการเรียนการสอน}
หลักสูตรดำเนินการร่วมกับสาขาวิชาในการจัดทำแผนพัฒนาบุคลากรสายสนับสนุนของสาขาวิชาฯ ให้สอดคล้องกับการดำเนินการของหลักสูตร และเสนอให้คณะฯ พัฒนาความรู้และความสามารถของบุคลากรส่วนกลางของคณะฯ ในการสนับสนุนการเรียนการสอน

 \section{การสนับสนุนและการให้คำแนะนำนักศึกษา}
 \subsection{การควบคุมหัวข้อวิทยานิพนธ์ให้สอดคล้องกับสาขาวิชาและความก้าวหน้าของศาสตร์}
 อาจารย์ประจำหลักสูตรกำหนดหลักเกณฑ์การกำหนดหัวข้อวิทยานิพนธ์ของนักศึกษาให้มีความสอดคล้องกับสาขาวิชาและความก้าวหน้าของศาสตร์ ดังนี้
  \begin{enumerate}
   	\item หัวข้อวิทยานิพนธ์ต้องมีความสอดคล้องกับสาขาวิชา\thdegreebranch
   	\item อาจารย์ที่ปรึกษาวิทยานิพนธ์จะต้องมีคุณวุฒิและมีความเชี่ยวชาญตรงกับกลุ่มวิชาที่เปิดสอนในหลักสูตร มีคุณวุฒิเหมาะสม มีผลงานตีพิมพ์ทางวิชาการ และมีประสบการณ์ในการทำวิจัย   	
   	\item  อาจารย์ที่ปรึกษาวิทยานิพนธ์แจ้งหัวข้อและขอบเขตวิทยานิพนธ์ของนักศึกษา เสนอต่อที่ประชุมอาจารย์ประจำหลักสูตรพิจารณาก่อนขอเสนอหัวข้อวิทยานิพนธ์  	
   	\end{enumerate}  
  
  \subsection{การช่วยเหลือ กำกับ ติดตาม ในการทำวิทยานิพนธ์ การตีพิมพ์ผลงานในระดับบัณฑิตศึกษา} 
  คณะกรรมการบริหารหลักสูตรกำหนดแผนการติดตามการทำวิทยานิพนธ์และการตีพิมพ์ผลงานในระดับบัณฑิตศึกษา ดังนี้ 
 \begin{enumerate}
   	\item มอบหมายให้อาจารย์ที่ปรึกษาแจกคู่มือการจัดทำวิทยานิพนธ์ให้นักศึกษา   
   	\item กำหนดให้นักศึกษาจัดทำแผนการทำวิทยานิพนธ์ภายใต้การควบคุมของอาจารย์ที่ปรึกษาและเข้าพบอาจารย์ที่ปรึกษาอย่างสม่ำเสมอ
   	\item กำหนดให้อาจารย์ที่ปรึกษารายงานความก้าวหน้าของนักศึกษาต่ออาจารย์ประจำหลักสูตร  	  
   	\item อาจารย์ประจำหลักสูตรประชาสัมพันธ์ข้อมูลข่าวสารการจัดประชุมวิชาการที่เกี่ยวข้องกับสาขาวิชาหรือสัมพันธ์กับหัวข้อวิทยานิพนธ์ของนักศึกษา
   	\item กำหนดให้นักศึกษาที่ลงทะเบียนวิชาวิทยานิพนธ์ นำเสนอความก้าวหน้าการทำวิทยานิพนธ์ให้แก่อาจารย์ประจำหลักสูตรและอาจารย์ผู้สอน พร้อมทั้งส่งรายงานให้อาจารย์ประจำหลักสูตรพิจารณาอย่างน้อยภาคการศึกษาละ 1 ครั้ง  	
   	\item  กำกับการสอบวิทยานิพนธ์เป็นไปตามข้อบังคับ\university ว่าด้วยการศึกษาระดับบัณฑิตศึกษา พ.ศ. 2559 และส่วนที่แก้ไขเพิ่มเติม (ภาคผนวก ค)   	 \end{enumerate}   
   
\section{ความต้องการของตลาดแรงงาน สังคม และ/หรือความพึงพอใจของผู้ใช้บัณฑิต}
หลักสูตรดำเนินการสำรวจความต้องการบุคลากรด้าน\thdegreebranch \,\,\,\,\,\,\,\,\,ตามกรอบระยะเวลาที่กำหนดเพื่อนำข้อมูลมาใช้ในการปรับปรุงหลักสูตรให้ทันสมัยทุก ๆ 1 ปี และ 4 ปี เพื่อปรับปรุงหลักสูตรให้ทันสมัยและสอดคล้องกับความต้องการของตลาดแรงงานและสังคมในปัจจุบัน และดำเนินการสำรวจความพึงพอใจของผู้ใช้บัณฑิตต่อคุณภาพของบัณฑิตที่สำเร็จการศึกษาจากหลักสูตรเป็นประจำทุกปีการศึกษา โดยนำผลการสำรวจมาพิจารณาใช้เป็นแนวทางในการปรับปรุงและพัฒนาหลักสูตร


\section{การบริหารความเสี่ยงเกี่ยวกับหลักสูตร}
\subsection{ความเสี่ยงที่อาจเกิดขึ้น}
 \begin{enumerate}  	
   	\item จำนวนนักศึกษาที่สมัครเข้าเรียนไม่เป็นไปตามแผน
   	\item นักศึกษาสำเร็จการศึกษาช้ากว่าที่หลักสูตรกำหนด
   	\item นักศึกษาสอบไม่ผ่านเกณฑ์มาตรฐานภาษาอังกฤษสำหรับจบการศึกษา
   	\item นักศึกษาตีพิมพ์บทความวิจัยหรือบทความวิชาการสำหรับจบการศึกษาช้ากว่ากำหนดหรือบทความวิจัย	ไม่ผ่านการพิจารณาเพื่อตีพิมพ์หรือเผยแพร่ในฐานข้อมูลที่กำหนด
   	\item การแต่งตั้งอาจารย์ผู้รับผิดชอบหลักสูตรไม่เป็นไปตามเกณฑ์มาตรฐานหลักสูตรระดับอุดมศึกษารวมถึงอาจารย์ผู้รับผิดชอบหลักสูตรที่จะเกษียรอายุราชการ หรือลาออก
   	\item การเกิดเหตุภัยธรรมชาติ ไฟฟ้าขัดข้อง ระบบอินเตอร์เน็ตล้มเหลว หรือโรคอุบัติใหม่ ที่ทำให้นักศึกษาไม่สามารถเรียนในรูปแบบ Onsite ได้   	
   	   	\end{enumerate}  

\subsection{ผลกระทบที่เกิดขึ้น}
 \begin{enumerate}  	
   	\item จำนวนนักศึกษาแรกเข้าของหลักสูตรไม่เป็นไปตามแผนที่หลักสูตรกำหนด
   	\item การดำเนินการจัดการเรียนการสอน การบริหารงบประมาณ และสิ่งสนับสนุนของหลักสูตรไม่เป็นไปตามแผนที่หลักสูตรกำหนด
   	\item นักศึกษาจบการศึกษาช้า หรือสอบภาษาอังกฤษไม่ผ่านตามข้อกำหนดของมหาลัยซึ่งเพิ่มค่าใช้จ่ายของนักศึกษา   
   	\item นักศึกษาใช้เวลาปรับแก้บทความวิจัยหรือบทความวิชาการ และส่งตีพิมพ์ในวารสารใหม่เพื่อให้ผ่านการพิจารณาเพื่อตีพิมพ์หรือเผยแพร่ในฐานข้อมูลที่กำหนด ทำให้นักศึกษาจบการศึกษาช้ากว่าที่กำหนด
   	\item หลักสูตรไม่ผ่านเกณฑ์มาตรฐานเนื่องจากผู้รับผิดชอบหลักสูตรไม่เป็นไปตามเกณฑ์
   	\item การเกิดเหตุภัยธรรมชาติ ไฟฟ้าขัดข้อง ระบบอินเตอร์เน็ตล้มเหลว หรือโรคอุบัติใหม่ที่ทำให้นักศึกษาไม่สามารถเรียนในรูปแบบ Onsite ได้   	
   	   	\end{enumerate} 

\subsection{การจัดการความเสี่ยง}
 \begin{enumerate}  	
   	\item จัดทำแผนกลยุทธ์ในการรับนักศึกษาให้เป็นไปตามแผนที่หลักสูตรกำหนด จัดทำแผนการศึกษาให้เหมาะกับจำนวนผู้เรียนและต้นทุนในการดำเนินการหลักสูตรในกรณีที่จำนวนนักศึกษาไม่เป็นไปตามแผนที่กำหนด  
   	\item อาจารย์ผู้รับผิดชอบหลักสูตรร่วมกับอาจารย์ที่ปรึกษา จัดทำแผนการศึกษาและกำกับติดตามความก้าวหน้าของนักศึกษาเป็นรายบุคคล ทุกภาคการศึกษา
   	\item จัดกิจกรรมเสริมหลักสูตรด้านภาษาอังกฤษหรือสนับสนุนส่งเสริมให้นักศึกษาเข้าร่วมกิจกรรมด้านภาษากับคณะและมหาวิทยาลัย รวมทั้งสนับสนุนส่งเสริมให้นักศึกษาเข้าร่วมกิจกรรมฑสื่อสารแลกเปลี่ยนความรู้ทางวิชาการด้วยภาษาอังกฤษ
   	\item จัดอบรมเทคนิคการเขียนบทความวิจัยหรือบทความวิชาการ โดยวิทยากรหรือผู้ทรงคุณวุฒิที่มีประสบการณ์สูง จัดกิจกรรมเพื่อนำเสนอแลกเปลี่ยนองค์ความรู้ระหว่างนักศึกษาและอาจารย์ประจำหลักสูตร   	
   	\item อาจารย์ผู้รับผิดชอบหลักสูตรจัดทำแผนพัฒนากำลังคนเพื่อเตรียมความพร้อมอาจารย์ประจำหลักสูตรให้มีคุณสมบัติเป็นไปตามเกณฑ์มาตรฐานที่หลักสูตรกำหนด และกำกับติดตามการพัฒนาผลงานทางวิชาการและงานวิจัยของอาจารย์ประจำหลักสูตรตลอดปีการศึกษา
   	\item มีการตรวจสอบระบบไฟฟ้าและระบบอินเตอร์เน็ตอย่างน้อยเดือนละ 1 ครั้ง และมีระบบไฟฟ้าสำรองเพื่อป้องกันความเสียหายกับเครื่องมือและอุปกรณ์ในห้องปฏิบัติการ  	
   	   	\end{enumerate} 


\section{ตัวบ่งชี้ผลการดำเนินงาน (Key Performance Indicators)}
ผลการดำเนินงานการบรรลุตามเป้าหมายตัวบ่งชี้ทั้งหมดอยู่ในเกณฑ์ดีต่อเนื่อง 2 ปี การศึกษาเพื่อติดตามการดำเนินการตาม TQF ต่อไป ทั้งนี้เกณฑ์การประเมินผ่าน คือ มีดำเนินงานตามข้อ 1-5 และอย่างน้อยร้อยละ 80 ของตัวบ่งชี้ผลการดำเนินงานที่ระบุไว้ในแต่ละปี	

\medskip\par\noindent
\begin{center}
\renewcommand{\arraystretch}{2}
\begin{longtable}{|p{0.57\textwidth}|c|c|c|c|c|}
\hline
\hfill\textbf{ดัชนีบ่งชี้ผลการดำเนินงาน}\hfill\,& \textbf{ปีที่ 1} & \textbf{ปีที่ 2} & \textbf{ปีที่ 3} & \textbf{ปีที่ 4} & \textbf{ปีที่ 5}  \\ 
\hline
\endhead
(1) อาจารย์ประจำหลักสูตรอย่างน้อยร้อยละ 80 มีส่วนร่วมใน
การประชุมเพื่อวางแผน ติดตาม และทบทวนการดำเนินงานของหลักสูตร โดยกำหนดให้มีการประชุม 4 ครั้ง/ปี 
& \ding{53} & \ding{53} & \ding{53} & \ding{53} & \ding{53}\\ 
\hline
(2) มีรายละเอียดของหลักสูตร ตามแบบ มคอ.2 ที่สอดคล้องกับ
กรอบมาตรฐานคุณวุฒิแห่งชาติและมาตรฐานหลักสูตรระดับบัณฑิต พ.ศ. 2565 
& \ding{53} & \ding{53} & \ding{53} & \ding{53} & \ding{53}\\ 
\hline
(3) มีรายละเอียดของรายวิชาและรายละเอียดของประสบการณ์
ภาคสนาม ตามแบบ มคอ.3 อย่างน้อยก่อนการเปิดสอนในแต่ละภาคการศึกษาให้ครบทุกรายวิชา
& \ding{53} & \ding{53} & \ding{53} & \ding{53} & \ding{53}\\ 
\hline
(4) จัดทำรายงานผลการดำเนินการของรายวิชา และรายงานผลการดำเนินการของประสบการณ์ภาคสนาม ตามแบบ มคอ.5 ภายใน 30 วัน หลังสิ้นสุดภาคการศึกษาที่เปิดสอนให้ครบทุกรายวิชา
& \ding{53} & \ding{53} & \ding{53} & \ding{53} & \ding{53}\\ 
\hline
(5) จัดทำรายงานผลการดำเนินการของหลักสูตร ตามแบบ มคอ.7 ภายใน 60 วันหลังสิ้นสุดปีการศึกษา
& \ding{53} & \ding{53} & \ding{53} & \ding{53} & \ding{53}\\ 
\hline
(6) มีการทวนสอบผลสัมฤทธิ์ของนักศึกษาตามมาตรฐาน
ผลการเรียนรู้ที่กำหนดใน มคอ.3 อย่างน้อยร้อยละ 25 ของรายวิชาที่เปิดสอนในแต่ละปีการศึกษา
& \ding{53} & \ding{53} & \ding{53} & \ding{53} & \ding{53}\\ 
\hline
(7) มีการพัฒนา/ปรับปรุงการจัดการเรียนการสอน กลยุทธ์
การสอนหรือการประเมินผลการเรียนรู้จากผลการประเมิน การดำเนินงานที่รายงานใน มคอ.7 ปีที่ผ่านมา
& \ding{53} & \ding{53} & \ding{53} & \ding{53} & \ding{53}\\ 
\hline
(8) อาจารย์ใหม่ทุกคนได้รับการปฐมนิเทศหรือคำแนะนำด้านการจัดการเรียนการสอน
& \ding{53} & \ding{53} & \ding{53} & \ding{53} & \ding{53}\\ 
\hline
(9) อาจารย์ประจำทุกคนได้รับการพัฒนาทางวิชาการหรือวิชาชีพ อย่างน้อยปีละ 1 ครั้ง
& \ding{53} & \ding{53} & \ding{53} & \ding{53} & \ding{53}\\ 
\hline
(10) จำนวนบุคคลากรสนับสนุนการเรียนการสอนได้รับการพัฒนาวิชาการหรือวิชาชีพ ไม่น้อยกว่าร้อยละ 50 ต่อปี
& \ding{53} & \ding{53} & \ding{53} & \ding{53} & \ding{53}\\ 
\hline
(11) ระดับความพึงพอใจของนักศึกษาปีสุดท้าย/มหาบัณฑิตใหม่
ที่มีต่อคุณภาพหลักสูตร เฉลี่ยไม่น้อยกว่า 3.5 จากคะแนน เต็ม 5.0
& \ding{53} & \ding{53} & \ding{53} & \ding{53} & \ding{53}\\ 
\hline
(12) ระดับความพึงพอใจของผู้ใช้มหาบัณฑิตใหม่ที่มีต่อมหาบัณฑิตใหม่ เฉลี่ยไม่น้อยกว่า 3.5 จากคะแนนเต็ม 5.0
& \ding{53} & \ding{53} & \ding{53} & \ding{53} & \ding{53}\\ 
\hline
\textbf{รวมตัวบ่งชี้บังคับที่ต้องดำเนินการ (ข้อ 1-5) ในแต่ละปี}
& 5 & 5 & 5 & 5 & 5\\ 
\hline
\textbf{รวมตัวบ่งชี้ในแต่ละปี (ตามที่คณะกำหนด)}
& 9 & 11 & 12 & 12 & 12\\ 
\hline
\end{longtable}
\end{center}











