\begin{longtable}{|p{0.08\linewidth}p{0.2\linewidth}|C{0.05\linewidth}|C{0.05\linewidth}|C{0.05\linewidth}|C{0.05\linewidth}|C{0.05\linewidth}|C{0.05\linewidth}|C{0.05\linewidth}|C{0.05\linewidth}|C{0.05\linewidth}|C{0.06\linewidth}|}
\hline
\multicolumn{2}{|C{0.28\linewidth}|}{ \textbf{ผลการเรียนรู้} }
 &  \textbf{PLO1} & \textbf{PLO2} & \textbf{PLO3} & \textbf{PLO4} & \textbf{PLO5} & \textbf{PLO6} & \textbf{PLO7} & \textbf{PLO8} & \textbf{PLO9} &\textbf{PLO10} \\ \hline
\endhead09-110-601 & ระเบียบวิธีวิจัยการหาค่าเหมาะที่สุดเชิงคำนวณและการเรียนรู้ของเครื่อง & & & \ding{108}& & & & & \ding{108}& & \ding{108}\\ \hline
09-110-602 & สัมมนา & & & & & & & & \ding{108}& & \ding{108}\\ \hline
09-111-601 & สถิติและความน่าจะเป็นสำหรับการเรียนรู้ของเครื่อง & & \ding{108}& & & & & & & & \ding{108}\\ \hline
09-111-602 & คณิตศาสตร์สำหรับการเรียนรู้ของเครื่อง & \ding{108}& \ding{108}& & & & & & & & \ding{108}\\ \hline
09-111-603 & การเรียนรู้ของเครื่อง & \ding{108}& \ding{108}& & & & & & & & \ding{108}\\ \hline
09-111-704 & การหาค่าเหมาะที่สุดสำหรับการเรียนรู้ของเครื่อง & \ding{108}& & & & & \ding{108}& & & & \ding{108}\\ \hline
09-111-705 & โครงสร้างข้อมูลและอัลกอริทึมสำหรับการเรียนรู้ของเครื่อง & \ding{108}& & & & & \ding{108}& & & & \ding{108}\\ \hline
09-112-601 & การวิเคราะห์เชิงฟังก์ชัน & & \ding{108}& & & & & & & & \ding{108}\\ \hline
09-112-702 & ทฤษฎีจุดตรึงและการประยุกต์ & & \ding{108}& & & & & & & & \ding{108}\\ \hline
09-113-601 & คณิตศาสตร์ขั้นสูงสำหรับการเรียนรู้ของเครื่อง & & \ding{108}& & & & & & & & \ding{108}\\ \hline
09-113-702 & ขั้นตอนวิธีเชิงตัวเลขสำหรับค่าเหมาะที่สุด  & \ding{108}& \ding{108}& & & & \ding{108}& & & & \ding{108}\\ \hline
09-113-603 & การตัดสินใจอย่างชาญฉลาดด้วยกำหนดการวิจัยดำเนินงาน & \ding{108}& & & & & \ding{108}& & & & \ding{108}\\ \hline
09-113-704 & หัวข้อพิเศษของคณิตศาสตร์เชิงคำนวณ  & \ding{108}& \ding{108}& & & & \ding{108}& & & & \ding{108}\\ \hline
09-114-701 & การเรียนรู้เชิงลึกและการประยุกต์   & \ding{108}& & & & & \ding{108}& & & & \ding{108}\\ \hline
09-114-702 & วิศวกรรมการเรียนรู้ของเครื่องและการดึงข้อมูล & \ding{108}& & & & & \ding{108}& & & & \ding{108}\\ \hline
09-114-603 & การวิเคราะห์ข้อมูล & \ding{108}& & & & & \ding{108}& & & & \ding{108}\\ \hline
09-114-604 & การทำให้เห็นข้อมูล & \ding{108}& & & & & \ding{108}& & & & \ding{108}\\ \hline
09-114-705 & การประยุกต์ใช้การเรียนรู้ของเครื่องในงานด้านการประมวลภาษาธรรมชาติ & \ding{108}& \ding{108}& & & & \ding{108}& \ding{108}& & & \ding{108}\\ \hline
09-114-706 & การประยุกต์ใช้การเรียนรู้ของเครื่องในงานด้านประมวลผลภาพและสัญญาณ & \ding{108}& \ding{108}& & & & \ding{108}& \ding{108}& & & \ding{108}\\ \hline
09-114-707 & การประยุกต์ใช้การเรียนรู้ของเครื่องในงานด้านการแพทย์  & \ding{108}& \ding{108}& & & & \ding{108}& \ding{108}& & & \ding{108}\\ \hline
09-114-708 & การประยุกต์ใช้การเรียนรู้ของเครื่องในด้านธุรกิจและการเงิน & \ding{108}& \ding{108}& & & & \ding{108}& \ding{108}& & & \ding{108}\\ \hline
09-114-709 & หัวข้อพิเศษของการเรียนรู้ของเครื่อง  & \ding{108}& \ding{108}& & & & \ding{108}& & & & \ding{108}\\ \hline
09-115-701 & สารนิพนธ์ & \ding{108}& \ding{108}& \ding{108}& \ding{108}& \ding{108}& \ding{108}& \ding{108}& \ding{108}& \ding{108}& \ding{108}\\ \hline
09-115-702 & วิทยานิพนธ์ & \ding{108}& \ding{108}& \ding{108}& \ding{108}& \ding{108}& \ding{108}& \ding{108}& \ding{108}& \ding{108}& \ding{108}\\ \hline
09-115-703 & วิทยานิพนธ์ & \ding{108}& \ding{108}& \ding{108}& \ding{108}& \ding{108}& \ding{108}& \ding{108}& \ding{108}& \ding{108}& \ding{108}\\ \hline
\end{longtable}
