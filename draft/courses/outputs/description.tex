\begin{longtable}{p{0.15\textwidth}p{0.6\textwidth}r{0.15\textwidth}}
09-110-601 & ระเบียบวิธีวิจัย & 3(3-0)\\*
 & Research Methodology & \phantom{x} \vspace{3mm} \\*
&  \multicolumn{2}{p{0.75\textwidth}}{กระบวนการทำวิจัย ประเภทการวิจัย การกำหนดปัญหาการวิจัย การทบทวนวรรณกรรม การสร้างข้อคาดการณ์หรือสมมติฐานการวิจัย การเขียนโครงร่างและรายงานการวิจัย การอ้างอิงผลงาน การนำเสนอผลงานวิจัยจรรยาบรรณของนักวิจัย เทคนิควิธีการวิจัยเฉพาะทางการหาค่าเหมาะที่สุดเชิงคำนวณและการเรียนรู้ของเครื่อง ภาษาอังกฤษเพื่อการเขียนงานวิจัย } \vspace{3mm} \\*
&  \multicolumn{2}{p{0.75\textwidth}}{Research process, research types, research problem determination, literature review; conjecture or assumption construction, proposal and research report writing, reference writing, ethics of researchers, research techniques in computational optimization and machine learning, English for research writing.} \vspace{8mm} \\*
09-110-602 & สัมมนา & 1(0-0)\\*
 & Seminar & \phantom{x} \vspace{3mm} \\*
&  \multicolumn{2}{p{0.75\textwidth}}{ศึกษาค้นคว้าบทความที่อยู่ในฐานข้อมูลทางวิทยาศาสตร์ นำเสนอผลการวิจัย วิเคราะห์ อภิปราย สรุปผล ตั้ง คำถามและตอบคำถามจากผู้ร่วมสัมมนาได้ นักศึกษาต้องเขียนรายงานและนำเสนอต่อคณะกรรมการของสาขาวิชา} \vspace{3mm} \\*
&  \multicolumn{2}{p{0.75\textwidth}}{Seminar on articles selected from scientific journals focusing on topics concerning computational optimization and machine learning, the students are obliged to analyze, summaries, give an oral presentation, discuss, and answer the questions, required written report and presentation the selected topics.} \vspace{8mm} \\*
09-111-601 & สถิติและความน่าจะเป็นสำหรับการเรียนรู้ของเครื่อง & 3(3-0)\\*
 & Statistics and Probability for Machine Learning & \phantom{x} \vspace{3mm} \\*
&  \multicolumn{2}{p{0.75\textwidth}}{ทฤษฎีพื้นฐานในสถิติสำหรับการเรียนรู้ของเครื่อง ความน่าจะเป็น ตัวแปรสุ่มแบบไม่ต่อเนื่อง ตัวแปรสุ่มแบบต่อเนื่อง การแจกแจงร่วม ค่าคาดหวัง ค่าคาดหวังแบบมีเงื่อนไข ทฤษฎีลิมิตทางสถิติ การประมาณค่าพารามิเตอร์ การประมาณภาวะน่าจะเป็นสูงสุด วิธีการแบบเบย์ในการประมาณค่าพารามิเตอร์ การทดสอบสมมติฐาน ช่วงความเชื่อมั่น กระบวนการเฟ้นสุ่ม} \vspace{3mm} \\*
&  \multicolumn{2}{p{0.75\textwidth}}{Basic theories in statistics for machine learning, probability, discrete random variables, continuous random variables, joint distributions, expectation, conditional expectation, statistical limit theorems, estimation of parameters, maximum likelihood estimation, Bayesian approach to parameter estimation, hypothesis testing, confidence intervals, random processes.} \vspace{8mm} \\*
09-111-602 & คณิตศาสตร์สำหรับการเรียนรู้ของเครื่อง & 3(3-0)\\*
 & Mathematics for Machine Learning & \phantom{x} \vspace{3mm} \\*
&  \multicolumn{2}{p{0.75\textwidth}}{เมทริกซ์และการดำเนินการบนเมทริกซ์ ระบบสมการเชิงเส้นและการหาผลเฉลย ปริภูมิเวกเตอร์ ความเป็นอิสระเชิงเส้น ฐานหลัก ฐานหลักเชิงตั้งฉาก การแปลงเชิงเส้น ค่าเจาะจงและเวกเตอร์เจาะจง การทำให้เป็นเมทริกซ์ทแยงมุม นอร์ม ผลคูณภายใน ความยาวและระยะทาง ส่วนประกอบเชิงตั้งฉาก การแยกเมทริกซ์ การแยกโชเลสกี การประมาณค่าเมทริกซ์} \vspace{3mm} \\*
&  \multicolumn{2}{p{0.75\textwidth}}{Matrices and matrix algebra, system of linear equation and solving systems of linear equations, vector space, linear Independence, basis, orthonormal basis, linear transformation, eigen value and eigen vector, diagonalization of matrices, norm, inner product, lengths and distances, orthogonal complement, matrix decompositions, Cholesky decomposition, matrix approximation.} \vspace{8mm} \\*
09-111-603 & การเรียนรู้ของเครื่องแบบมีผู้สอน & 3(2-2)\\*
 & Supervised Machine Learning & \phantom{x} \vspace{3mm} \\*
&  \multicolumn{2}{p{0.75\textwidth}}{แนวคิดและหลักการของการเรียนรู้ของเครื่องแบบมีผู้สอน การเตรียมข้อมูล ขั้นตอนการเรียนรู้ เช่น การถดถอยเชิงเส้น การถดถอยเชิงเส้นพหุคูณ การถดถอยโลจีสติกส์ ย่านใกล้เคียงที่สุดเค เบย์อย่างง่าย ต้นไม้ตัดสินใจ การทดสอบประสิทธิภาพตัวแบบ การใช้ตัวแบบไปประยุกต์ใช้ในการพยากรณ์และการจำแนกข้อมูล} \vspace{3mm} \\*
&  \multicolumn{2}{p{0.75\textwidth}}{Concepts and principles of supervised machine learning, data preparation, learning algorithm, such as linear regression, multiple linear regression, logistic regression, k-nearest neighbors, decision tree. Model evaluation, application model to forecasting and data classification.} \vspace{8mm} \\*
09-111-704 & การหาค่าเหมาะที่สุดสำหรับการเรียนรู้ของเครื่อง & 3(3-0)\\*
 & Optimization for Machine Learning & \phantom{x} \vspace{3mm} \\*
&  \multicolumn{2}{p{0.75\textwidth}}{ทฤษฎีพื้นฐานของปัญหาการหาค่าเหมาะที่สุด ปัญหาการหาค่าเหมาะที่สุดแบบมีข้อจำกัด ปัญหาการหาค่าเหมาะที่สุดแบบไม่มีข้อจำกัด ปัญหาการหาค่าเหมาะที่สุดแบบปรับเรียบ และไม่ปรับเรียบ อัลกอริทึมค่าเหมาะที่สุดอันดับหนึ่ง อัลกอริทึมค่าเหมาะที่สุดอันดับสอง อัลกอริทึมเคลื่อนลงตามความชันสโตแคสติก อัลกอริทึมเคลื่อนลงแบบใกล้เคียง การใช้โปรแกรมไพธอนในการพัฒนาอัลกอริทึม} \vspace{3mm} \\*
&  \multicolumn{2}{p{0.75\textwidth}}{Basic theories of optimization, constrained optimization, unconstrained optimization, smooth and nonsmooth optimization, first-order optimization algorithms, second-order optimization algorithms, stochastic gradient descent algorithm, proximal gradient method, algorithm implementation in Python.} \vspace{8mm} \\*
09-111-705 & โครงสร้างข้อมูลและอัลกอริทึมสำหรับการเรียนรู้ของเครื่อง & 3(3-0)\\*
 & Data Structures and Algorithms for Machine Learning & \phantom{x} \vspace{3mm} \\*
&  \multicolumn{2}{p{0.75\textwidth}}{แนวคิดของโครงสร้างข้อมูล โครงสร้างข้อมูลเบื้องต้น การดำเนินการบนโครงสร้างข้อมูล เทคนิคการค้นและเทคนิคการเรียงลำดับ การวิเคราะห์โครงสร้างข้อมูล การประยุกต์และอัลกอริทึมสำหรับการแก้ปัญหาในกระบวนการของการเรียนรู้ของเครื่อง} \vspace{3mm} \\*
&  \multicolumn{2}{p{0.75\textwidth}}{Concepts of data structures, fundamental data structures, operations of data structures, basic searching and sorting techniques, data structure analysis, applications and problem solving algorithms for machine learning processes.} \vspace{8mm} \\*
09-112-601 & การวิเคราะห์เชิงฟังก์ชัน & 3(3-0)\\*
 & Functional Analysis & \phantom{x} \vspace{3mm} \\*
&  \multicolumn{2}{p{0.75\textwidth}}{ปริภูมิเมตริก ปริภูมินอร์ม ปริภูมิบานาค ตัวดำเนินการเชิงเส้น ปริภูมิผลคูณภายในและปริภูมิฮิลเบิร์ต ทฤษฎีบทฮาห์น-บานาค ทฤษฎีบทการมีขอบเขตแบบเอกรูป ปริภูมิคู่กัน} \vspace{3mm} \\*
&  \multicolumn{2}{p{0.75\textwidth}}{Metric space, normed space, Banach spaces, linear operator, inner product and Hilbert spaces, Hahn-Banach theorem, uniform boundedness theorem, dual space.} \vspace{8mm} \\*
09-112-702 & ทฤษฎีจุดตรึงและการประยุกต์ & 3(3-0)\\*
 & Fixed Point Theory and Applications & \phantom{x} \vspace{3mm} \\*
&  \multicolumn{2}{p{0.75\textwidth}}{ทฤษฎีจุดตรึงในปริภูมิเมตริก ทฤษฎีจุดตรึงในปริภูมิฮิลเบิร์ต ทฤษฎีจุดตรึงในปริภูมิบานาค การทำซ้ำเพื่อหาจุดตรึง} \vspace{3mm} \\*
&  \multicolumn{2}{p{0.75\textwidth}}{Fixed point theory in metric space, fixed point theory in Hilbert space, fixed point theory in Banach space, fixed point iteration.} \vspace{8mm} \\*
09-113-601 & คณิตศาสตร์ขั้นสูงสำหรับการเรียนรู้ของเครื่อง & 3(3-0)\\*
 & Advanced Mathematics for Machine Learning & \phantom{x} \vspace{3mm} \\*
&  \multicolumn{2}{p{0.75\textwidth}}{แคลคูลัสสำหรับการเรียนรู้ของเครื่อง: ฟังก์ชันหลายตัวแปร ลิมิตและความต่อเนื่อง อนุพันธ์ของฟังก์ชันหลายตัวแปร กฎลูกโซ่ จาโคเบียน เกรเดียนของฟังก์ชันค่าเวกเตอร์ เกรเดียนของเมทริกซ์ อนุพันธ์อันดับสูง ทฤษฎีของเทย์เลอร์ โครงข่ายประสาทเทียม ฟังก์ชันกระตุ้น ฟังก์ชันการสูญเสีย อัลกอริทึมเพิ่มประสิทธิภาพเพื่อปรับปรุงความแม่นยำของเครือข่ายประสาทเทียม สมการเชิงอนุพันธ์สามัญและทฤษฎีพื้นฐาน การพยากรณ์ข้อมูลด้วยแบบจำลองของสมการเชิงอนุพันธ์สามัญ สมการเชิงอนุพันธ์ย่อยและทฤษฎีพื้นฐาน แบบจำลองของสมการเชิงอนุพันธ์ย่อย การพยากรณ์ข้อมูลด้วยแบบจำลองของสมการเชิงอนุพันธ์ย่อย} \vspace{3mm} \\*
&  \multicolumn{2}{p{0.75\textwidth}}{Calculus for machine learning: multivariable functions, limit and continuity, derivative of multivariable functions, chain rule, Jacobian, gradient of vector-valued function, gradient of matrices, high order derivatives and Taylor’s Theorem. Neural networks: activation functions, loss function, backpropagation algorithm. Ordinary differential equations (ODEs) and basic theory: Prediction with model of ODEs. Partial differential equations (PDEs) and basic theory: Model of PDEs, prediction with model of PDEs.} \vspace{8mm} \\*
09-113-702 & ขั้นตอนวิธีเชิงตัวเลขสำหรับค่าเหมาะที่สุด  & 3(2-2)\\*
 & Numerical Algorithm for Optimization & \phantom{x} \vspace{3mm} \\*
&  \multicolumn{2}{p{0.75\textwidth}}{ทฤษฎีค่าเหมาะที่สุดในปริภูมิฮิลเบิร์ตและปริภูมิบานาค ขั้นตอนวิธีสำหรับจุดตรึง วิธีอินเนอร์เทียล ปัญหาอสมการเชิงแปรผัน ปัญหาดุลยภาพ ปัญหารวมแบบกึ่ง ปัญหาเป็นไปได้แบบแยก การสร้างขั้นตอนวิธีเพื่อหาผลเฉลยของปัญหาค่าเหมาะที่สุด} \vspace{3mm} \\*
&  \multicolumn{2}{p{0.75\textwidth}}{Optimization in Hilbert and Banach spaces, algorithm for fixed point, inertial method, variational inequality problem, equilibrium problem, quasi-inclusion problem, split feasibility problem, construction algorithm for solution of optimization problems.} \vspace{8mm} \\*
09-113-603 & การตัดสินใจอย่างชาญฉลาดด้วยกำหนดการวิจัยดำเนินงาน & 3(2-2)\\*
 & Intelligence Decision Making with Operation Research & \phantom{x} \vspace{3mm} \\*
&  \multicolumn{2}{p{0.75\textwidth}}{กำหนดการเชิงเส้น: ตัวแบบกำหนดการเชิงเส้นและการหาผลเฉลยโดยวิธีกราฟ หลักการของวิธีซิมเพล็กซ์ ปัญหาควบคู่และการวิเคราะห์ความไว หลักการของวิธีซิมเพล็กซ์ควบคู่ ปัญหาการขนส่ง ปัญหาเครือข่าย ปัญหาการลงทุน กำหนดการพลวัต การพัฒนาแอพพลิเคชั่นช่วยตัดสินใจในการแก้ปัญหาโดยใช้การวิจัยดำเนินงาน} \vspace{3mm} \\*
&  \multicolumn{2}{p{0.75\textwidth}}{Linear programming: linear programming model and graphical solution, principles of the simplex method, dual problem and sensitivity analysis, principles of the dual simplex method, transportation models and its applications, logistics problems, network problems, investment problems, dynamic programming, development to operation research as an application to assist decision making.} \vspace{8mm} \\*
09-113-704 & หัวข้อพิเศษของคณิตศาสตร์เชิงคำนวณ  & 3(3-0)\\*
 & Special Topic in Computational Mathematics for Machine Learning & \phantom{x} \vspace{3mm} \\*
&  \multicolumn{2}{p{0.75\textwidth}}{ความก้าวหน้าเชิงทฤษฎีและการประยุกต์คณิตศาสตร์เชิงคำนวณสำหรับการเรียนรู้ของเครื่อง เรื่องเฉพาะแปรเปลี่ยนตามความสนใจของผู้สอนและนักศึกษา ซึ่งสอดคล้องกับ ความก้าวหน้าทางวิทยาศาสตร์และเทคโนโลยีในปัจจุบัน} \vspace{3mm} \\*
&  \multicolumn{2}{p{0.75\textwidth}}{Theoretical advances and applications of computational mathematics for machine learning, specific topics based on contemporary advances in science and technology, interests of individual instructor and students.} \vspace{8mm} \\*
09-114-601 & การเรียนรู้ของเครื่องแบบไม่มีผู้สอน & 3(2-2)\\*
 & Unsupervised Machine Learning & \phantom{x} \vspace{3mm} \\*
&  \multicolumn{2}{p{0.75\textwidth}}{แนวคิดและหลักการของการเรียนรู้ของเครื่องแบบไม่มีผู้สอน การจัดกลุ่มข้อมูล การจัดกลุ่มแบบค่าเฉลี่ยเค การหากฎความสัมพันธ์ ปัจจัยสนับสนุนนและปัจจัยความเชื่อมั่น ขั้นตอนวิธีแบบนิรนัย การใช้ตัวแบบไปประยุกต์ใช้ในการจัดกลุ่มข้อมูล} \vspace{3mm} \\*
&  \multicolumn{2}{p{0.75\textwidth}}{Concepts and principles of unsupervised machine learning, data clustering, k-means clustering. Association data: support and confident factors, apriori algorithm, application model to data clustering.} \vspace{8mm} \\*
09-114-702 & การเรียนรู้ของเครื่องแบบเสริมแรง & 3(2-2)\\*
 & Reinforcement Machine Learning & \phantom{x} \vspace{3mm} \\*
&  \multicolumn{2}{p{0.75\textwidth}}{แนวคิดและหลักการของการเรียนรู้ของเครื่องแบบเสริมแรง การทดลองของตัวแทน การให้รางวัลและการลงโทษ การเรียนรู้คิว โครงข่ายคิวเชิงลึก การตัดสินใจแบบต่อเนื่องกัน การควบคุมหุ่นยนต์ การจัดการจราจร และการวางกลยุทธ์ในเกม} \vspace{3mm} \\*
&  \multicolumn{2}{p{0.75\textwidth}}{Concepts and principles of reinforcement machine learning, agent, reward and punishment, Q-learning, deep Q-network, sequential decision-making, robot control, traffic control and gameplay strategy.} \vspace{8mm} \\*
09-114-703 & การเรียนรู้เชิงลึกและการประยุกต์   & 2(2-0)\\*
 & Deep Learning and Applications & \phantom{x} \vspace{3mm} \\*
&  \multicolumn{2}{p{0.75\textwidth}}{พื้นฐานโครงข่ายประสาทเทียม การเรียนรู้แบบป้อนหน้าและการแพร่กลับ การใช้งานเฟรมเวิร์กเช่น TensorFlow หรือ PyTorch โครงข่ายประสาทแบบคอนโว ลูชัน โครงข่ายประสาทแบบหมุนเวียน เทคนิคป้องกันการฟิตเกิน การปรับจูนไฮ เปอร์พารามิเตอร์ การประยุกต์ใช้ในการประมวลผลภาพ และการประมวลผลภาษาธรรมชาติ} \vspace{3mm} \\*
&  \multicolumn{2}{p{0.75\textwidth}}{Fundamentals of neural networks, feed-forward and backpropagation learning, implementation with frameworks such as tensor flow or PyTorch, convolutional neural networks, recurrent neural networks, techniques to prevent overfitting, hyperparameter tuning, applications in image processing, and natural language processing.} \vspace{8mm} \\*
09-114-704 & วิศวกรรมการเรียนรู้ของเครื่อง & 2(2-0)\\*
 & Machine Learning Engineering & \phantom{x} \vspace{3mm} \\*
&  \multicolumn{2}{p{0.75\textwidth}}{หลักการของวิศวกรรมการเรียนรู้ของเครื่อง กระบวนการและการออกแบบการเรียนรู้ของเครื่อง การพัฒนาและการปรับใช้การเรียนรู้ของเครื่อง การดึงข้อมูลสำหรับการเรียนรู้ของเครื่อง การพัฒนาเว็บแอพพลิเคชั่นเพื่องานการเรียนรู้ของเครื่อง } \vspace{3mm} \\*
&  \multicolumn{2}{p{0.75\textwidth}}{Principles of Machine learning engineering, process and design of machine learning, machine learning development and deployment, data scraping for machine learning, machine learning web application development.} \vspace{8mm} \\*
09-114-605 & การวิเคราะห์ข้อมูล & 2(2-0)\\*
 & Data Analytics & \phantom{x} \vspace{3mm} \\*
&  \multicolumn{2}{p{0.75\textwidth}}{เทคโนโลยีในการจัดการเก็บและวิเคราะห์ข้อมูลขนาดใหญ่ การจัดกลุ่มข้อมูล การวิเคราะห์เส้นทาง การวิเคราะห์ปัจจัย และตัวแบบสมการโครงสร้าง การใช้โปรแกรมสำเร็จรูปหรือภาษาโปรแกรมในการวิเคราะห์} \vspace{3mm} \\*
&  \multicolumn{2}{p{0.75\textwidth}}{Technologies used in manipulating, storing, and analyzing big data, clustering data, path analysis, factor analysis and structural equation model, utilization of software packages or programming language for analysis.} \vspace{8mm} \\*
09-114-606 & การทำให้เห็นข้อมูล & 2(2-0)\\*
 & Data Visualization & \phantom{x} \vspace{3mm} \\*
&  \multicolumn{2}{p{0.75\textwidth}}{ระบบพิกัดและแกน สเกลสี การแสดงภาพจำนวน การแสดงภาพการกระจาย การแสดงภาพสัดส่วน การแสดงภาพอนุกรมเวลา การแสดงภาพแนวโน้ม การแสดงภาพความไม่แน่นอน หลักการออกแบบภาพ เช่น หลักการของน้ำหมึกตามสัดส่วน และการจัดการจุดที่ทับซ้อนกัน} \vspace{3mm} \\*
&  \multicolumn{2}{p{0.75\textwidth}}{Coordinate systems and axes, color scales, visualizing amounts, visualizing distributions, visualizing proportions, visualizing time series, visualizing trends, visualizing uncertainty, principles of figure design, the principle of proportional ink, and handling overlapping points.} \vspace{8mm} \\*
09-114-707 & แบบจำลองภาษาขนาดใหญ่ & 2(2-0)\\*
 & Large Language Model & \phantom{x} \vspace{3mm} \\*
&  \multicolumn{2}{p{0.75\textwidth}}{ความหมายและลักษณะของแบบจำลองภาษาขนาดใหญ่ การประมวลผลข้อความล่วงหน้า การวิเคราะห์ความหมายและไวยากรณ์ การแยกคุณสมบัติ รูปแบบ TF-IDF รูปแบบของคำและเอกสารที่เป็นเวกเตอร์ การรู้จําและการสังเคราะห์เสียง การประยุกต์ LLM ในการจำแนก การสกัดข้อมูล การขุดและการดึงข้อความ การประยุกต์การเรียนรู้เชิงลึกใน LLM โครงข่ายประสาทเทียมใน LLM} \vspace{3mm} \\*
&  \multicolumn{2}{p{0.75\textwidth}}{Definition and characteristic of large language model (LLM), text pre-processing, semantic and grammatical analysis, features extraction, TF-IDF model, word and document vectors, speech recognition and synthesis, application of LLM to classification, information extraction, text mining and information retrieval, application of deep learning to LLM, neural networks in LLM.} \vspace{8mm} \\*
09-114-708 & การประยุกต์ใช้การเรียนรู้ของเครื่องในงานด้านประมวลผลภาพและสัญญาณ & 2(2-0)\\*
 & Applications of Machine Learning in Image and Signal Processing & \phantom{x} \vspace{3mm} \\*
&  \multicolumn{2}{p{0.75\textwidth}}{การประยุกต์ใช้การเรียนรู้ของเครื่องในงานด้านประมวลผลภาพและสัญญาณ เช่น การใช้การเรียนรู้ของเครื่องสำหรับวิเคราะห์ภาพถ่ายและสัญญาณ ปัญหาภาพเบลอ ภาพเบลอแบบเกาส์เซียน ภาพเบลอแบบเคลื่อนไหว ภาพเบลอแบบหลุดความสนใจ การกำจัดสัญญาณรบกวน การบีบอัดภาพและสัญญาณ เทคนิคการประมวลผลภาพล่วงหน้าและการเพิ่มภาพ } \vspace{3mm} \\*
&  \multicolumn{2}{p{0.75\textwidth}}{Applications of Machine Learning in Image and signal processing: machine learning for analyze image and signal. Image deblurring problem, Gaussian blur, motion blur, out of focus blur, noise reduction, image and signal compression, image preprocessing and augmentation techniques.} \vspace{8mm} \\*
09-114-709 & การประยุกต์ใช้การเรียนรู้ของเครื่องในงานด้านการแพทย์  & 2(2-0)\\*
 & Applications of Machine Learning in Medical & \phantom{x} \vspace{3mm} \\*
&  \multicolumn{2}{p{0.75\textwidth}}{การประยุกต์ใช้การเรียนรู้ของเครื่องในด้านการแพทย์ เช่น การใช้การเรียนรู้ของเครื่องสำหรับการวินิจฉัยโรค การใช้การเรียนรู้ของเครื่องสำหรับการระบุชนิดของโรค การวิเคราะห์ข้อมูลขนาดใหญ่ทางการแพทย์} \vspace{3mm} \\*
&  \multicolumn{2}{p{0.75\textwidth}}{Application of machine learning in the medical: machine learning for disease diagnosis and identifying the types of diseases, data analytic of medical big data.} \vspace{8mm} \\*
09-114-710 & การประยุกต์ใช้การเรียนรู้ของเครื่องในด้านธุรกิจและการเงิน & 2(2-0)\\*
 & Applications of Machine Learning in Business and Finance & \phantom{x} \vspace{3mm} \\*
&  \multicolumn{2}{p{0.75\textwidth}}{การประยุกต์ใช้การเรียนรู้ของเครื่องในงานด้านธุรกิจและการเงิน เช่น การใช้การเรียนรู้ของเครื่องสำหรับการวิเคราะห์แนวโน้มของหุ้น การใช้การเรียนรู้ของเครื่องสำหรับการตัดสินใจทางธุรกิจ} \vspace{3mm} \\*
&  \multicolumn{2}{p{0.75\textwidth}}{Application of machine learning in business and finance: machine learning for stock trend analysis and making business decisions.} \vspace{8mm} \\*
09-114-711 & หัวข้อพิเศษของการเรียนรู้ของเครื่อง  & 2(2-0)\\*
 & Special Topic in Machine Learning & \phantom{x} \vspace{3mm} \\*
&  \multicolumn{2}{p{0.75\textwidth}}{ความก้าวหน้าของการประยุกต์ใช้การเรียนรู้ของเครื่อง เรื่องเฉพาะแปรเปลี่ยนตามความสนใจของผู้สอนและนักศึกษา ซึ่งสอดคล้องกับ ความก้าวหน้าทางวิทยาศาสตร์และเทคโนโลยีในปัจจุบัน } \vspace{3mm} \\*
&  \multicolumn{2}{p{0.75\textwidth}}{Theoretical advances for applications of machine learning, specific topics based on contemporary advances in science and technology, interests of individual instructor and students.} \vspace{8mm} \\*
09-115-701 & สารนิพนธ์ & 6(0-0)\\*
 & Independent Study & \phantom{x} \vspace{3mm} \\*
&  \multicolumn{2}{p{0.75\textwidth}}{นักศึกษาที่จะทำสารนิพนธ์จะต้องผ่านวิชาบังคับในหลักสูตรอย่างน้อย 10 หน่วยกิต หรือตามที่ภาควิชาฯ เห็นชอบ หัวข้อสารนิพนธ์จะต้องได้รับการเห็นชอบจากอาจารย์ที่ปรึกษาและภาควิชาฯ และต้องเป็นหัวข้อที่เกี่ยวข้องกับเนื้อหาวิชาที่ได้เรียนมาในหลักสูตร } \vspace{3mm} \\*
&  \multicolumn{2}{p{0.75\textwidth}}{Students are expected to complete at least 10 credits of study with approval from advisors. This must be related with the subject or knowledge, which students have learned from the courses.} \vspace{8mm} \\*
09-115-702 & วิทยานิพนธ์ & 12(0-0)\\*
 & Thesis & \phantom{x} \vspace{3mm} \\*
&  \multicolumn{2}{p{0.75\textwidth}}{นักศึกษาต้องทำวิทยานิพนธ์ภายใต้คำแนะนำของอาจารยที่ปรึกษาที่ได้รับการแต่ง ตั้งโดยบัณฑิตวิทยาลัย นักศึกษาต้องปฏิบัติตามกูฏและข้อบังคับที่กำหนดโดยภาควิชาและบัณฑิตวิทยาลัยอย่างเคร่งครัด} \vspace{3mm} \\*
&  \multicolumn{2}{p{0.75\textwidth}}{Students are required to conduct a thesis under supervision of advisors appointed by graduate college. Rules and regulations for under taking thesis set by students’ department and graduate college must be observed strictly.} \vspace{8mm} \\*
09-115-703 & วิทยานิพนธ์ & 36(0-0)\\*
 & Thesis & \phantom{x} \vspace{3mm} \\*
&  \multicolumn{2}{p{0.75\textwidth}}{นักศึกษาต้องทำวิทยานิพนธ์ภายใต้คำแนะนำของอาจารยที่ปรึกษาที่ได้รับการแต่ง ตั้งโดยบัณฑิตวิทยาลัย นักศึกษาต้องปฏิบัติตามกูฏและข้อบังคับที่กำหนดโดยภาควิชาและบัณฑิตวิทยาลัยอย่างเคร่งครัด} \vspace{3mm} \\*
&  \multicolumn{2}{p{0.75\textwidth}}{Students are required to conduct a thesis under supervision of advisors appointed by graduate college. Rules and regulations for under taking thesis set by students’ department and graduate college must be observed strictly.} \vspace{8mm} \\*
\end{longtable}
