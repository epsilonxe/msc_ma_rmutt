\begin{longtable}{p{0.15\textwidth}p{0.6\textwidth}r{0.15\textwidth}}
09-112-101 & การวิเคราะห์เชิงฟังก์ชัน & 3(3-0-9)\\*
 & Functional Analysis & \phantom{x} \vspace{3mm} \\*
&  \multicolumn{2}{p{0.75\textwidth}}{ปริภูมิเมตริก ปริภูมินอร์ม ปริภูมิบานาค ตัวดำเนินการเชิงเส้น ปริภูมิผลคูณภายในและปริภูมิฮิลเบิร์ต ทฤษฎีบทฮาห์น-บานาค ทฤษฎีบทการมีขอบเขตแบบเอกรูป ปริภูมิคู่กัน} \vspace{3mm} \\*
&  \multicolumn{2}{p{0.75\textwidth}}{Metric space, normed space, Banach spaces, linear operator, inner product and Hilbert spaces, Hahn-Banach theorem, uniform boundedness theorem, dual space} \vspace{8mm} \\*
09-112-204 & สมการเชิงอนุพันธ์และระบบพลวัต & 3(3-0-9)\\*
 & Differential Equation and Dynamical System & \phantom{x} \vspace{3mm} \\*
&  \multicolumn{2}{p{0.75\textwidth}}{ทฤษฎีบทการมีอยู่จริงและมีเพียงหนึ่งเดียวของผลเฉลย ระบบเชิงอนุพันธ์เชิงเส้นและไม่เชิงเส้น ระบบออโตโนมัส จุดดุลยภาพของระบบสมการไม่เชิงเส้น เสถียรภาพของผลเฉลย การวิเคราะห์ไบเฟอร์เคชัน การทำให้เป็นเชิงเส้น} \vspace{3mm} \\*
&  \multicolumn{2}{p{0.75\textwidth}}{Existence and uniqueness theorem of solution, linear and nonlinear differential system, autonomous system, equilibrium point of nonlinear system equation, stability of solution, bifurcation analysis, linearization} \vspace{8mm} \\*
09-112-205 & สมการเชิงอนุพันธ์เศษส่วน & 3(3-0-9)\\*
 & Fractional Differential Equations & \phantom{x} \vspace{3mm} \\*
&  \multicolumn{2}{p{0.75\textwidth}}{ฟังก์ชันพิเศษ แคลคูลัสเชิงเศษส่วน การแปลงปริพันธ์ของปริพันธ์และอนุพันธ์เชิงเศษส่วน สมการเชิงอนุพันธ์เชิงเศษส่วนเชิงเส้นและไม่เชิงเส้น ระบบเศษส่วนของสมการเชิงอนุพันธ์} \vspace{3mm} \\*
&  \multicolumn{2}{p{0.75\textwidth}}{Special functions, fractional calculus, Integral transform of fractional integral and derivative, linear and nonlinear fractional differential equations, fractional system of differential equation} \vspace{8mm} \\*
09-112-206 & สมการเชิงปริพันธ์ & 3(3-0-9)\\*
 & Integral Equation & \phantom{x} \vspace{3mm} \\*
&  \multicolumn{2}{p{0.75\textwidth}}{สมการปริพันธ์เฟรดโฮล์ม เคอเนลแบบทั่วไปและแบบเฮลมินตัน สมการปริพันธ์โวเทอร์รา สมการเชิงอนุพันธ์และสมการปริพันธ์เชิงอนุพันธ์ สมการปริพันธ์ไม่เชิงเส้น สมการปริพันธ์เอกฐาน ระบบเชิงเส้นและไม่เชิงเส้นของสมการปริพันธ์} \vspace{3mm} \\*
&  \multicolumn{2}{p{0.75\textwidth}}{Fredholm integral equation, general and Hermitian kernels, Voltera integral equation, differential equation and integral-differential equation, nonlinear integral equation, singular integral equation, linear and nonlinear systems of integral equation} \vspace{8mm} \\*
09-112-102 & ทฤษฎีจุดตรึงและการประยุกต์ & 3(3-0-9)\\*
 & Fixed Point Theory and Applications & \phantom{x} \vspace{3mm} \\*
&  \multicolumn{2}{p{0.75\textwidth}}{ทฤษฎีจุดตรึงในปริภูมิเมตริก ทฤษฎีจุดตรึงในปริภูมิฮิลเบิร์ต ทฤษฎีจุดตรึงในปริภูมิบานาค การทำซ้ำเพื่อหาจุดตรึง} \vspace{3mm} \\*
&  \multicolumn{2}{p{0.75\textwidth}}{Fixed point theory in metric space, fixed point theory in Hilbert space, fixed point theory in Banach space, fixed point iteration} \vspace{8mm} \\*
09-112-207 & วิธีเชิงคำนวณสำหรับสมการเชิงอนุพันธ์ & 3(3-0-9)\\*
 & Computational Method for Differential Equation & \phantom{x} \vspace{3mm} \\*
&  \multicolumn{2}{p{0.75\textwidth}}{ผลเฉลยเชิงตัวเลขของสมการเชิงอนุพันธ์เชิงเส้นและไม่เชิงเส้น การแปลงอนุพันธ์และการแปลงอินทิกรัลเพื่อหาผลเฉลยของสมการเชิงอนุพันธ์; วิธีวิเคราะห์โฮโมโทฟี; วิธีแยกส่วน} \vspace{3mm} \\*
&  \multicolumn{2}{p{0.75\textwidth}}{Solution of linear and nonlinear differential equation, differential transform and integral transform for solution of differential equation, homotopy analysis method, decomposition method} \vspace{8mm} \\*
09-112-103 & ขั้นตอนวิธีเชิงตัวเลขสำหรับค่าเหมาะที่สุด & 3(3-0-9)\\*
 & Numerical Algorithm for Optimization & \phantom{x} \vspace{3mm} \\*
&  \multicolumn{2}{p{0.75\textwidth}}{ทฤษฎีค่าเหมาะที่สุดในปริภูมิฮิลเบิร์ตและปริภูมิบานาค ขั้นตอนวิธีสำหรับจุดตรึง วิธีอินเนอร์เทียล ปัญหาอสมการเชิงแปรผัน ปัญหาดุลยภาพ ปัญหารวมแบบกึ่ง ปัญหาเป็นไปได้แบบแยก การสร้างขั้นตอนวิธีเพื่อหาผลเฉลยปัญหาค่าเหมาะที่สุด} \vspace{3mm} \\*
&  \multicolumn{2}{p{0.75\textwidth}}{Optimization in Hilbert and Banach spaces, algorithm for fixed point, inertial method, variational inequality problem, equilibrium problem, quasi-inclusion problem, split feasibility problem, construction algorithm for solution of optimization problems} \vspace{8mm} \\*
09-112-208 & วิธีเชิงตัวเลขสำหรับสมการเชิงอนุพันธ์สามัญ & 3(3-0-9)\\*
 & Numerical Methods for Ordinary Differential Equations & \phantom{x} \vspace{3mm} \\*
&  \multicolumn{2}{p{0.75\textwidth}}{ปัญหาค่าเริ่มต้น วิธีขั้นเดียว วิธีหลายขั้นเชิงเส้น วิธีกาเลอร์คิน ปัญหาค่าขอบ วิธีผลต่างอันตะ} \vspace{3mm} \\*
&  \multicolumn{2}{p{0.75\textwidth}}{Initial value problems, one-step methods, linear multistep methods, Galerkin method, Boundary value problems, finite difference method} \vspace{8mm} \\*
09-112-209 & สมการเชิงอนุพันธ์ย่อย & 3(3-0-9)\\*
 & Partial Differential Equations & \phantom{x} \vspace{3mm} \\*
&  \multicolumn{2}{p{0.75\textwidth}}{แนวคิดเบื้องต้น สมการเชิงอนุพันธ์ย่อยอันดับหนึ่ง การจำแนกสมการเชิงอนุพันธ์ย่อยอันดับสอง ปัญหาค่าขอบและปัญหาค่าเริ่มต้น สมการพาราโบลิก สมการไฮเพอร์โบลิก สมการอิลิปติก} \vspace{3mm} \\*
&  \multicolumn{2}{p{0.75\textwidth}}{Basic concepts, first order partial differential equations, classification of second order equations, boundary and initial value problems, parabolic equations, hyperbolic equations, elliptic equations} \vspace{8mm} \\*
09-112-210 & วิธีเชิงตัวเลขสำหรับสมการเชิงอนุพันธ์ย่อย & 3(3-0-9)\\*
 & Numerical Methods for Partial Differential Equations & \phantom{x} \vspace{3mm} \\*
&  \multicolumn{2}{p{0.75\textwidth}}{บทนำสำหรับสมการเชิงอนุพันธ์ย่อย วิธีผลต่างอันตะสำหรับปัญหาในรูปแบบเชิงวงรี วิธีสมาชิกจำกัดสำหรับปัญหาในรูปแบบเชิงวงรี วิธีเชิงตัวเลขสำหรับปัญหาใน รูปแบบพาราโบลา ขั้นตอนสำหรับวิธีเชิงตัวเลขของปัญหาในรูปแบบไฮเพอร์โบลา} \vspace{3mm} \\*
&  \multicolumn{2}{p{0.75\textwidth}}{Introduction to partial differential equations, finite difference method for elliptic problems, finite element methods for elliptic problems, numerical methods for parabolic problems, procedure for numerical methods for hyperbolic problems} \vspace{8mm} \\*
09-112-211 & การเรียนรู้ของเครื่องเบื้องต้น & 3(3-0-9)\\*
 & Introduction to Machine Learning & \phantom{x} \vspace{3mm} \\*
&  \multicolumn{2}{p{0.75\textwidth}}{หลักการพื้นฐานของการเรียนรู้ของเครื่อง เช่น การถดถอยเชิงเส้น การจำแนกข้อมูล และการประยุกต์ใช้สำหรับปัญหาจริง} \vspace{3mm} \\*
&  \multicolumn{2}{p{0.75\textwidth}}{Fundamental principles of machine learning, including linear regression, classification, and practical applications} \vspace{8mm} \\*
09-112-212 & การเรียนรู้เชิงลึก & 3(3-0-9)\\*
 & Deep Learning & \phantom{x} \vspace{3mm} \\*
&  \multicolumn{2}{p{0.75\textwidth}}{หลักการของโครงข่ายประสาทและการเรียนรู้เชิงลึก เทคนิคในการปรับปรุงโครงข่ายประสาท การปรับให้เป็นระเบียบและการเพิ่มประสิทธิภาพ การปรับค่าไฮเปอร์พารามิเตอร์ และกรอบการเรียนรู้เชิงลึก โครงข่ายประสาทเทียมแบบสังวัตนาการและการประยุกต์ โครงข่ายประสาทเทียมแบบวนซ้ำและ การประยุกต์ โครงข่ายความขัดแย้งเชิงกำเนิด การเสริมสร้างการเรียนรู้เชิงลึก การโจมตีฝ่ายตรงข้าม} \vspace{3mm} \\*
&  \multicolumn{2}{p{0.75\textwidth}}{Principles of neural network and deep learning, technique to improve neural network: regularization and optimization, hyperparameter tuning and deep learning framework, convolutional neural network and application, recurrent neural network and application, generative adversarial network, deep reinforcement learning, adversarial attack} \vspace{8mm} \\*
09-112-213 & การออกแบบและวิเคราะห์ขั้นตอนวิธี & 3(3-0-9)\\*
 & Algorithm Design and Analysis & \phantom{x} \vspace{3mm} \\*
&  \multicolumn{2}{p{0.75\textwidth}}{ความซับซ้อนของขั้นตอนวิธี การวิเคราะห์กรณีเฉลี่ยและกรณีสูงสุด การจัดอันดับ ค่าสูงสุดและค่าต่ำสุดในกลุ่ม ขั้นตอนวิธีกราฟ กำหนดการพลวัต ขั้นตอนวิธีพหุนามเวลา ขั้นตอนวิธีเอ็นพี ความบริบูรณ์เอ็นพี และขั้นตอนวิธีแบบขนาน} \vspace{3mm} \\*
&  \multicolumn{2}{p{0.75\textwidth}}{Complexity of algorithm; analysis of mean and maximum cases, ordering, maximum and minimum in group, graph algorithm, dynamic programming, polynomial-time algorithm, NP algorithm, NP completeness and parallel algorithm} \vspace{8mm} \\*
09-112-214 & หัวข้อพิเศษทางด้านวิทยาศาตร์เชิงคำนวณ & 3(3-0-9)\\*
 & Special Topics in Computational Science & \phantom{x} \vspace{3mm} \\*
&  \multicolumn{2}{p{0.75\textwidth}}{หัวข้อครอบคลุมพัฒนาการร่วมสมัยในเรื่องที่เกี่ยวกับวิทยาศาตร์เชิงคำนวณ} \vspace{3mm} \\*
&  \multicolumn{2}{p{0.75\textwidth}}{Typical content includes contemporary development in computational science.} \vspace{8mm} \\*
\end{longtable}
