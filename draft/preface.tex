\cleardoublepage
\chapter*{คำนำ}

\faculty{} และคณะกรรมการพัฒนาหลักสูตรได้ดำเนินการพัฒนาหลักสูตร\thdegree{} สาขาวิชา\thdegreebranch{} (หลักสูตรใหม่ พ.ศ. 2569) เพื่อพัฒนาคนให้มีความรู้ด้านคณิตศาสตร์ สถิติ และคอมพิวเตอร์ เพื่อสร้างนวัตกรรมเชิงคำนวณที่ตอบโจทย์เศรษฐกิจและสังคมดิจิทัล และผลิตบัณฑิตที่มีศักยภาพในการคิด วิเคราะห์ และพัฒนานวัตกรรมอย่างยั่งยืน ที่ช่วยการยกระดับขีดความสามารถในการแข่งขันและแก้ปัญหาให้กับประเทศไทยได้สอดคล้องกับเป้าหมายของมหาวิทยาลัยเทคโนโลยีราชมงคลธัญบุรี ที่มุ่งเน้นการพัฒนากำลังคนทางวิทยาศาสตร์และเทคโนโลยีให้เป็นที่ยอมรับทั้งในระดับชาติและระดับสากล นอกจากนี้คณะกรรมการพัฒนาหลักสูตรได้ใช้แนวทาง Outcome Based Learning ที่เน้นผลลัพธ์การเรียนรู้ซึ่งสอดคล้องกับความต้องการของผู้ใช้บัณฑิตและการพัฒนาประเทศมาเป็นแนวทางในการพัฒนาหลักสูตรนี้ โดยมีวัตถุประสงค์ผลิตบุคลากรที่มีความเชี่ยวชาญในการพัฒนาแบบจำลอง อัลกอริทึม และตัวปรับแต่ง (Optimizers) ในสาขาการเรียนรู้ของเครื่องซึ่งจะช่วยในการแก้ไขปัญหาที่ซับซ้อนและท้าทายที่ประเทศไทยและโลกกำลังเผชิญ เช่น การวิเคราะห์ข้อมูลขนาดใหญ่ การประมวลผลภาษาธรรมชาติ และการประยุกต์ใช้ปัญญาประดิษฐ์ในภาคอุตสาหกรรมต่าง ๆ ต่อยอดงานวิจัยและสร้างองค์ความรู้ใหม่ มีทักษะทางด้านเทคโนโลยีเพื่องานวิจัย ควบคู่กับจริยธรรมและจรรยาบรรณวิชาชีพ การสื่อสารทางวิชาการ และการทำงานในสังคมพหุวัฒนธรรมได้ คณะกรรมการพัฒนาหลักสูตร \faculty{} หวังเป็นอย่างยิ่งว่าหลักสูตรฉบับนี้จะเป็นประโยชน์ในการผลิตมหาบัณฑิต สาขาวิชาการเรียนรู้ของเครื่องประยุกต์ ที่มีคุณลักษณะตามเป้าหมายของหลักสูตร และเกิดประโยชน์ต่อการพัฒนาประเทศต่อไป

\vspace{5mm}\par
\begin{flushright}
\faculty\\
\university
\end{flushright}